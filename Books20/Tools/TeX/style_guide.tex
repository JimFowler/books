%%
%%
%% style_guide.tex
%%
%%   The style that I will used for 
%%   the book project "Some Important Books in Astronomy
%%    and Astrophysics in the 20th Century"
%%
%%   Copyright 2017 James R. Fowler
%%
%%   All rights reserved. No part of this publication may be
%%   reproduced, stored in a retrival system, or transmitted
%%   in any form or by any means, electronic, mechanical,
%%   photocopying, recording, or otherwise, without prior written
%%   permission of the author.
%%
%%
%%
\documentclass{article}

\usepackage{books20}

\begin{document}
\title{Style Guide for the \PTitle\ Project}
\author{James R. Fowler}
\date{2017-12-15}

\maketitle

\section{General Comments}

Documents for the \PTitle\ project should be written in \LaTeX\ and use
the books20.sty package Most documents will be articles but the
manuscript should be in the books document class. The reasoning behind
using \LaTeX\ is that I expect academic publishers to prefer or at
least have available a \LaTeX\ class package.  It will be difficult if
I need to convert these documents to Microsoft Word\texttrademark or
some other formatting package.

The books20 style package imports the following packages

\begin{itemize}
\item hyperref.sty
\item inputenc.sty with the utf8 option
\item longtable.sty
\end{itemize}


\subsection{Document Design}

\begin{itemize}
  
\item The \LaTeX\ style files books20.sty shall be used in all
  \LaTeX\ documents. To include it use \verb|\usepackage{books20}|, there
  are no options to this style file at this time. Note that books20.sty
  is designed for \LaTeXe\ and may not work with \LaTeX\ 2.09.
  

\item Place sections in a separate file and use an include command in the
  document file. Then it is possible to use the section file in a
  different document with out having to clean up the document commands.

\item {The input encoding package inputenc.sty shall be used so that
  UTF-8 character encoding can be utilized. Note that this package is
  included automatically with the books20.sty file.

  You can used UTF-8 characters by using the inputenc.sty
  package.  Then a sentence might be written as
  
  \verb| Das astronomische Weltbild gemäß jüngster Forshung|
  to produce

  Das astronomische Weltbild gemäß jüngster Forshung

  But you can still use the standard \LaTeXe\ encodings
  
  \verb| Das astronomische Weltbild gem\"{a}{\ss} j\"{u}ngster Forshung|
  to produce the same thing.

  Das astronomische Weltbild gem\"{a}{\ss} j\"{u}ngster Forshung

  This will be useful when I create \LaTeXe\ files from the database
  which is in UTF-8 encoding.
  }
  
\end{itemize}

\subsection{Specifics}

\begin{itemize}

\item Dashes and Hypens --- The dash `-' is used for compound words and for
  hypenation at the end of a line.  The en-dash `--' is used for a range
  of numbers. The em-dash `---' is used for punctuation,
  e.g.\ separating phrases.

\item Date format --- Use the
  \href{https://en.wikipedia.org/wiki/ISO_8601}{ISO-8601} style
  YYYY-MM-DD for dates and hh:mm:ss for time.

\item Tables should be in longtable style and should have a caption with two
  hlines after the heading and at the end.  All documents should include
  the package \verb|\usepackage{longtable}|.

\item Use the serial comma --- for example, we write this, that, and the
  other things.

\item Books title shall be in \bt{italics}. Use the command
  \verb|\bt{My Book Title}| to produce \bt{My Book Title} in a
  sentence. Note that \verb|\bt{}| can be changed in books20.sty
  to produce another style if necessary.

\end{itemize}


\section{Defined command in books20.sty}

\begin{itemize}

\item \verb|\Cen{19}| is used to produce century names,
  e.g.\ \Cen{19}. Note that it works only for centuries that use the
  \textsuperscript{th} format.  That is, it won't work for the
  21\textsuperscript{st} Century, the 2\textsuperscript{nd} Century,
  or the 3\textsuperscript{rd} Century, etc. I should probably fix
  this issue.

\item \verb|\bt{Book Title}| produces book titles in italic face,
e.g.\ \bt{Book Title} is the title of a book.

\item \verb|\PTitle\ | produces the project title in bold face,
  e.g.\ the \PTitle\ project is the name of this project.

\item \verb|\BTitle\ | produces the book title in italic face,
  e.g.\ \BTitle\ or whatever I (or my editor) chooses for a title.

\end{itemize}

\section{An Example Document}

Here is an example document so you can see what the \LaTeX\ commands are.

\begin{verbatim}

%%
%%
%% style_guide.tex
%%
%%   The style that I will used for 
%%   the book project "Some Important Books in Astronomy
%%    and Astrophysics in the 20th Century"
%%
%%   Copyright 2017 James R. Fowler
%%
%%   All rights reserved. No part of this publication may be
%%   reproduced, stored in a retrival system, or transmitted
%%   in any form or by any means, electronic, mechanical,
%%   photocopying, recording, or otherwise, without prior written
%%   permission of the author.
%%
%%
%%
\documentclass{article}

% Should these be included in books20.sty ??
\usepackage{hyperref}
\usepackage{longtable}
\usepackage[utf8]{inputenc}

\usepackage{books20}

\begin{document}
\title{Style Guide for the \PTitle\ Project}
\author{James R. Fowler}
\date{2017-12-15}

\maketitle
\tableofcontents
\listoftables 

\include{my_section}

\end{document}
%
% The following would be included in the my_section.tex document
%
\section{First Section}

The first of many sections.

\subsection{Useing UTF-8}

You can used UTF-8 characters by using the inputenc.sty
package.  Then a sentence might be written as

Das astronomische Weltbild gemäß jüngster Forshung

But you can still use the standard \LaTeXe\ encodings

Der neuentdeckte Himmel; Das astronomische Weltbild gem\"{a}{\ss} j\'{u}ngster Forshung

You can make a table in longtable format.

\begin{longtable}[p]{l l l}
  \caption{\bf Harvard Observatory Monographs} \\
  \label{HMA:1} \\

  Title & Author(s) & Date \\
  \hline\hline
  \endfirsthead

  \multicolumn{3}{c}{Continuation of Harvard Obs.\ Monographs} \\
  Title & Author(s) & Date \\
  \hline\hline
  \endhead

  \hline
  \endfoot
  
  \hline\hline
  \endlastfoot

  1 \bt{Stellar Atmospheres} & Cecilia Payne & 1925 \\

  2 \bt{Star Clusters} & Harlow Shapley & 1930 \\

  3 \bt{The Stars of High Luminosity} & Cecilia Payne & 1930 \\

  
\end{longtable}

\end{verbatim}



\end{document}


