%%
%%
%% style_guide.tex
%%
%%   The style that I will used for 
%%   the book project "Some Important Books in Astronomy
%%    and Astrophysics in the 20th Century"
%%
%%   Copyright 2017 James R. Fowler
%%
%%   All rights reserved. No part of this publication may be
%%   reproduced, stored in a retrival system, or transmitted
%%   in any form or by any means, electronic, mechanical,
%%   photocopying, recording, or otherwise, without prior written
%%   permission of the author.
%%
%%
%%
\documentclass{article}

\usepackage{books20}
\addbibresource{localbib.bib}

\begin{document}
\title{Style Guide for the \ProjectTitle\ Project}
\author{James R. Fowler}
\date{2017-12-15\\ updated 2018-07-29}

\maketitle

\section{General Comments}

Documents for the \ProjectTitle\ project should be written in \LaTeXe\ and use
the books20.sty package. Most documents will be articles but the
manuscript should be in the books document class. The reasoning behind
using \LaTeXe\ is that I expect academic publishers to prefer or at
least have available a \LaTeXe\ class package.  It will be difficult if
I need to convert these documents to Microsoft Word\texttrademark or
some other formatting package.

A good, general purpose on-line reference for \LaTeXe\ can be found at
the Share \LaTeX\ web site,
\href{http://www.sharelatex.com/learn}{www.sharelatex.com/learn}.
More detailed information can be found at The Comprehensive
\TeX\ Archive Network, \href{ctan.org}{ctan.org}

The \texttt{books20} style package imports the following packages so they are
not required in your \LaTeXe\ documents

\begin{itemize}
\item inputenc.sty \cite{Jeffrey2018} (with the utf8 option)
\item hyperref.sty \cite{Rahtz2017}
\item longtable.sty \cite{Carlisle2014}
\item import.sty \cite{Arseneau2009}
\item biblatex.sty \cite{Lehman2018} (with biber backend and MasterBib.bib as the master
  bibliography file)
\end{itemize}


\subsection{Document Design}

\begin{itemize}
  
\item The \LaTeXe\ style files books20.sty shall be used in all
  \LaTeXe\ documents. To include it use \verb|\usepackage{books20}|, there
  are no options to this style file at this time. Note that books20.sty
  is designed for \LaTeXe\ and may not work with \LaTeX\ 2.09.
  

\item The \texttt{import.sty} allows you to place sections in a
  separate file and use the \verb|\import{}{}| command in the document
  file. The first argument is the directory and the second argument is
  the file name. Then it is possible to use the same text in a section
  or chapter in different documents without having to clean up the
  document commands. The \verb|\subimport{}{}| command has the same
  calling syntax but the file locations are relative to the first
  \verb|\import|. It is possible to nest \verb|\import| commands where
  it is not possible to next \verb|\include| commands. This feature shall be
  used in documents to allow greater flexibility in reusing 

\item {The input encoding package \texttt{inputenc.sty} shall be used
  so that UTF-8 character encoding can be utilized. Note that this
  package is included automatically with the books20.sty file.

  You can used UTF-8 characters by using the \texttt{inputenc.sty}
  package.  Then a sentence might be written as
  
  \verb| Das astronomische Weltbild gemäß jüngster Forshung|
  to produce

  Das astronomische Weltbild gemäß jüngster Forshung

  But you can still use the standard \LaTeX\ encoding
  
  \verb| Das astronomische Weltbild gem\"{a}{\ss} j\"{u}ngster Forshung|
  to produce the same thing.

  Das astronomische Weltbild gem\"{a}{\ss} j\"{u}ngster Forshung

  This will be useful when I create \LaTeXe\ files from the database
  which is in UTF-8 encoding.
}

\item In Emacs we can produce UTF-8 characters with C-x 8 mark letter.
  For example, to produce ö use the key strokes C-x 8 \verb|"| o. Use C-x 8 C-h
  to get a list of character encoding.
  
\item Use the \texttt{biblatex} package rather than the
  \texttt{bibtex} package. This is provided in the \texttt{books20}
  package.  For example, we can cite this book by Basbanes
  \cite{basbanes:2003} from the global file as well as this web site
  with the documentation for the \texttt{biblatex} package \cite{Lehman2018}
  from the local file for further information. The file
  \texttt{./Docs/MasterBib.bib} is the global bibliography file. You
  may use additional local biblography files as well with this
  package.  The choice of which references should be global and which
  should be local will probably change with time.

  Need to work out how to handle link resources and references in
  footnotes, particular URLs in footnotes. Should all URLs be at the back
  of the document in the bibliography?  This is possible if we use local
  .bib files
  
\item May wish to consider the use of the \texttt{subfiles} or
  \texttt{standalone} packages if the number of files gets
  large. These packages allow you to precompile files and avoid long
  compiles, simlar to the \texttt{make} command in Unix.  However,
  these packages may conflict with the \texttt{import} package.


\end{itemize}

\subsection{Specifics}

\begin{itemize}

\item Dashes and Hyphens --- The dash `-' is used for compound-words and for
  hyphenation at the end of a line.  The en-dash `--' is used for a range
  of numbers, 1 -- 10. The em-dash `---' is used for punctuation ---
  e.g.\ separating phrases.

\item Date format --- Use the ISO-8601\cite{isotime} style
  YYYY-MM-DD for dates and hh:mm:ss for time. Use YYYY or YYYY-MM
  and hh:mm for abbreviated year-month and hours-minutes.

\item Tables should be in longtable style and should have a caption
  with two hlines after the heading and at the end. The
  \texttt{longtable} package is loaded in the \texttt{books20}
  package.

\item You may also use the functions in table.py in a python script to create
  a table in \LaTeX\ longtable format.  I'll write some documentation someday
  but for now you can look at ./Doc/Series/assl\_table.py.

\item Use the serial comma --- for example, we write this, that, and the
  other things.

\item Books title shall be in \bt{italics}. Use the command
  \verb|\bt{My Book Title}| to produce \bt{My Book Title} in a
  sentence. Note that \verb|\bt{}| can be changed in \texttt{books20}
  packages to produce another style if necessary.

 Other font choices?

\end{itemize}


\section{Defined command in books20.sty}

\begin{itemize}

\item \verb|\Cen{19}| is used to produce century names,
  e.g.\ \Cen{19}. Note that it works only for centuries that use the
  \textsuperscript{th} format.  That is, it won't work for the
  21\textsuperscript{st} Century, the 2\textsuperscript{nd} Century,
  or the 3\textsuperscript{rd} Century, etc. I should probably fix
  this issue.

\item \verb|\bt{Book Title}| produces book titles in italic face,
e.g.\ \bt{Book Title} is the title of a book.

\item \verb|\ProjectTitle\ | produces the project title in bold face,
  e.g.\ the \ProjectTitle\ project is the name of this project.

\item \verb|\BookTitle\ | produces the book title in italic face,
  e.g.\ \BookTitle\ or whatever I (or my editor) chooses for a title.

\end{itemize}

\section{Still to Do}

\begin{itemize}

\item Determine a good (useful) format style for citations and the
  bibliography.  Then figure out how to do it in \texttt{biblatex}.

\item Work out other font formats as required. For example, an included
  file or package should be in teletype format using \verb|\texttt{}|.
  Although there will not be much call for such in the manuscript.

\item Convert series.tex, handbuch.tex, the files in \texttt{./Docs}, as
  well as the manuscript to use the \texttt{books20} package.

\end{itemize}

\section{An Example Document}

Here is an example document so you can see what the \LaTeX\ commands are.

\begin{verbatim}

%%
%%
%% style_guide.tex
%%
%%   The style that I will used for 
%%   the book project "Some Important Books in Astronomy
%%    and Astrophysics in the 20th Century"
%%
%%   Copyright 2017 James R. Fowler
%%
%%   All rights reserved. No part of this publication may be
%%   reproduced, stored in a retrival system, or transmitted
%%   in any form or by any means, electronic, mechanical,
%%   photocopying, recording, or otherwise, without prior written
%%   permission of the author.
%%
%%
%%
\documentclass{article}

\usepackage{books20}
\addbibresource{localbib.bib}

\begin{document}
\title{Style Guide for the \ProjectTitle\ Project}
\author{James R. Fowler}
\date{2017-12-15}

\maketitle
\tableofcontents
\listoftables 

\section{First Section}
\import{my_section}

\end{document}

%
% The following would be included in the my_section.tex document
%

The first of many sections for the book \BookTitle.  You can cite 
Basbanes \cite{basbanes:2003} book on libraries like this.

\subsection{Using UTF-8}

You can use UTF-8 characters by using the \texttt{inputenc}
package.  Then a sentence might be written as

Das astronomische Weltbild gemäß jüngster Forshung

But you can still use the standard \LaTeXe\ encodings

Der neuentdeckte Himmel; Das astronomische Weltbild gem\"{a}{\ss} j\'{u}ngster Forshung

\subsection{Long Tables}

You can make a table in longtable format.

\begin{longtable}[p]{l l l}
  \caption{\bf Harvard Observatory Monographs} \\
  \label{HMA:1} \\

  Title & Author(s) & Date \\
  \hline\hline
  \endfirsthead

  \multicolumn{3}{c}{Continuation of Harvard Obs.\ Monographs} \\
  Title & Author(s) & Date \\
  \hline\hline
  \endhead

  \hline
  \endfoot
  
  \hline\hline
  \endlastfoot

  1 \bt{Stellar Atmospheres} & Cecilia Payne & 1925 \\

  2 \bt{Star Clusters} & Harlow Shapley & 1930 \\

  3 \bt{The Stars of High Luminosity} & Cecilia Payne & 1930 \\

  
\end{longtable}

\printbibliography

\end{verbatim}


\printbibliography

\end{document}


