%%
%%
%% style_python.tex
%%
%%   The style that I will use for Pythn code in 
%%   the book project "Some Important Books in Astronomy
%%    and Astrophysics in the 20th Century"
%%
%%   Copyright 2018 James R. Fowler
%%
%%   All rights reserved. No part of this publication may be
%%   reproduced, stored in a retrival system, or transmitted
%%   in any form or by any means, electronic, mechanical,
%%   photocopying, recording, or otherwise, without prior written
%%   permission of the author.
%%
%%
%%
\section{General Comments}

I will be using Python \url{http://www.python.org} for much of the software
that will be needed for the \ProjectTitle\ project.

\section{Required Python Packages}

We currently use Python version 3.6 or greater with PyQt version 5.10.
The standard packages that are used include \texttt{os}, \texttt{sys},
\texttt{argparse}, \texttt{re}, \texttt{fileinput},\texttt{traceback},
and \texttt{configparser}.

\subsection{Additional Packages}

In addtion to the standard packages normally included with the Python
distribution, we are also us \texttt[PyQt}5 and\texttt{lxml}
particularly \texttt{etree}. 

\subsection{Auxillary Packages}
The following third-party packages are required.

\subsubsection{nameparser}

Use to parse human names in the ajbbooks program.  Where did it come
from? Refer to documentation if possible.

\subsubsection{modgrammer}

Used for parsing comment fields in version 1.0 of ajbbooks. This
capability is no longer used however the ability to read/write
text files with comments is still available. Where did it come from?

\subsubsection{PyQt 5.x}

\section{Installation Directory}

The installation directory shall be my home directory on what ever machine
I install on.

\section{Standard Command Line Options and Flags}

All python programs shall use the \texttt{argparse} package and
support and accept the following command line options. Consider
writing a stardard parser package for all of my python programs.

\begin{itemize}

  \item --version, -v -- print the version information for the program .
   This requires a version variable in the creation of the
   \texttt{argparse} parser.  [Add reference for the \texttt{argparse}
   package.]

  \item --help, -h -- display a window with or print some brief
    description about the program. Requires a help string variable in the 
    \texttt{argparse} parser 

  \item --verbose -- print extra information. Programs don't
    need to actually print extra information if this flag is found but
    they must accept the flag on the command line. Note that 
    there should be no short argument version of this flag
\end{itemize}

The following flags shall be used if needed in the program.

\begin{itemize}
  \item --input, -i -- specify the input file if needed or an alternate
   input file if there is a default. Input files may be required on the
   command line, in which case this flag is unnecessary.

 \item --config, -c -- specify an alternate configuration file if one
   is used.
\end{itemize}


\section{Specific Program Documentation}

Documentation detailing the specific of a python program shall be
written in Sphinx.

Use PEP-8  (\url{https://www.python.org/dev/peps/pep-0008/}) as a
style standard for all Python programs. Use \texttt{pylint} to test
code files againts PEP-8. 


