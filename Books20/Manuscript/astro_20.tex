%%
%%
%%  astro_20.tex
%%
%%   The chapter 'Astronomy and Asrophysics in the 20th Century' for
%%     the book "Some Important Books in Astronomy and
%%      Astrophysics in the 20th Century"
%%
%%   copyright 2008 James R. Fowler
%%
%%   All rights reserved. No part of this publication may be
%%   reproduced, stored in a retrival system, or transmitted
%%   in any form or by any means, electronic, mechanical,
%%   photocopying, recording, or otherwise, without prior written
%%   permission of the author.
%% 
%%
%%   The last known changes were checked in by $Author$
%%   as revision $LastChangedRevision$
%%   on $Date$
%%
%%
\chapter{Astronomy and Astrophysics in the \Cen{20}}
\markboth{The \Cen{20}}{The \Cen{20}}

Five Periods:
\begin{itemize}
	\item 1900-1918, pre-WWI
	\item 1919-1945, the best of times, the worst of times
	\item 1945-1958, pre-Sputnik, the author is born
	\item 1959-1974, the era of spaceflight, the author graduates High School 
	\item 1975-1999, solid state detectors and computers, the author gains his Ph.D
\end{itemize}

\section{New techniques of the century}

A list of new technologies and new science.
\begin{itemize}
	\item 1900-1918
		\begin{itemize}
			\item telescopes, Mt.\ Wilson
			\item spectroscopy improvements
		\end{itemize}
	\item 1919-1945
		\begin{itemize}
			\item quantum physics
			\item stellar evolution
			\item improved photographic plates
			\item extra-galactic understanding
			\item distance scales
			\item telescopes, Palomar
		\end{itemize}
	\item 1945-1958
		\begin{itemize}
			\item photo-electrics detectors
			\item radio astronomy
			\item color systems
		\end{itemize}
	\item 1959-1974
		\begin{itemize}
			\item space flight
			\item large computers
			\item first missions to planets
			\item interplanetary space and magnetic interactions
			\item small space telescopes, UV, gamma ray, x-ray, IR
		\end{itemize}
	\item 1975-1999
		\begin{itemize}
			\item space based great observatories, Hubble, Spitzer, Chandra, Fermi
			\item large ground based observatories, Kitt Peak, VLA, Keck, VLT, HET
			\item university telescope building, MMT, NTT, NOT, Apache Point
			\item small, powerful personal computers
			\item extrasolar planets
		\end{itemize}
\end{itemize}

Most of the first period books dealt with the questions at the turn listed in Clerke, 1903,
\cite{clerke:1903}.  Second period dominated by quantun physics and its application to
stellar evolution.
