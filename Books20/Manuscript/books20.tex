%% Based on a TeXnicCenter-Template by Gyorgy SZEIDL.
%%%%%%%%%%%%%%%%%%%%%%%%%%%%%%%%%%%%%%%%%%%%%%%%%%%%%%%%%%%%%
%%
%%  Some Important Books in Astronomy and Astrophysics 
%%   from the 20th Century
%%
%%   Copyright 2008 James R. Fowler
%%
%%   All rights reserved. No part of this publication may be
%%   reproduced, stored in a retrival system, or transmitted
%%   in any form or by any means, electronic, mechanical,
%%   photocopying, recording, or otherwise, without prior written
%%   permission of the author.
%% 
%%   The last known changes were checked in by $Author$
%%   as revision $LastChangedRevision$
%%   on $Date$
%%
%----------------------------------------------------------
%
\documentclass{book}%
%
%----------------------------------------------------------
% This is a sample document for the standard LaTeX Book Class
% Class options
%       --  Body text point size:
%                        10pt (default), 11pt, 12pt
%       --  Paper size:  letterpaper (8.5x11 inch, default)
%                        a4paper, a5paper, b5paper,
%                        legalpaper, executivepaper
%       --  Orientation (portrait is the default):
%                        landscape
%       --  Printside:   oneside, twoside (default)
%       --  Quality:     final(default), draft
%       --  Title page:  titlepage, notitlepage
%       --  Columns:     onecolumn (default), twocolumn
%       --  Start chapter on left:
%                        openright(no, default), openany
%       --  Equation numbering (equation numbers on right is the default):
%                        leqno
%       --  Displayed equations (centered is the default):
%                        fleqn (flush left)
%       --  Open bibliography style (closed bibliography is the default):
%                        openbib
% For instance the command
%          \documentclass[a4paper,12pt,reqno]{book}
% ensures that the paper size is a4, fonts are typeset at the size 12p
% and the equation numbers are on the right side.
%
\usepackage{amsmath}%
\usepackage{amsfonts}%
\usepackage{amssymb}%
\usepackage{graphicx}

%----------------------------------------------------------
\newcommand{\Cen}[1]{\ensuremath{#1^{th}} Century}
%----------------------------------------------------------
%
% History
% created 30 Nov 2008
%
%----------------------------------------------------------
\begin{document}

\frontmatter
\title{Some Important Books in Astronomy and Astrophysics from the \Cen{20}}
\author{James R. Fowler}
\date{copy of \today}
\maketitle

\vspace*{5 in}
Copyright \copyright 2008 by James R. Fowler

All rights reserved. No part of this publication may be
reproduced, stored in a retrival system, or transmitted
in any form or by any means, electronic, mechanical,
photocopying, recording, or otherwise, without prior written
permission of the author.

\tableofcontents

%%
%%
%%  preface.tex
%%
%%   The preface for
%%     the book "Some Important Books in Astronomy and
%%      Astrophysics in the 20th Century"
%%
%%   Copyright 2008 James R. Fowler
%%
%%   All rights reserved. No part of this publication may be
%%   reproduced, stored in a retrival system, or transmitted
%%   in any form or by any means, electronic, mechanical,
%%   photocopying, recording, or otherwise, without prior written
%%   permission of the author.
%% 
%%
%%   The last known changes were checked in by $Author$
%%   as revision $LastChangedRevision$
%%   on $Date$
%%
%%
\chapter{Preface}

Thanks to all the little people who made this possible.

\mainmatter

%% Begin copyright
%%
%%  /home/jrf/Documents/books/Books20/Manuscript/intro.tex
%%  
%%   Part of the Books20 Project
%%
%%   Copyright 2008 James R. Fowler
%%
%%   All rights reserved. No part of this publication may be
%%   reproduced, stored in a retrival system, or transmitted
%%   in any form or by any means, electronic, mechanical,
%%   photocopying, recording, or otherwise, without prior written
%%   permission of the author.
%%
%%
%% End copyright

%%   The introduction to "Some Important Books in Astronomy and
%%      Astrophysics in the 20th Century"


``For of the making of lists there is no end...'' The tradition of
list making in the bibliographic sciences has a long history.  Alberto
Manguel, in \emph{The Library at Night}, p 110-112, (The Library as
Shadow), states that
\begin{quote}
...in the first half of the second century B.C., a couple of the
  principal librarians of Alexandria, Aristophanes of Byzantium and
  his disciple Aristarchus of Samothrace, decided to assist their
  readers in a similar fashion.  Not only did they select and gloss
  all manner of important works, but they also set out to compile a
  catalogue of authors who, in their opinion, surpassed all others in
  literary excellence. \cite{manguel:2008}
\end{quote}
and he references Casson, \emph{Libraries in the Ancient World},
\cite{casson:2001}.  We also have Dibner, \emph{Heralds of Science},
\cite{dibner:1955}, Devorkin's, \emph{The History of Modern Astronomy
  and Astrophysics: A Selected, Annotated Bibliography},
\cite{devorkin:1982}, and the standard but incomplete list of
astronomy books through 1880, Houzeau and Lancaster,
\emph{Bibliographe g\'{e}n\'{r}ale de l'astronomie},
\cite{houzeau:1882}.

List of bibliographies that include the \Cen{20}...

Manguel goes on to say that the creation of lists is dangerous for, by
definition, the books that are included on the list become the
important ones and the books that are left off are relegated to the
dust bins of history.

There is a certain hubris in selecting the ``most interesting'' books
of the \Cen{20}. I am not worthy of this honour.

%%
%% How books are used in Science
%%
\section{How books are used in Science}
During the \Cen{18} books were primarily used to introduce topics to
other scientists.  Journals were just starting.

The \Cen{19} was a period of transition for books. Some introduced a
new topic, some books summarized a period of work.

During the \Cen{20} books were primarily used to summarize a field.
After many journal papers the scientist would gather together and
organize the arguments in book form.  The book would then be
referenced in later journal articles rather than the previous articles
on the topic. Publication by S.\ Chandreshkar are a good example. In
the later half of the \Cen{20} book were published more as graduate
text books rather than as fundamental contributions to the field.

In 19xx the Astrophysical Journal began publishing upcoming articles
in journals so as the give a heads up to scientists who were not on
the pre-print distibution list.  This later became ArXiv PrePrint
server on the world wide web first at Los Alomos then a
http://lanl.arXiv.org/


%%
%% What Consitutes an important book
%%
\section{What constitutes an important book}

Mostly American and British books. I don't read German, French,
Italian, or Russian, much to my regret.  However, the set back to
science from the two world wars means that the dominant texts came
from America and Britain. A truly influential book would most likely
have been translated into other languages

What is the definition of an important book? I would argue that it is
one that influenced the opinions of other astronomers either by
introducing a subject in a new way or was used as a dominant text
books at the graduate level.  One needs to be aware of picking only
the popular books or that ones that get mentioned in the popular
histories.  An example of such an item would be the Shapely/Curtis
debate which had little influence on professional astronomers at the
time but became large in the popular press 50 years later.

This compilation will almost certainly be incomplete.  I am unfamiliar
with many areas of astronomy and astrophysics in the late \Cen{20}.

Plot of the number of books cited as a function of year.  More during
the later periods.



%%
%%
%%  astro_pre.tex
%%
%%   The chapter 'Astronomy before the 20th Century' for
%%     the book "Some Important Books in Astronomy and
%%      Astrophysics in the 20th Century"
%%
%%   Copyright 2008 James R. Fowler
%%
%%   All rights reserved. No part of this publication may be
%%   reproduced, stored in a retrival system, or transmitted
%%   in any form or by any means, electronic, mechanical,
%%   photocopying, recording, or otherwise, without prior written
%%   permission of the author.
%% 
%%
%% The last known changes were checked in by $Author$
%% as revision $LastChangedRevision$
%% on $Date$
%%
%%
\chapter{Astronomy before the \Cen{20}}

Photography in the blue well advanced. Color systems?

Spectrograph invented ???? Kirchoff explains absorption and emission in 
1862?  Secchi classification of stars 1877.  When was the Harvard classification
made? Major question at end of century about how classification and evolution
were related.

Important questions in solar physics.

Speculations about the nature of nebulae?  (Solved by Hubble.)

Planetary studies.

Mathematical astronomy and orbit determination well advanced.

Double star work.  Plotting orbit well advanced.


%% Begin copyright
%%
%%  /home/jrf/Documents/books/Books20/Manuscript/astro_20.tex
%%  
%%   Part of the Books20 Project
%%
%%   Copyright 2008 James R. Fowler
%%
%%   All rights reserved. No part of this publication may be
%%   reproduced, stored in a retrival system, or transmitted
%%   in any form or by any means, electronic, mechanical,
%%   photocopying, recording, or otherwise, without prior written
%%   permission of the author.
%%
%%
%% End copyright

%%
%%   The chapter 'Astronomy and Asrophysics in the 20th Century' for
%%     the book "Some Important Books in Astronomy and
%%      Astrophysics in the 20th Century"


Five Periods:
\begin{itemize}
\item 1900-1918, pre-WWI
\item 1919-1945, the best of times, the worst of times
\item 1945-1958, pre-Sputnik, the author is born
\item 1959-1974, the era of spaceflight, the author graduates High School 
\item 1975-1999, solid state detectors and computers, the author gains his Ph.D
\end{itemize}

\section{New techniques of the century}

A list of new technologies and new science.
\begin{itemize}
\item 1900-1918
  \begin{itemize}
  \item telescopes, Mt.\ Wilson
  \item spectroscopy improvements
  \end{itemize}
\item 1919-1945
  \begin{itemize}
  \item quantum physics
  \item stellar evolution
  \item improved photographic plates
  \item extra-galactic understanding
  \item distance scales
  \item telescopes, Palomar
  \end{itemize}
\item 1945-1958
  \begin{itemize}
  \item photo-electrics detectors
  \item radio astronomy
  \item color systems
  \end{itemize}
\item 1959-1974
  \begin{itemize}
  \item space flight
  \item large computers
  \item first missions to planets
  \item interplanetary space and magnetic interactions
  \item small space telescopes, UV, gamma ray, x-ray, IR
  \end{itemize}
\item 1975-1999
  \begin{itemize}
  \item space based great observatories, Hubble, Spitzer, Chandra, Fermi
  \item large ground based observatories, Kitt Peak, VLA, Keck, VLT, HET
  \item university telescope building, MMT, NTT, NOT, Apache Point
  \item small, powerful personal computers
  \item extrasolar planets
  \end{itemize}
\end{itemize}

Most of the first period books deal with the questions at the turn
listed in Clerke, 1903, \cite{clerke:1903}.  Second period dominated
by quantum physics and its application to stellar evolution.


%% Begin copyright
%%
%%  /home/jrf/Documents/books/Books20/Manuscript/the_process.tex
%%  
%%   Part of the Books20 Project
%%
%%   Copyright 2008 James R. Fowler
%%
%%   All rights reserved. No part of this publication may be
%%   reproduced, stored in a retrival system, or transmitted
%%   in any form or by any means, electronic, mechanical,
%%   photocopying, recording, or otherwise, without prior written
%%   permission of the author.
%%
%%
%% End copyright

%%
%%   The chapter 'The Process' for
%%     the book "Some Important Books in Astronomy and
%%      Astrophysics in the 20th Century"
%%

\begin{itemize}
\item Chapter One: Introduction
  \begin{itemize}
  \item How were books used in Astronomy and Astrophysics when journal article
    were the main communications media?
  \item How does the use of books differ from that of the \Cen{19}?
  \item How does the use of books differ from other fields of art or science?
  \item What is an 'important' book? influence, topic matter, citations?
  \end{itemize}
  
\item Chapter Two: Astronomy before the \Cen{20}
  \begin{itemize}
  \item What was the state of Astronomy at the turn of the century?
  \item What were the important questions?
    \begin{itemize}
    \item Problems in Astronomy, Agnus Clerke, 1903, \cite{clerke:1903}
    \item Recent Advances in Astronomy, Alfred Frison, 1899 \cite{frison:1899}
    \item Norton History of Astronomy and Cosmology, 1995
    \item The General History of Astronomy, Cambridge Univ.\ Press
    \end{itemize}
  \end{itemize}
  
\item Chapter Three: Astronomy in the \Cen{20}
  \begin{itemize}
  \item What astronomical topics were important in the \Cen{20}?
    \begin{itemize}
    \item Source Book in Astronomy, 1900-1950, Harlow Shapely \cite{shapley:1960}
    \item A Source Book in Astronomy and Astrophysics, 1900-1975, Lang and Gingerich,
      \cite{lang:1978}.
    \item Norton History of Astronomy and Cosmology, 1995
    \item The General History of Astronomy, Cambridge Univ.\ Press
    \item IAU Commission for Documentation and Bibliography
    \end{itemize}	
  \end{itemize}
  
\item Chapter Four: The Process
  \begin{itemize}
  \item How do I locate books published in the century?
    \begin{itemize}
    \item Kemp 1970
    \item DeVorkin 1982
    \end{itemize}
  \item How do I identify the important ones? Ask astronomers; read biographies; check
    correspondence.
  \end{itemize}
  
\item Chapter Five: Speculations on the Future
  \begin{itemize}
  \item What are the important questions at the end of the \Cen{20}?
  \item Where will books go from here with the advent of the web?
  \end{itemize}
  
\end{itemize}


\chapter{Speculations on the Future}

It is always perilous to speculate about the future.
What will be the exciting topics in astronomy and astrophysics?
Will we see images of exo-planets?  Will larger space telescopes
give us views we can't imagine today?

what of books in the age of the internet?
It is going to be interesting to see how publishing
changes in this age of information.  Will
summary works still be printed? What of the permanence
of documents published on the web only? 

\appendix

%%
%%
%%  the_books.tex
%%
%%   The appendix 'The Books' for
%%     the book "Some Important Books in Astronomy and
%%      Astrophysics in the 20th Century"
%%
%%   Copyright 2008 James R. Fowler
%%
%%   All rights reserved. No part of this publication may be
%%   reproduced, stored in a retrival system, or transmitted
%%   in any form or by any means, electronic, mechanical,
%%   photocopying, recording, or otherwise, without prior written
%%   permission of the author.
%% 
%%
%%   The last known changes were checked in by $Author$
%%   as revision $LastChangedRevision$
%%   on $Date$
%%
%%

%%
%% \bkentry{author}{year}{title}{publishing}{description}{references}
%%
\newcounter{bksctr}

\newcommand{\bkentry}[6]{
\stepcounter{bksctr}
\vspace*{1 cm}
\noindent
{\bf\arabic{bksctr}  #2}\newline
{\it\large #3\hfil}\newline
{\bf #1}, #4\newline
#5\newline
#6\newline
Comments:\newline
}

%
% Example book entry, no index number
%
\newcommand{\exbkentry}[6]{
\vspace*{1 cm}
\noindent
{\bf Index  #2}\newline
{\it\large #3}\newline
{\bf #1}, #4
#5\newline
#6\newline
Comments:\newline
}


\chapter{The Books}

\textbf{What order should the books be listed in? Alphabetic, by author or title? By subject,
stellar, galactic, radio, visual? By date? I'm inclined to list them by date at this time.
What about multiple editions or first editions published in Europe and the USA?  How do
we list them under one book entry?}

The books are listed by date of publication in order to emphasize the historical
development of astronomy.  Where there are multiple books
published in a single year, they are listed alphabeticly by the author's last
name.

The entries contain an index number, the author(s), title, date, place
of publication, and publisher.  Also included is a bibliographic description of the 
book.  This is followed by any references to other lists, if such exist. Finally
a comment as to why the book is considered important as well as any publishing
history known, e.g.\ number of edition, translation, etc.

A book entry is formated as follows:

\exbkentry{Year}{Author}{Title}{place of publication, Publisher}
{Bibliographic description}{References to other lists}
Reasons why the book is important and any publishing history.

\section*{References}
References to other lists are as follows:
\begin{itemize}
\item AAA --- \emph{Astronomy and Astrophysics Abstracts}, \cite{aaa:1969},
References are given as AAAvvv.sss.nnn, where vvv is the volume number, sss is the section number (usually 003, books), and nnn is the index number in the section.
\item AJB --- \emph{Astronomischer Jahresbericht}, \cite{ajr:1900}
\item BEA --- \emph{Biographical Encyclopedia of Astronomy} [get citation]
\item DeV --- DeVorkin, D., \emph{The History of Modern Astronomy and Astrophysics},
\cite{devorkin:1982}
\item DSB --- Gillispie, C.\ C., \emph{Dictionary of Scientific Biographies}, \cite{gillispie:1970}
\item Kp --- Kemp, D.A., \emph{Astronomy and Astrophysics, a bibliographical guide}, \cite{kemp:1970}
\end{itemize}

\textbf{Need BEA and DSB entries for all authors. Do we want to give the doi: for any entries?
What is a doi:?}
\newpage

\section*{Books}
\setcounter{bksctr}{0}
%%\indent

\bkentry{1914}{Eddington, Arthur}
{Stellar Movement and the Structure of the Universe}
{London, Macmillian \& Co}
{15 x 22.5 cm, xii, 266, (2) pp, 4 plates, 22 figures, index, 2 pages of advertisments
in the back, blue cloth}
{}

\bkentry{1929}{Jeans, James}
{Astronomy and Cosmology}
{Cambridge, Cambridge University Press}
{18 x 26.25 cm, x, 428 pp, 16 plates, 63 figures, 32 tables, dark blue cloth}
{}

\bkentry{1930}{Eddington, Arthur}
{The Internal Constitution of Stars}
{Cambridge, Cambridge University Press}
{18 x 26.25 cm, x, 407 pp, 5 figs, 48 tables}
{BEA 134; DSB 2:337; DeV 1140}

\bkentry{1930}{Payne, Ceclia}
{The Stars of High Luminosity}
{New York, McGraw-Hill}
{15 x 22.5 cm, xiv, 320 pp, red cloth}
{}

\bkentry{1930}{Shapley, Harlow}
{Star Clusters}
{New York, McGraw-Hill}
{15 x 22 cm, xi, 276 pp, illus, bibliography, red cloth}
{}
Harvard Observatory Monographs number 1.

\bkentry{1931}{Smarrt, William}
{Textbook on Spherical Astronomy}
{Cambridge, Cambridge University Press}
{13.75 x 21.25 cm, xii, 414 pp, 146 figures, blue cloth}
{}

\bkentry{1936}{Hubble, Edwin}
{The Realm of the Nebulae}
{New Haven, Yale University Press}
{15 x 22.5 cm, xii, (2), 210 pp, blue cloth}
{}

\bkentry{1936}{Rossland, Sven}
{Theoretical Astrophysics}
{Oxford, Claredon Press}
{23 x 16.25 cm, xx, 355 pp, brown cloth}
{}

\bkentry{1938}{Payne-Gaposhkin, Ceclia and Sergie Gaposhkin}
{Variable Stars}
{Cambridge, Harvard Observatory}
{15 x 22 cm, xiv, (1),386 pp, red cloth, gilt spine}
{}
Harvard Observatory Monographs number 5.

\bkentry{1938}{Smartt, W.~M.}
{Stellar Dynamics}
{Cambridge, Cambridge University Press}
{20 x 25 cm, viii, 434 pp, 52 tables, 75 figures, blue cloth}
{}

\bkentry{1939}{Chandrasekhar, Subramanian}
{An Introduction to the Study of Stellar Structure}
{Cambridge, Cambridge University Press}
{XX x XX cm, ix, (1), 509 pp, 44 tables, 38 figures, cloth}
{}

\bkentry{1943}{Goldberg, Leo and Lawrence Aller}
{Atoms, Stars and Nebulae}
{Philadelphia, Blakiston Company}
{21.25 x 19 cm, vi, 323, (3) pp, 16 tables, 150 figures, cloth with dust jacket}
{}

\bkentry{1950}{Chandrasekhar, Subramanian}
{Radiative Transfer}
{Oxford, Oxford University Press}
{241 x 165 mm, xiv, 393 pp, 35 tables, 35 figures, dark blue cloth with brown dust jacket}
{}

\bkentry{1953--1963}{Kuiper, Gerald P.\ and Barbara M.\ MiddleHurst}
{The Solar System}
{Chicago, The University of Chicago Press}
{4 vols.}
{}

\bkentry{1954}{Dufay, Jean}
{N\'{e}buleuses Galactiques et Mati\`{e}re Interstellaire}
{Paris, The French Publishing Company}
{229 x 260 mm, 352 pp, cloth}
{}
First published in France in 1954, it was translated by A.\ J.\ Pomerans
and published in the UK in 1957 by Hutchinson \& Co., London, as part of their Scientific and 
Technical Publications.  The first American publication also occurred in 1957 and was 
handled by the Philosophical Library, New York. It was later republished in 1968 by Dover
Publications from the Hutchinson edition.

\bkentry{1958}{Schwarzschild, Martin}
{Structure and Evolution of Stars}
{Princeton, Princeton University Press}
{235 x 260 mm, xvii, (1), 296 pp, cloth with dust jacket}
{}
The second major work on stellar evolution after Chandrasekhar's 1939 work {\bf [c.f\ 11]}.

\bkentry{1959}{Kopal, Zden\u{e}k}
{Close Binary Systems}
{New York, John Wiley \& Sons}
{241 x 159 mm, xiv, (1), 558 pp blue/green cloth}
{}
Volume five of The International Astrophysics Series.

\bkentry{1960}{Chandrasekhar, Subrahmanyan}
{Hydrodynamic and Hydromagnetic Stability}
{Cambridge, Claredon Press}
{}
{}
Goodman and Ji, J. Fluid Mech. (2002) 462, 365, point out that Chandra
dropped an important term in his discussion of magnetorotational
instabilty (MRI). This term turned out to be important in the
discussion of accretion disks and angular momentum transfer.  I found
this report in Ji and Balbus, Physics Today (Aug 2013) 66 Nr. 8 27

\bkentry{1960-1975}{Kuiper, Gerald P.\ and Barbara M.\ Middlehurst}
{Stars and Stellar Systems}
{Chicago, The University of Chicago Press}
{8 vols.}
{}

\bkentry{1962}{Podobed, V.\ V.}
{Fundamental Astrometry (in Russian, of course)}
{Moscow, Fizmatgiz}
{XX x XX mm, XX pp, cheap cloth}
{}
First published in Moscow, it was translated into English by Scripta
Technica, Inc., with editorial work by A.\ N.\ Vyssotsky, and published
by the University of Chicago Press, Chicago, in 1965.  Although astrometry
was no longer the leading subject in astronomy as it was in the early \Cen{19},
it was becoming more important with the rise of the space program and the requirement
of hitting what you aim for.

\bkentry{1963--}{}
{Annual Reviews of Astronomy and Astrophysics}
{Palo Alto, Annual Reviews}
{}
{DeV 122}
An annual series of review articles in topics of current interest.  This has become
the standard to which astronomers and students turn when they wish to
begin research on a subject.  From 1977 on the volumes also included a brief biographical
article by a senior astronomer. The series was edited by Leo Goldberg, A.J.\ Deutsch,
D.\ Layzer, J.\ Phillips and G.\ Burbidge.

\textbf{Need to note when the spine includes AR logo. Find out if dust jackets ever issued.}

\begin{itemize}
	\item (1963), vol.\ 1, eds.\ Goldberg, Leo, Armin J.\ Deutsch and David Layzer, 228x157 mm,
	viii, 418, (1) pp, subject index,
	blue cloth with gold lettering, gold title on cover enclosed in	four gold box outlines,
	spine quarter black surrounding gold title, year and volume number with blue cloth,
	
	\item (1964), vol.\ 2, eds.\ Goldberg, Leo, Armin J.\ Deutsch and David Layzer, 228x157 mm,
	viii, 462, (2) pp, subject index, cumulative title/author index vols.\ 1--2,
	blue cloth with gold lettering, gold title on cover enclosed in four gold box outlines
  spine quarter black surrounding gold title, year and volume number with blue cloth.
	
	\item (1965), vol.\ 3, eds.\ Goldberg, Leo, Armin J.\ Deutsch and David Layzer, 228x157 mm,
	viii, 438, (1) pp, subject index, cumulative title/author index vols.\ 1--3,
	blue cloth with gold lettering, gold title on cover enclosed in four gold box outlines
  spine quarter black surrounding gold title, year and volume number with blue cloth.
	
	\item (1966), vol.\ 4, eds.\ Goldberg, Leo, David Layzer and John G.\ Phillips, 228x157 mm,
	viii, 513 pp, subject index, cumulative title/author index vols.\ 1--4,
	blue cloth with gold lettering, gold title on cover enclosed in four gold box outlines
  spine quarter black surrounding gold title, year and volume number with blue cloth.
	
	\item (1967), vol.\ 5, eds.\ Goldberg, Leo, David Layzer and John G.\ Phillips, 228x157 mm,
	viii, 694, pp, subject index, cumulative title/author index vols.\ 1--5,
	blue cloth with gold lettering, gold title on cover enclosed in four gold box outlines
  spine quarter black surrounding gold title, year and volume number with blue cloth.
	
	\item (1968), vol.\ 6, eds.\ Goldberg, Leo, David Layzer and John G.\ Phillips, 228x157 mm,
	viii, 525 pp, subject index, cumulative title/author index vols.\ 2--6,
	blue cloth with gold lettering, gold title on cover enclosed in four gold box outlines
  spine quarter black surrounding gold title, year and volume number with blue cloth.
	
	\item (1969), vol.\ 7, eds.\ Goldberg, Leo, David Layzer and John G.\ Phillips, 228x157 mm,
	viii, 717 pp, subject index, cumulative title/author index vols.\ 3--7,
	blue cloth with gold lettering, gold title on cover enclosed in four gold box outlines
  spine quarter black surrounding gold title, year and volume number with blue cloth.
	
	\item (1970), vol.\ 8, eds.\ Goldberg, Leo, David Layzer and John G.\ Phillips, 228x157 mm,
	ix, 495 pp, subject index, cumulative title/author index vols.\ 4--8,
	blue cloth with gold lettering, gold title on cover enclosed in four gold box outlines
  spine quarter black surrounding gold title, year and volume number with blue cloth.
	
	\item (1971), vol.\ 9, eds.\ Goldberg, Leo, David Layzer and John G.\ Phillips, 228x157 mm,
	ix, 394 pp, subject index, cumulative title/author index vols.\ 5--9,
	blue cloth with gold lettering, gold title on cover enclosed in four gold box outlines
  spine quarter black surrounding gold title, year and volume number with blue cloth.
	
	\item (1972), vol.\ 10, eds.\ Goldberg, Leo, David Layzer and John G.\ Phillips, 228x157 mm,
	viii, 435 pp, subject index, cumulative title/author index vols.\ 6--10,
	blue cloth with gold lettering, gold title on cover enclosed in four gold box outlines
  spine quarter black surrounding gold title, year and volume number with blue cloth.
	
	\item (1973), vol.\ 11, eds.\ Goldberg, Leo, David Layzer and John G.\ Phillips, 228x157 mm,
	x, 513, pp, subject index, cumulative title/author index vols.\ 7--11,
	blue cloth with gold lettering, gold title,
  spine quarter black surrounding gold title, year and volume number with blue cloth.
  
  The gold rectangles on the cover disappeared in this and subsequent volumes.
	
	\item (1974), vol.\ 12, eds.\ Burbidge, Geoffrey, David Layzer and John G.\ Phillips, 228x157 mm,
	(4), 495, (1) pp, subject index, cumulative title/author index vols.\ 8--12,
	blue cloth with gold lettering, gold title,
  spine quarter black surrounding gold title, year and volume number with blue cloth.
	
	\item (1975), vol.\ 13, eds.\ Burbidge, Geoffrey, David Layzer and John G.\ Phillips, 228x157 mm,
	(5), 557, (4) pp, subject index, cumulative title/author index vols.\ 9--13,
	blue cloth with gold lettering, gold title
  spine quarter black surrounding gold title, year and volume number with blue cloth.
	
	\item (1976), vol.\ 14, eds.\ Burbidge, Geoffrey, David Layzer and John G.\ Phillips, 228x157 mm,
	(5), 500, (1) pp, subject index, cumulative title/author index vols.\ 10--14,
	blue cloth with gold lettering, gold title
  spine quarter black surrounding gold title, year and volume number with blue cloth.
	
	\item (1977), vol.\ 15, eds.\ Burbidge, Geoffrey, David Layzer and John G.\ Phillips, 228x157 mm,
	(6), 602, (1) pp, subject index, cumulative title/author index vols.\ 11--15,
	blue cloth with gold lettering, gold title with list of authors,
  spine quarter black surrounding gold title, year and volume number with blue cloth and list of
  editors.
  
  For the first time the authors were listed on the front cover and the editors on the spine.
  All subsequent volumes have this style.  The first biographical article appears written by
  E.J.\ Opik.
	
	\item (1978), vol.\ 16, eds.\ Burbidge, Geoffrey, David Layzer and John G.\ Phillips, 228x157 mm,
	(5), 652, (1) pp, subject index, cumulative title/author index vols.\ 12--16,
	blue cloth with gold lettering, gold title with list of authors,
  spine quarter black surrounding gold title, year and volume number with blue cloth and list of
  editors.
  
  Instead of a biographical article, C.\ Payne-Gaposchkin wrote an article about the development
  of our knowledge of variable stars.
   
	\item (1979), vol.\ 17, eds.\ Burbidge, Geoffrey, David Layzer and John G.\ Phillips, 228x157 mm,
	(5), 585, (2) pp, subject index, cumulative title/author index vols.\ 13--17,
	blue cloth with gold lettering, gold title with list of authors,
  spine quarter black surrounding gold title, year and volume number with blue cloth and list of
  editors, two page lite-blue order form in rear before free endpaper, no dust jacket issued.
  
  The return of the biographical article, this time by Paul Swings.  All subsequent volumes will
  include such an article. The first order form appears in this volume.
  
	\item (1980), vol.\ 18, eds.\ Burbidge, Geoffrey, David Layzer and John G.\ Phillips, 228x157 mm,
	(5), 591, (3) pp, subject index, cumulative title/author index vols.\ 14--18,
	blue cloth with gold lettering, gold title with list of authors,
  spine quarter black surrounding gold title, year and volume number with blue cloth and list of
  editors, two page lite-blue order form in rear before free endpaper, no dust jacket issued.
  
	\item (1981), vol.\ 19, eds.\ Burbidge, Geoffrey, David Layzer and John G.\ Phillips, 228x157 mm,
	(5), 479, (3) pp, subject index, cumulative title/author index vols.\ 15--19,
	blue cloth with gold lettering, gold title with list of authors,
  spine quarter black surrounding gold title, year and volume number with blue cloth and list of
  editors, two page lite-blue order form in rear before free endpaper, no dust jacket issued.
  
	\item (1982), vol.\ 20, eds.\ Burbidge, Geoffrey, David Layzer and John G.\ Phillips, 228x157 mm,
	xii, 631, (2) pp, subject index, cumulative title/author index vols.\ 16--20,
	blue cloth with gold lettering, gold title with list of authors,
  spine quarter black surrounding gold title, year and volume number with blue cloth and list of
  editors, two page lite-blue order form in rear before free endpaper, no dust jacket issued.
  
  \textbf{Verify end pages in this volume.}
  
	\item (1983), vol.\ 21, eds.\ Burbidge, Geoffrey, David Layzer and John G.\ Phillips, 228x157 mm,
	(6), 482, (1) pp, subject index, cumulative title/author index vols.\ 11--21,
	blue cloth with gold lettering, gold title with list of authors,
  spine quarter black surrounding gold title, year and volume number with blue cloth and list of
  editors, two page lite-blue order form in rear before free endpaper, no dust jacket issued.
  
  The last volume to not use any page numbering for the front matter; the first volume in which
  the cumulative index covers ten years instead of five.
  
	\item (1984), vol.\ 22, eds.\ Burbidge, Geoffrey, David Layzer and John G.\ Phillips, 228x157 mm,
	x, 635, (1) pp, subject index, cumulative title/author index vols.\ 12--22,
	blue cloth with gold lettering, gold title with list of authors,
  spine quarter black surrounding gold title, year and volume number with blue cloth and list of
  editors, two page lite-blue order form in rear before free endpaper, no dust jacket issued.
  
	\item (1985), vol.\ 23, eds.\ Burbidge, Geoffrey, David Layzer and John G.\ Phillips, 228x157 mm,
	x, 466, (2) pp, subject index, cumulative title/author index vols.\ 13--23,
	blue cloth with gold lettering, gold title with list of authors,
  spine quarter black surrounding gold title, year and volume number with blue cloth and list of
  editors, two page lite-blue order form in rear before free endpaper, no dust jacket issued.

	\item (1986), vol.\ 24, eds.\ Burbidge, Geoffrey, David Layzer and John G.\ Phillips, 228x157 mm,
	x, 627, (1) pp, subject index, cumulative title/author index vols 14--24,
	blue cloth with gold and black lettering, spine quarter black with silver lettering,
	two page dark blue order form in rear before free endpaper, no dust jacket issued.

	\item (1987), vol.\ 25, eds.\ Burbidge, Geoffrey, David Layzer and John G.\ Phillips, 228x157 mm,
	x, 684, (5) pp, subject index, cumulative title/author index vols 15--25,
	blue cloth with gold and black lettering, spine quarter black with gold lettering,
	two page dark blue order form in rear before free endpaper, no dust jacket issued.
	
	The appearance of the spine changed with this volume when the lettering was switched
	from silver to gold.  The front cover retained the gold lettering. 

	\item (1988), vol.\ 26, eds.\ Burbidge, Geoffrey, David Layzer and John G.\ Phillips, 228x157 mm,
	x, 703, (3) pp, subject index, cumulative title/author index vols 16--26,
	blue cloth with gold and black lettering, spine quarter black with gold lettering,
	two page dark blue order form in rear before free endpaper, no dust jacket issued.

	\item (1989), vol.\ 27, eds.\ Burbidge, Geoffrey, David Layzer and John G.\ Phillips, 228x157 mm,
	xii, 773, (3) pp, subject index, cumulative title/author index vols 17--27,
	blue cloth with gold and black lettering, spine quarter black with gold lettering,
	two page dark blue order form in rear before free endpaper, no dust jacket issued.

	\item (1990), vol.\ 28, eds.\ Burbidge, Geoffrey, David Layzer and Allen Sandage, 228x157 mm,
	xii, 767, (2) pp, subject index, cumulative title/author index vols 18--28,
	blue cloth with gold and black lettering, spine quarter black with gold lettering,
	two page green order form in rear before free endpaper, no dust jacket issued.

	\item (1991), vol.\ 29, eds.\ Burbidge, Geoffrey, David Layzer and Allen Sandage, 228x157 mm,
	x, 777, (1) pp, subject index, cumulative title/author index vols 19--29,
	blue cloth with gold and black lettering, spine quarter black with gold lettering,
	two page purple order form in rear before free endpaper, no dust jacket issued.
	
	\item (1992), vol.\ 30, eds.\ Burbidge, Geoffrey, David Layzer and Allen Sandage, 228x157 mm,
	xii, 767, (2) pp, subject index, cumulative title/author index vols 20--30,
	blue cloth with gold and black lettering, spine quarter black with gold lettering,
	two page blue order form in rear before free endpaper, no dust jacket issued.
	
	\item (1993), vol.\ 31, eds.\ Burbidge, Geoffrey, David Layzer and Allen Sandage, 228x157 mm,
	xi, 787, (2) pp, subject index, cumulative title/author index vols 21--31,
	blue cloth with gold and black lettering,
	two page green order form in back before free endpaper, no dust jacket issued.
	
	\item (1994), vol.\ 32, eds.\ Burbidge, Geoffrey, David Layzer and Allen Sandage, 228x157 mm,
	x, 662, (2) pp, subject index, cumulative title/author index vols 22--32,
	blue cloth with gold and black lettering, spine quarter black with gold lettering,
	two page blue order form in back before free endpaper, no dust jacket	issued.
	
	\item (1995), vol.\ 33, eds.\ Burbidge, Geoffrey, David Layzer and Allen Sandage, 228x157 mm,
	x, 656, (2) pp, subject index, cumulative title/author index vols 23--33,
	blue cloth with gold and black lettering, spine quarter black with gold lettering,
	two page blue order form in back before free endpaper, no dust jacket issued.
	
	\item (1996), vol.\ 34 revised, eds.\ Burbidge, Geoffrey, and Allen Sandage, 228x157 mm,
	x, 825, (2) pp, subject index, cumulative title/author index vols 24--34,
	blue cloth with gold and black lettering, spine quarter black with gold lettering,
	two page blue order form in back before free endpaper, no dust jacket issued.
	
	This is a revised edition.  A original edition was printed and distributed in September 1996
	but there were problems in the final stages of production and the volume was reprinted and
	reissued in December of 1996.  The revised edition states ``(revised edition)'' under the
	volume number on the title page and contains the editors preface after the copyright page.
	
	\item (1997), vol.\ 35, eds.\ Burbidge, Geoffrey, Allen Sandage and Frank H.\ Shu, 228x157 mm,
	xii, 699, pp, subject index, cumulative title/author index vols 25--35,
	blue cloth with gold and black lettering, spine quarter black with gold lettering,
	two page blue order form in back before free endpaper, no dust jacket	issued.
	
	\item (1998), vol.\ 36, eds.\ Burbidge, Geoffrey, Allen Sandage and Frank H.\ Shu, 228x157 mm,
	x, 692, (1) pp, subject index, cumulative title/author index vols 26--36,
	blue cloth with gold and black lettering, spine quarter black with gold lettering,
	two page blue order form in back before free endpaper, no dust jacket	issued.
	
	\item (1999), vol.\ 37, eds.\ Burbidge, Geoffrey, Allen Sandage and Frank H.\ Shu, 236x157 mm,
	viii, (1), 689, (2) pp, subject index, cumulative title/author index vols 27--37,
	grey-blue cloth with gold and black lettering, spine quarter black with gold lettering,
	two page blue order form in back before free endpaper, no dust jacket	issued.
	
	The appearance of the volumes changes with this issue; they are now a little taller
	and the color of the binding has gotten more grey.
	
\end{itemize}


\bkentry{1966}{Shlovskii, I.\ S.\ and Carl Sagan}
{Intellegent Life in the Universe}
{San Francisco, Holden-Day, Inc.}
{172 x 232 mm, xiv, (2), 509 pp, illus, red cloth with dust jacket}
{}
This work is a translation, extension and revision of I.\ S.\ Schlovskii's
1963 semipopuler book {\it Vselennaia, Zhizn, Razum (Universe, Life, Mind)}.
The english translation was done by Paula Fern and Sagan indicates that it
was also translated in numerous other languages.  The first serious scientific study 
of the possibilities of intelegent life elsewhere in the universe.

\bkentry{1968}{Spitzer, Lyman}
{Diffuse Matter in Interstellar Space}
{New York, Interscience Publishers}
{216 x 140 mm, xiii, 262 pp, illus, yellow cloth}
{}

\bkentry{1968}{Glasstone, Samuel}
{The Book of Mars}
{Washington, D.C., NASA}
{248 x 178 mm, vii, frontis, 315 pp, red cloth}
{}
NASA SP-179

\bkentry{1969}{Chandrasekhar, Subramanian}
{Ellipsoidal Figures of Equilibrium}
{New Haven, Yale University Press}
{235 x 165 mm, xi, 252 pp, cloth}
{}

\bkentry{1971}{Peebles, P.\ James E.}
{Physical Cosmology}
{Princeton, Princeton University Press}
{235 x 152 mm, xvi, 282 pp, cloth}
{}

\bkentry{1973}{Harwitt, Martin}
{Astrophysical Concepts}
{New York, John Wiley \& Sons}
{235 x 165 mm, Xiv, 561 pp, blue paper with a silver and red stamps}
{}
Second edition, 1988, Springer-Verlag, (bib data),
third edition, 19xx, Springer-Verlag, (bib data),
fourth edition, 2006, Springer-Verlag, (bib data).


\bkentry{1974}{Osterbrock, Donald}
{Astrophysics of Gaseous Nebulae}
{San Francisco, W.\ H.\ Freeman and Company}
{235 x 152 mm, xiv, 251 pp, cloth}
{}

\bkentry{1976}{Mutch, Thomas et.\ al.}
{The Geology of Mars}
{Princeton, Princeton University Press}
{286 x 222 mm, ix, 400 oom cloth with dust jacket}
{}
Other authors are Raymond Arvidson, James Head, Kenneth Jones,
and R.\ Saunders.

\bkentry{1978}{Tassoul, Jean-Louis}
{Theory of Rotating Stars}
{Princeton, Princeton University Press}
{235 x 159 mm, xvi, 506 p, cloth with dust jacket}
{}
The Princeton Series in Astrophysics number One

\bkentry{1978}{Spitzer, Lyman}
{Physical Processes in the Interstellar Medium}
{New York, John Wiley and Sons}
{229 x 152 mm, xvii, 318 pp, blue cloth}
{}

\bkentry{1978}{Morgan, William W., Helmut Abt and J.\ Tapscott}
{Revised MK Spectral Atlas For Stars Earlier the the Sun}
{Williams Bay, Yerkes Observatory}
{381 x 311 mm, 14 pp, 32 plates, blue plastic wrappers}
{}

\bkentry{1970}{Mihalis, Dimitri}
{Stellar Atmospheres}
{San Francisco, W.\ H.\ Freeman and Company}
{235 x 152 mm, xx, 623 pp, cloth}
{}
Second edition published in 1978 (San Francisco, W.\ H.\ Freeman and Company,
235 x 152 mm, xx, 623 pp, red stamped cloth.)

\bkentry{1979}{Kopal, Zden\u{e}k}
{Language of the Stars}
{Dordrecht, D.\ Reidel}
{241 x 165 mm, vii, 280 pp, cloth}
{}

\bkentry{1980}{Peebles, P.\ J.\ E.}
{The Large Scale Structure of the Universe}
{Princeton, Princeton University Press}
{235 x 159 mm, xvii, 422 pp, cloth with dust jacket}
{}
Work on the new(?) topic of cosmology.

\bkentry{1980}{Cox, John}
{Theory of Stellar Pulsation}
{Princeton, Princeton University Press}
{235 x 165 mm, xiv, 380 pp, cloth with dust jacket}
{}

\bkentry{1981}{Carr, Michael}
{The Surface of Mars}
{New Haven, Yale University Press}
{279 x 292 mm, xi, 232 pp, cloth with dust jacket}
{}

\bkentry{1981}{Mihalis, Dimitri and James Binney}
{Galactic Astronomy}
{San Francisco, W.\ H.\ Freeman and Compeny}
{235 x 159 mm, xiv, (2), 597 pp, cloth}
{}

\bkentry{1983}{Murray, C.}
{Vectorial Astronomy}
{Bristol, Adam Hilger}
{232 x 156 mm, xiii, (1), 353 pp, cloth with dust jacket}
{}
An important work on telescope pointing during the period
of large telescope construction.

\bkentry{1983}{Shapiro, S.\ and S.\ Teulkosky}
{Black Holes, White Dwarfs and Neutron Stars}
{New York, John Wiley and Sons}
{229 x 165 mm, xvii, 645 pp, cloth with dust jacket}
{}

\bkentry{1984}{Mihalis, Dimtri and Barbara Mihalis}
{Foundations of Radiation Hydrodynamics}
{Oxford, Oxford University Press}
{235 x 165 mm, xv, 718 pp, cloth}
{}
A revision and extention of his previous work on {\it Stellar Atmospheres.}

\bkentry{1987}{Binney, James and Scott Tremaine}
{Galactic Dynamics}
{Princeton, Princeton University Press}
{235 x 165 mm, xv, 733 pp, cloth with dust jacket}
{}
A revised and enlarged version of Binney and Mihalis, {\it Galactic Astronomy.}

\bkentry{1987}{Spitzer, Lyman}
{Dynamical Evolution of Globular Clusters}
{Princeton, Princeton University Press}
{235 x 165 mm, x, 180 pp, cloth with dust jacket}
{}

%%
%% $Header$
%%
%%
%%  $Log$
%%
%%


\backmatter

\bibliographystyle{plain}
\bibliography{references}


\chapter{Afterword}

The back matter often includes one or more of an index, an afterword,
acknowledgements, a bibliography, a colophon, or any other similar item. In
the back matter, chapters do not produce a chapter number, but they are
entered in the table of contents. If you are not using anything in the back
matter, you can delete the back matter \TeX field and everything that follows it.

%
% The End
%
\end{document}
