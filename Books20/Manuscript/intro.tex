%%
%%
%%  intro.tex
%%
%%   The introduction to "Some Important Books in Astronomy and
%%      Astrophysics in the 20th Century"
%%
%%   Copyright 2008 James R. Fowler
%%
%%   All rights reserved. No part of this publication may be
%%   reproduced, stored in a retrival system, or transmitted
%%   in any form or by any means, electronic, mechanical,
%%   photocopying, recording, or otherwise, without prior written
%%   permission of the author.
%% 
%%
%%   The last known changes were checked in by $Author$
%%   as revision $LastChangedRevision$
%%   on $Date$
%%
%%
\chapter{Introduction}

``For of the making of lists there is no end...'' The
tradition of list making in the bibliographic sciences has
a long history.
Alberto Manguel, in \emph{The Library at Night}, p 110-112, (The Library
as Shadow), states that
\begin{quote}
...in the first half of the second century B.C., a couple of the
principal librarians of Alexandria, Aristophanes of Byzantium and his
disciple Aristarchus of Samothrace, decided to assist their readers in a similar fashion.
Not only did they select and gloss all manner of important works, but they also set out
to compile a catalogue of authors who, in their opinion, surpassed all others in 
literary excellence. \cite{manguel:2008}
\end{quote}
and he references Casson, \emph{Libraries in the Ancient World}, \cite{casson:2001}.
We also have Dibner, \emph{Heralds of Science}, \cite{dibner:1955},
Devorkin's, \emph{The History of Modern Astronomy and Astrophysics: A Selected,
Annotated Bibliography}, \cite{devorkin:1982},
and the standard but incomplete list of astronomy books through 1880, Houzeau and Lancaster,
\emph{Bibliographe g\'{e}n\'{r}ale de l'astronomie}, \cite{houzeau:1882}.

List of bibliographies that include the \Cen{20}...

Manguel goes on to say that the creation of lists is dangerous
for, by definition, the books that are included on the list
become the important ones and the books that are left off
are relegated to the dust bins of history.

There is a certain hubris in selecting the ``most important'' books
of the \Cen{20}. I am not worthy of this honour.

%%
%% How books are used in Science
%%
\section{How books are used in Science}
During the \Cen{18} books were primarily used to introduce topics to
other scientists.  Journals were just starting.

The \Cen{19} was a period of transition for books. Some introduced a
new topic, some books summarized a period of work.

During the \Cen{20} books were primarily used to summarize a field.
After many journal papers the scientist would gather together and
organize the arguments in book form.  The book would then be
referenced in later journal articles rather than the previous articles
on the topic. Publication by S.\ Chandreshkar are a good example. In
the later half of the \Cen{20} book were published more as graduate
text books rather than as fundamental contributions to the field.

In 19xx the Astrophysical Journal began publishing upcoming articles
in journals so as the give a heads up to scientists who were not on
the pre-print distibution list.  This later became ArXiv PrePrint
server on the world wide web first at Los Alomos then a
http://lanl.arXiv.org/


%%
%% What Consitutes an important book
%%
\section{What constitutes an important book}

Mostly American and British books. I don't read German, French,
Italian, or Russian, much to my regret.  However, the set back to
science from the two world wars means that the dominant texts came
from America and Britain. A truly influential book would most likely
have been translated into other languages

What is the definition of an important book? I would argue that it is
one that influenced the opinions of other astronomers either by
introducing a subject in a new way or was used as a dominant text
books at the graduate level.  One needs to be aware of picking only
the popular books or that ones that get mentioned in the popular
histories.  An example of such an item would be the Shapely/Curtis
debate which had little influence on professional astronomers at the
time but became large in the popular press 50 years later.

This compilation will almost certainly be incomplete.  I am unfamiliar
with many areas of astronomy and astrophysics in the late \Cen{20}.

Plot of the number of books cited as a function of year.  More during
the later periods.

