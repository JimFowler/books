%%
%%
%% 18Cent.txt
%%
%%   Notes on the books and articles that have been read in conjunction
%%    with 18 Century astronomy. For the book project "Some Important Books
%%     in Astronomy and Astrophysics in the 20th Century"
%%
%%   Copyright 2009 James R. Fowler
%%
%%   All rights reserved. No part of this publication may be
%%   reproduced, stored in a retrival system, or transmitted
%%   in any form or by any means, electronic, mechanical,
%%   photocopying, recording, or otherwise, without prior written
%%   permission of the author.
%%
%%
%% The last known changes were checked in by $Author$
%% as revision $LastChangedRevision$
%% on $Date$
%%
%%

%% 
{\bf 7/11/2009}

\emph{Lacaille: Astronomet=r, Traveler}, Davis S. Evans, \cite{evans:1992}
gives a brief introduction and overview of \Cen{18} astronomy.

Lacaille 1713---1762

Spain was losing its empire, Britain and France were gaining ground.
Many wards between the commercial empires for control of trade.  To
maintain their control of these commercial empires nations need good
protection and good navigation.  The British Board for the Discovery
of Longtitude was created in 1714 but Royal Societies had be sent up
in France and England in the \Cen{17}.  Physics was becoming
mathematical with the publication of Newton's Principia follewed by
additional work in Classical Mechanics. Precession, nutation, and the
abberation of starlight were known effects but there quatitative
contributions were not yet know with enough accuracy.  The lunar
method and the occultations of the satellites of Jupiter were known
to be able to measure longitude but the accuracy of star catalogs and
orbits was not know well enough to provide the accuarcy required.

Evans lists four problems that dominated the \Cen{18}.

\begin{itemize}

\item Law of atmospheric refraction --- necessary for the accurate
  determination of latitude.

\item Figure of the earth --- length of a degree of latitude was necessary
  for accurate maps of the earth.

\item Distance to the planets --- essentially the length of one
  astronomical unit.  Best method would be transit of Venus.

\item Refinement of earth's orbital parameters --- for calculating
ephemerides for naviagation and for accurate star catalogs.

\end{itemize}

{\bf 9/8/2012}

\emph{From Eudoxes to Einstein: A History of
Mathematical Astronomy}, C. M. Linton, \cite{linton:2004}

Chapter 9

page 291, ``The seventeenth century witnessed a complete
transformation in astronomy.  The Ptolemaic universe of uniform
circular motions disappeared and was replaced by a system based on
mechanical principles and Keplerian orbits. At the beginning of the
eighteenth century, the methods being used to analyse the motions of
the heavens were rooted still in the geometry of the ancients Greeks,
but by the end of the century this, too, had changed, with dynamics
reduced to the solution of differential equations: `physical
astronomy' became `celestial mechanics'.''

page 292, The major figures in the first half of the \Cen{18} were
Alexis-Claude Clairaut, Leonhard Euler, and Jean le Rond d'Alembert.

page 317, The second half of the \Cen{18} was dominated by Joseph Lous
Lagrange Pierre-Simon Laplace

page 305, Halley's comet returned in 1759. Belief in Newtonian
dynamics reinforced amoung the general public (cf. Waff 1986, JHA
17,1--37)

page 352, Near the end of the \Cen{17} Newton devised universal
gravitation, by the end of the \Cen{18} Laplace had practically
perfected it. A mechanistic universe, the eternal clockwork, seemed to
be the nature of the world.

Chapter 10

page 389, the prediction of Neptune by Leverrier and Adams built
further confidence in the mind of the general public about the truth
of Newtonian dynamics and the law of Universal Gravitation.

Chapter 11

The \Cen{19} saw an improvement in the techniques of celestial
mechanics with important work done by Hamilton and Jacobi but the
only real theoretical insight was that the Earth (and by implication
the other planets) was not a rigid body

page 405, ``In 1800, celestial mechanics was pursued for one purpose
only, i.e.\ to provide the theory necessary to produce accurate tables
with which to predict future positions of heavenly bodies. By 1900,
another competing goal had emerged as mathematicians began to analyse
the equations of celestial mechanics in their own right to see what
general conclusions could be drawn about the dynamics of the Solar
System.  The mathematical tools needed to do this came to the fore in
attempts to understand the motion of the Moon.''

page 406, Simon Newcomb made significant technical improvements
in celestial mechanics during the \Cen{18}.


