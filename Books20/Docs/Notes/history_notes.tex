%% Begin copyright
%%
%%  /home/jrf/Documents/books/Books20/Docs/Notes/history_notes.tex
%%  
%%   Part of the Books20 Project
%%
%%   Copyright 2014 James R. Fowler
%%
%%   All rights reserved. No part of this publication may be
%%   reproduced, stored in a retrival system, or transmitted
%%   in any form or by any means, electronic, mechanical,
%%   photocopying, recording, or otherwise, without prior written
%%   permission of the author.
%%
%%
%% End copyright

%%
%% Notes and references about history and histriography.
%% This document is included in Notes.tex
%%
{\bf 6/27/2014}

{\it A Few Home Truths}, Jonathan Ree, London Review of Books, 19 June
2014, in a review of {\it R.\ G.\ Collingwood: 'An Autobiography' and
Other Writings, with essays on Collingwood's Life and Work.}, David
Boucher and Teresa Smith, Oxford, 2013

Archaeology taught him that if you wanted to be a good historian you
could not content yourself with collecting established facts and
arranging them in chronological order. You had to put away your
'scissors and paste', as he put it, and start using your imagination
--- 'getting inside other people's heads, looking at their situation
through their eyes, and thinking for yourself whether the way in which
they tackled it was the right way'. And if, as often happens, you
found yourself tempted to dismiss their notions as primitive,
irrational or bizarre, you should reflect that the fault may lie not
in them but in you. The chances are that the problems that bothered
them were nothing like the ones that strike you as obvious or
inevitable, and that they were offering sensible answers to their own
questions rather that foolish answers to yours. That was where history
began to hold up a light to the problems of philosophy: it suggested
that knowledge develops not throught the accumulation of separate
facts, but through the transformation of one set of problems into
another. Problems were constantly being solved, of course, but every
solution opened up a new field of problems, and so on without end. The
only truth we could be sure of was that the growth of knowledge is
unpredictable, and that no truth, however good its credentials, can be
expected to satisfy us for ever.

