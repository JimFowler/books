%% Begin copyright
%%
%%  /home/jrf/Documents/books/Books20/Docs/Notes/book_references.tex
%%
%%   Part of the Books20 Project
%%
%%   Copyright 2019 James R. Fowler
%%
%%   All rights reserved. No part of this publication may be
%%   reproduced, stored in a retrival system, or transmitted
%%   in any form or by any means, electronic, mechanical,
%%   photocopying, recording, or otherwise, without prior written
%%   permission of the author.
%%
%%
%% End copyright

%%
%% Notes about books referenced in other books and articles. Also
%% descriptions about how these references are used by the others.
%%
%% Record which books, particularly\Cen{20} books, and why they are
%% referenced, what the books contained that made their reference important.
%%
{\bf 2019-10-17}

Alexander et.\ al., in their article {\it The Exploration History of
Europa} \cite{Alexander2009} reference a number of books in a
discussion of the history of the observation and study of Jupiter's
moon Europa.  The books range from Galileo Galilei's \bt{Sidereus
Nuncius} \cite{Galilei1989} to Arthur C.\ Clarke's novel \bt{2010:
Odyssey Two} \cite{Clarke1982}.

Laplace 1805, \bt{Mechnique Celeste, vol.\ 4} \cite{Laplace1805} first
demonstrated orbital resonance and was the first work to give a mass
for Europa and the other Galilean satellites. More modern estimates of
the masses are given in Brouwer and Clemence \cite{Brouwer1961} in
their review in \bt{Planets and Satellites} \cite{Kuiper1961} as well
as Kovalevsky \cite{Kovalevsky1970} in \bt{Surfaces and Interiors on
Planets and Satellites} \cite{Dollfus1970}.

Secchi in 1859 \cite{Secchi1859} provided an early result for the
diameter Europa which was 6\% too large.  Popular astronomy books by
Herschel (1861) \cite{Herschel1861} and Newcomb
(1878) \cite{Newcomb1878}.

He also mentions Russell et.\ al.\ (1941) \cite{Russell1941} textbook
\bt{Astronomy} as well as the textbook by Payne-Gaposchkin (1956),
\bt{Introduction to Astronomy} \cite{Payne-Gaposchkin1956} while he mentions
Urey (1952), in \bt{The Planets} \cite{Urey1952} with the idea that
the outer worlds are made of water ice.

Infrared spectra of satellites were extensively reviewed by Johnson
and Pilcher (1977) \cite{Johnson1977} for their article in the
University of Arizona Space Science Series book \bt{Planetary
Satellites} \cite{Burns1977}. A review of the discoveries about
the Jovian magnetosphere during the 1970's calls upon \bt{Physics of the
Jovian Magnetosphere} (1983) \cite{Dressler1983}.

The post-Pioneer summations are given in two volumes from the
University of Arizona Space Science Series \bt{Planetary
Satellites} \cite{Burns1977} and\bt{Jupiter} \cite{Gehrels1976}.


\bt{The Satellites of Jupiter} \cite{Morrison1982}
