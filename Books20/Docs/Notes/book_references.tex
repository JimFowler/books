%% Begin copyright
%%
%%  /home/jrf/Documents/books/Books20/Docs/Notes/book_references.tex
%%
%%   Part of the Books20 Project
%%
%%   Copyright 2019 James R. Fowler
%%
%%   All rights reserved. No part of this publication may be
%%   reproduced, stored in a retrival system, or transmitted
%%   in any form or by any means, electronic, mechanical,
%%   photocopying, recording, or otherwise, without prior written
%%   permission of the author.
%%
%%
%% End copyright

%%
%% Notes about books referenced in other books and articles. Also
%% descriptions about how these references are used by the others.
%%
%% Record which books, particularly\Cen{20} books, and why they are
%% referenced, what the books contained that made their reference important.
%%

{\bf 2021-03-20}

Mitton and Mitton, in their biography of Vera Rubin
give some of the popular astronomy books that might have influenced
Rubin. In particular she names books by Jeans \bt{The Universe Around Us},
\bt{Through Time and Space}, and \bt{The Stars in their Courses} as
ones she might have read as well as Hubble \bt{The Realm of the
Nebulae}. Rubin stated in an interview that she had read
Edditington's \bt{The Expanding Universe} (\cite{Mitton2021} p. 18).
Rubin attend Vassar in 1945, Maud Makemson was there as director of
the Maria Mitchell Observatory and the instructor in astronomy.  The
freshman astronomy textbook was \bt{Astronomy} (1939) by William
T.~Skilling and Robert S.~Richardson. A review of this text when it
first came out (W.F. Meyer, PASP 51 (304) 370) considered it too
simple for astronomy students. Makemson also taught a history of
astronomy class for which she used Shapely and Howard \bt{A Source
Book in Astronomy}. Note that would have been the first edition from
1929 rather than the second edition from 1960.

Her second year at Vassar courses were taught by Katherine Prescott
Tinker.  Text books mentioned are \bt{Astronomy} by Russell, Dugan,
and Stewart which was the first text book to introduce the concept of
astrophysics in the \Ord{3}{rd} edition of 1945. Also mentioned is
Smart's \bt{Spherical Astronomy}.

Rubin's autobiography is \bt{Bright Galaxies, Dark Matter} (Woodbury,
American Institute of Physics, 1977.  She also has a chapter
in \bt{Annual Reviews of Astronomy and Astrophysics} volume 49 2011.
In addition, oral interviews with David DeVorkin are in the archive of
the Niels Bohr LIbrary and Archives of the American Institute of
Physics.

As a graduate student and instructor for the laboratory work in
astronomy at Cornell she used Shaw and Boothroyd \bt{A Manual of
Astronomy} \Ord{3}{rd} edition. This had been privately printed in 1942 for the use of
students at Cornell and then published in 1947, with revisions in 1958
and 1967. Shaw was the current chairman of the astronomy department and
Boothroyd had been so before him. In 1053 she attended the Wisconsin
Summer School (Phys Today 47 (12) 34 1994).


{\bf 2020-06-19}

AJB 03.04.151 H.A. Howe, \bt{Astronomical Books for the Use of Students}
Popular Astronomy 9 29, 61, 169, and 225. Obtained from ADS 2020-06-19.

{\bf 2019-10-17}

Alexander et.\ al., in their article {\it The Exploration History of
Europa} \cite{Alexander2009} reference a number of books in a
discussion of the history of the observation and study of Jupiter's
moon Europa.  The books range from Galileo Galilei's \bt{Sidereus
Nuncius} \cite{Galilei1989} to Arthur C.\ Clarke's novel \bt{2010:
Odyssey Two} \cite{Clarke1982}.

Laplace 1805, \bt{Mechnique Celeste, vol.\ 4} \cite{Laplace1805} first
demonstrated orbital resonance and was the first work to give a mass
for Europa and the other Galilean satellites. More modern estimates of
the masses are given in Brouwer and Clemence \cite{Brouwer1961} in
their review in \bt{Planets and Satellites} \cite{Kuiper1961} as well
as Kovalevsky \cite{Kovalevsky1970} in \bt{Surfaces and Interiors on
Planets and Satellites} \cite{Dollfus1970}.

Secchi in 1859 \cite{Secchi1859} provided an early result for the
diameter Europa which was 6\% too large.  Popular astronomy books by
Herschel (1861) \cite{Herschel1861} and Newcomb
(1878) \cite{Newcomb1878}.

He also mentions Russell et.\ al.\ (1941) \cite{Russell1941} textbook
\bt{Astronomy} as well as the textbook by Payne-Gaposchkin (1956),
\bt{Introduction to Astronomy} \cite{Payne-Gaposchkin1956} while he mentions
Urey (1952), in \bt{The Planets} \cite{Urey1952} with the idea that
the outer worlds are made of water ice.

Infrared spectra of satellites were extensively reviewed by Johnson
and Pilcher (1977) \cite{Johnson1977} for their article in the
University of Arizona Space Science Series book \bt{Planetary
Satellites} \cite{Burns1977}. A review of the discoveries about
the Jovian magnetosphere during the 1970's calls upon \bt{Physics of the
Jovian Magnetosphere} (1983) \cite{Dressler1983}.

The post-Pioneer summations are given in two volumes from the
University of Arizona Space Science Series \bt{Planetary
Satellites} \cite{Burns1977} and\bt{Jupiter} \cite{Gehrels1976}.


\bt{The Satellites of Jupiter} \cite{Morrison1982}
