%% Begin copyright
%%
%%  /home/jrf/Documents/books/Books20/Docs/Notes/biblio_notes.tex
%%  
%%   Part of the Books20 Project
%%
%%   Copyright 2010 James R. Fowler
%%
%%   All rights reserved. No part of this publication may be
%%   reproduced, stored in a retrival system, or transmitted
%%   in any form or by any means, electronic, mechanical,
%%   photocopying, recording, or otherwise, without prior written
%%   permission of the author.
%%
%%
%% End copyright

%%
%% notes and references in the books about bibliography
%%  read for this project. This document is included in Notes.tex
%%
{\bf 10/19/2010}
{\it Astronomischer Jahresbericht, volume 68 and 67} \cite{Ajb1900}

So the whole idea of trying to go through {\it Astronomischer
Jahresbericht} by hand was truly daunting and I avoided it for two
years.  However, it had to be done. I finally decided that by starting
at the last volume I would find more books in English as compared to
volume 1 (year 1899) and I could more easily figure out what the book
entries looked like.  Also if I worked my way backwards through the
book, starting at section 15, Star Systems, rather than in section 1,
History and Reports, or section 2 Border Fields, where the conference
proceeding are, the entries would be dominated by journal article
rather than books.  This turned out to be the case; indeed, sections 1
and 2 contributed at least 50 percent of the 247 books found for the year
1968.

Some general principles that seem to work, at least for the later
years of AJb. I'll add more as I work through the earlier volumes

\begin{itemize}

\item Look for a bold face number. This is usually a journal
volume number. Then look for the journal name preceding the
number. However, some journal, particularly Soviet ones, do not us a
journal number. Instead they use a Year and part/section number. For
example ``RJ UdSSR 1967 5.62.290'', see AJb 67.136.46.
\item Book entries are usually formatted as author, title, place, publisher,
year, page count, price, reviewed in.  Look for the work ``Pries''
(price) in the entry.  Page counts will be of the form ``nn+nnn S. +
nn Tafeln.''  where ``S.'' is the German abbreviation Seite, page, or
Seitenz\"{a}hlung, pagination, and ``Tafeln'' is the German word for
plates.
\item Try to avoid reading the actual entry.  They are interesting
but you are just wasting your time.
\end{itemize}

The main problem is determining what is a book. In general books have
publishers and page counts and prices.  Books do not have or are not
part of a regular sequence of publications, i.e.\ no volume numbers
tied to a particular year.  However, there are some entries that are
not journal publications but are not listed with a regular publisher,
The contributions from an observatory would be one example.  These may
be printed and bound in hardcover but may not have a price.  When in
doubt I include the entry.  It can always be removed later but it will
be difficult to include any potential books after the fact. I don't
want to have to go through the AJb again!

I do not include NASA Technical Notes (NASA-TN) or NASA Technical
Reports (NASA-TR) but I do include NASA Special Publication (NASA-SP).

Always proof read the list of book entries and compare to the entries
in AJb. I spent three days proof reading and parsing into Excel for
volume 68. This should be reduced for the other volumes as I was still
deciding on the format of my entries for this volume.

Note that the for conference proceedings the book is listed once in
the section on conference proceedings and all the articles are listed
by author under the various subject heading. So a 13 article volume of
proceedings would have 14 entries in AJB.

{\bf 10/23/2010}
I do not include SAO Special Reports as books. I do not include Part
4, section 34, Year books, Almanacs, and Ephemerides or Part 4 section
35 Eclipses and Chronology, although I do look through these section
just in case something looks interesting.

{\bf 11/6/2010}
AJB 66: is there a peak in book production during a
fields life time?  I note that for this volume photography is a mature
field in astronomy.  Do I expect any new stuff to be published in
books or will there just be refinements?

Noted that AJB 66.11.23 and AJB 67.11.18 seem to be the same book, R.C
Jennison, Introduction to Radio Astronomy. The only difference is in
the reviews and the page count, 160 vs 168.  Will have to look for
duplicates, prior editions and translations when the data base is built.


{\bf 21/3/2011}
AJB65: There was no copy in the McDonald Library so I had to go through
this one by looking at the GIFs in ARIBIB.  While proof reading this list
I noted that many authors appear every year.  There must be some folks
who have found a niche in writing books.  We can do a count by year of
various authors and define some metric to determine prolific authors.

{\bf 27/3/2011}
Procedure for transcribing AJB.
\begin {itemize}
\item Read AJB, transcribe book entries into a Word 2010 document using
the default UTF-16 encoding. Skip sections 34 and 35 in Part 4.
\item Proofread the Word document.
\item Save as a text document using UTF-8 encoding.
\item Read into Excel 2010 spreadsheet. Review all entries for proper placement
of commas.  Column seven should be the price. Fix all entries in the
Word document that parse incorrectly in Excel.
\item Save revised Word document as a text document named ajbYY\_books.txt using UTF-8 encoding.
\item Transfer Word and text documents to biblion into Books20/Data/Ajb.
\item Test the text document with the PERL parsing scripts in the directory Testing.
\end{itemize}

%% \vskip\baselineskip
