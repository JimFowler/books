%% Begin copyright
%%
%%  /home/jrf/Documents/books/Books20/Docs/Notes/world_history_notes.tex
%%  
%%   Part of the Books20 Project
%%
%%   Copyright 2018 James R. Fowler
%%
%%   All rights reserved. No part of this publication may be
%%   reproduced, stored in a retrival system, or transmitted
%%   in any form or by any means, electronic, mechanical,
%%   photocopying, recording, or otherwise, without prior written
%%   permission of the author.
%%
%%
%% End copyright

%%
%% notes and references in the books read for this project. This 
%% document is included in Notes.tex
%%
{\bf 1/1/2014}

{\it The Bully Pulpit: Theodore Roosevelt, William Howard Taft,
  and the Golden Age of Journalism}, Doris Kearns Goodwin,
  chapter 7. \cite{Goodwin2013}

The 1890's and early 1900's saw rising discontent with the industrial
order. Although economic and technical progress had boomed during the
industrial revolution, the gap between rich and poor had also skyrocketed.
With the closing of the American West there was not longer a way to 
escape from the misery. Most workers could not acquire the very goods and
benefits they were creating. The Sherman Anit-Trust act was passed in 1890
but the law was unenforced until much later.

However, this combination of startling progress and enormous accumulation
of wealth provided a new source of technology, wealth, and ego to fund
large observatories and universities.  Lick and Yerkes Observatories along
with the University of California and the new University of Chicago are
examples.

The collapse of the railroad boom in 1893, the most serious depression
in the US to date, slowed the boom but did not stop it.

\vskip\baselineskip
