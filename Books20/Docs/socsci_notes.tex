%%
%%
%% book_notes.tex
%%
%%
%% notes and references in the books read for this project. This 
%% document is included in Notes.tex
%%
%%
%%   Copyright 2014 James R. Fowler
%%
%%   All rights reserved. No part of this publication may be
%%   reproduced, stored in a retrival system, or transmitted
%%   in any form or by any means, electronic, mechanical,
%%   photocopying, recording, or otherwise, without prior written
%%   permission of the author.
%% 
%% The last known changes were checked in by $Author$
%% as revision $LastChangedRevision$
%% on $Date$
%%
%%
{\bf 3/6/2014}

{\it On the Frontier of Science: An American Rhetoric of Exploration
and Exploitation}, Leah Ceccarelli \cite{ceccarelli:2013}

At the Columbian Exposition of 1893, held in Chicago to celebrate the
400th anniversary of the ``discovery'' of America, Frederick Jackson
Turner discussed the differences between the American and the European
frontiers and argued that the American frontier had shaped the nature
and character of (white) Americans. Although Turner was trying to get
historians to be aware of the significance of the frontier, the idea of
the frontier and a pioneer spirit resonated with Americans and became part
of the popular culture of the \Cen{20}.

Turner expanded these notions in a speech given in 1910 where he
stated that with the closing of the geographical frontier Americans
must open up a new frontier in science and technology which have
unlimited resources of knowledge for conquest and discovery.

Ms.\ Ceccarelli then discusses how the rhetoric of the frontier of
science forms a ``terministic screen'' through which scientists look
at the outside world and themselves.  The choice of this particular
rhetorical path constrains how scientists can talk about themselves
and other; like blinders on a horse who can only see the road in front
of him, this device controls what scientists can see and say.

Within the general Rhetoric of Science, how scientists see and talk
about themselves, Ceccarelli is interested in the use of the frontier
of science metaphor as used within public speeches about science. She
discusses three people in particular, E.\ O.\ Wilson's speech about
biodiversity, Francis Collins' talk at the announcement of the completion
of the Human Genome, and George W. Bush's speech that closed the door
on stem cell research.

The book successfully trys to answer three questions:
\begin{enumerate}
  \item What is selected and what is deflected by this metaphor?
  \item What effects might these selections and deflections have on the
scientific research projects of those who use the metaphor?
  \item What moves do rhetors make when they try to escape the fly
paper trap of the metaphor?
\end{enumerate}

The question for me is to ask what is the rhetoric of the astronomers
that I read.  How do they present themselves when talking to other
astronomers, other scientists, to non-scientists (public speeches),
and to both through auto-biography and popular books?


%{\bf 3/6/2014}
