%%
%%
%% Notes.bib
%%
%%   Notes on the books and articles that have been read in conjunction
%%   with the book project "Some Important Books in Astronomy
%%    and Astrophysics in the 20th Century"
%%
%%   Copyright 2009 James R. Fowler
%%
%%   All rights reserved. No part of this publication may be
%%   reproduced, stored in a retrival system, or transmitted
%%   in any form or by any means, electronic, mechanical,
%%   photocopying, recording, or otherwise, without prior written
%%   permission of the author.
%%
%%
%% The last known changes were checked in by $Author$
%% as revision $LastChangedRevision$
%% on $Date$
%%
%%

%% 

\documentclass{article}

%----------------------------------------------------------
\newcommand{\Cen}[1]{\ensuremath{#1^{th}} Century}
%----------------------------------------------------------

\begin{document}

%%\frontmatter
\title{Some Notes on Books Read}
\author{James R. Fowler}
\date{copy of \today}
\maketitle

\section{Notes on Notes}

{\bf 10/19/2010} Format for Note Taking

Still can't quite make up my mind on how I want to take
notes. Handwritten notes on paper has been my medium of choice for 30
years now. However, since any writing I do will be on the computer
electronics notes are advantageous.  So paper notes that I might take
at times when my computer is not available will be transcribed into
electronic form. This will be a nuisance as it means writing notes
twice but it also means that notes are readable five years later.
When a computer is available notes will be written directly to
electronic form and if necessary, copied into this document.  By using
\TeX\ the notes can be formatted neatly although one has a lot more
flexibility with handwritten notes. On the other hand, with electronic
notes I can change my mind about the note during the composition.
Indeed, I changed by mind about the rules portion of this note; at
first is was to be a paragraph but I quickly realized that it would be
more readable as a list.

I need to consolidate the note taking in some preferred format.  Right now
I have paper notes as well as electronic notes on multiple computer
and in multiple formats.

My rules for taking notes:
\begin{itemize}
\item {\bf Always} include a date for the note. Put the date in bold face.
\item If a note is about a book or from a book, include the title,
  author and a citation from the master bibliography as well as
  chapter and page reference. The master bibliography should include
  Library of Congress and Dewey Decimal catalog numbers.
\item Never change an electronic note after it has been written. This
  includes misspellings. Changes may be made while composing notes but
  not the next day. Instead make a dated addition to the note. We wish
  to preserve the historical flow and avoid any {\it 1984} style of
  rewriting history.
\item Keep a paper copy. Scribble on it as necessary. Update this file.
The paper copy of this file should aways include the date that it was
processed from \TeX\ to PDF.
\end{itemize}

Some references that I have used for trying to find the best way to take
notes.
\begin{itemize}
\item My thirty years of reading and note taking mostly with pen and  paper.
\item Booth, W., et al.\ {\it The Craft of Research 3rd. ed.}
  \cite{booth:2008}
\item Cronon, William, {\it Learning to Do Historical Research}
  http://williamcronon.net/researching/index.htm specifically the
  section on note taking
  http://williamcronon.net/researching/notetaking.htm
  \cite{cronon:2008}
\item Cantor and Schnieder, {\it How to Study History} \cite{cantor:1967}
\end{itemize}


%%\mainmatter

\section{Notes about the History of Astronomy}\
%% Begin copyright
%%
%%  /home/jrf/Documents/books/Books20/Docs/Notes/18Cent.tex
%%  
%%   Part of the Books20 Project
%%
%%   Copyright 2009 James R. Fowler
%%
%%   All rights reserved. No part of this publication may be
%%   reproduced, stored in a retrival system, or transmitted
%%   in any form or by any means, electronic, mechanical,
%%   photocopying, recording, or otherwise, without prior written
%%   permission of the author.
%%
%%
%% End copyright

%% 
%%
%%   Notes on the books and articles that have been read in conjunction
%%    with 18 Century astronomy. For the book project "Some Important Books
%%     in Astronomy and Astrophysics in the 20th Century"
%%

{\bf 7/11/2009}

\bt{Lacaille: Astronomer, Traveler}, Davis S. Evans, \cite{Evans1992}
gives a brief introduction and overview of \Cen{18} astronomy.

Lacaille 1713---1762

Spain was losing its empire, Britain and France were gaining ground.
Many wars between the commercial empires for control of trade.  To
maintain their control of these commercial empires nations need good
protection and good navigation.  The British Board for the Discovery
of Longtitude was created in 1714 but Royal Societies had be sent up
in France and England in the \Cen{17}.  Physics was becoming
mathematical with the publication of Newton's Principia follewed by
additional work in Classical Mechanics. Precession, nutation, and the
abberation of starlight were known effects but there quatitative
contributions were not yet know with enough accuracy.  The lunar
method and the occultations of the satellites of Jupiter were known
to be able to measure longitude but the accuracy of star catalogs and
orbits was not know well enough to provide the accuarcy required.

Evans lists four problems that dominated the \Cen{18}.

\begin{itemize}

\item Law of atmospheric refraction --- necessary for the accurate
  determination of latitude.

\item Figure of the earth --- length of a degree of latitude was necessary
  for accurate maps of the earth.

\item Distance to the planets --- essentially the length of one
  astronomical unit.  Best method would be transit of Venus.

\item Refinement of earth's orbital parameters --- for calculating
ephemerides for naviagation and for accurate star catalogs.

\end{itemize}

{\bf 9/8/2012}

\bt{From Eudoxes to Einstein: A History of
Mathematical Astronomy}, C. M. Linton, \cite{Linton2004}

Chapter 9

page 291, ``The seventeenth century witnessed a complete
transformation in astronomy.  The Ptolemaic universe of uniform
circular motions disappeared and was replaced by a system based on
mechanical principles and Keplerian orbits. At the beginning of the
eighteenth century, the methods being used to analyse the motions of
the heavens were rooted still in the geometry of the ancients Greeks,
but by the end of the century this, too, had changed, with dynamics
reduced to the solution of differential equations: `physical
astronomy' became `celestial mechanics'.''

page 292, The major figures in the first half of the \Cen{18} were
Alexis-Claude Clairaut, Leonhard Euler, and Jean le Rond d'Alembert.

page 317, The second half of the \Cen{18} was dominated by Joseph Lous
Lagrange Pierre-Simon Laplace

page 305, Halley's comet returned in 1759. Belief in Newtonian
dynamics reinforced amoung the general public (cf. Waff 1986, JHA
17,1--37)

page 352, Near the end of the \Cen{17} Newton devised universal
gravitation, by the end of the \Cen{18} Laplace had practically
perfected it. A mechanistic universe, the eternal clockwork, seemed to
be the nature of the world.

Chapter 10

page 389, the prediction of Neptune by Leverrier and Adams built
further confidence in the mind of the general public about the truth
of Newtonian dynamics and the law of Universal Gravitation.

Chapter 11

The \Cen{19} saw an improvement in the techniques of celestial
mechanics with important work done by Hamilton and Jacobi but the
only real theoretical insight was that the Earth (and by implication
the other planets) was not a rigid body

page 405, ``In 1800, celestial mechanics was pursued for one purpose
only, i.e.\ to provide the theory necessary to produce accurate tables
with which to predict future positions of heavenly bodies. By 1900,
another competing goal had emerged as mathematicians began to analyse
the equations of celestial mechanics in their own right to see what
general conclusions could be drawn about the dynamics of the Solar
System.  The mathematical tools needed to do this came to the fore in
attempts to understand the motion of the Moon.''

page 406, Simon Newcomb made significant technical improvements
in celestial mechanics during the \Cen{18}.




\section{Note about Books}\
%% Begin copyright
%%
%%  /home/jrf/Documents/books/Books20/Docs/Notes/book_notes.tex
%%  
%%   Part of the Books20 Project
%%
%%   Copyright 2009 James R. Fowler
%%
%%   All rights reserved. No part of this publication may be
%%   reproduced, stored in a retrival system, or transmitted
%%   in any form or by any means, electronic, mechanical,
%%   photocopying, recording, or otherwise, without prior written
%%   permission of the author.
%%
%%
%% End copyright

%%
%% notes and references in the books read for this project. This 
%% document is included in Notes.tex
%%
{\bf 8/2/2009}

{\it A Splendor of Letters: The Permanence of Books in an Impermanent
  World}, Nicolas A.\ Basbanes, chapter 7. \cite{Basbanes2003}

Books are not only important for their contents they are also
important for the contextual information they provide about the
period, about the publishing, and about the author.  The materials
that make up the physical artifact, the cover, bookjacket,
etc.\ provide some commentary about the work and its place in the world.

Many books in the \Cen{20} were printed on acidic paper making them 
unlikely to survive for another century.


\vskip\baselineskip
{\bf 10/3/2009}

{\it Under Newton's Shadow: Astronomical Practices in the Seventeenth Century},
 Lesley Murdin, p 39, \cite{Murdin1985}

 The first 10 sections of Newton's Principia, 1686, were given as
 lectures at Cambridge in 1684.


\vskip\baselineskip
{\bf 1/4/2010}

{\it The Scientific Literature}, Joseph E.\ Harmon and Alan G.\ Gross,
\cite{Harmon2007}

A good short review of how science journal articles developed and changed
over time from the mid-\Cen{17} to the begining of the \Cen{21}.
 
\begin{itemize}
\item p XVII, Jan 1665, Journaldes S\c{c}avans (Journal of the Learned)
first published in Paris by Denys de Sallo.
\item Mar 1665 Philosophical Transactions first published in London for
the Royal Society by Henry Olenburg.
\item p XVIII, prior to the journals publication was by letters (limited
circulation) or by books (long gestation period)
\item p XIX, goal to show variety of written and visual expression over time.
\begin{enumerate}
  \item focus on equations, tables, and visuals as well as words  
  \item include commentary about selections
  \item include articles from all over the world
  \item tell story about the origin of the scientific journal
\end{enumerate}
\item p XX limitations, only brief article selection, limited to western science,
too few women.
\item p XXII, writing style of article depends on the goal of the author,
theoretical, experimental, observational, or review.
\item p 2, books were important for communicating the scientific revolution
1600-1700, new journals were seen as ancillary, more formal letters.
\item p 4, Philosophical Transactions avoids ``fine speaking'' in preference
for the facts.
\item p 5 no formal style but modern parts, introduction, methods and materials,
results, and discussion are there.
\item p 7, in the history of science new theories grab the headlines, but progress
is made mostly by creation and coninual refinement of new research tools, c.f.\ Kuhn, Scientific Revolution.
\item p 13 multiple voices in early journals, the editor commenting on a recieved
communication, the author relating observations, only latter do we find
the impersonal voice of nature speeking through the scientist.
\item p 37 1752 Philosophical Transactions institutes an editorial committee
to review articles, 1831 it institutes outside peer review.
\item p 76, increased specialization in journals, in 1863 there were 1500 journals
while in 1763 there were only 100.
\item p 301, 1858 Darwin publishes a brief article in the Journal of the
Proceedings ore the Linnaen Society in order to establish his independent 
claim with Wallace. The article has no impact but his book in 1859 is
instantly controversial.
\item p 188, modern organization of articles reached in 1830-1840 in
German chemical literature.
\item p 190, review articles describe and evaluate the recent literature
in the field.  They also serve another purpose, that of culling unfit
knowledge from the succesful ones. Review article generally have high
citation rates.
\item p 192, abstracts first appears in the 1920s but did not become
routine practice until the 1950s.
\item p 193, the introduction defines the historical context of the probem,
states the specific problem the paper looks at, and finally summarizes to
solution found.
\end{itemize}


\vskip\baselineskip
{\bf 2/6/2010}

{\it Scientific Books, Libraries and Collectors} John L.\ Thornton
and R.\ I.\ J.\ Tully \cite{Thornton1971}

\begin{itemize}
\item page 200 ``Classics of science may not be recognized until a
considerable period after publication, and best-sellers of today may
be entirely neglected a few years after their appearance. It is
therefore dificult to assess the value of comparatively modern
literature, and in this chapter we mention merely the high lights in
each major branch of science.

In some subjects in is impossible to indicate certain books as
outstanding contributions to scientific thought, either because the
more eminent exponents did not write books, or their discoveries were
announced in the periodical press. Astronomy, for example, made
extraordinary progress, largely as the result of great improvements in
the size of the telescopes employed.  Yet few eminent books devoted to
the subject were published in the nineteenth century. This despite the fact
that several prominent astronomers, both amateur and professional, were
recording important discoveries in the heavens.''

\item p 294 ``Every investigation must begin with a bibliography,
and end with a better bibliography.'' George Sarton, {\it A Guide to
the History of Science}, Chronica Botanica, 1951, Waltham MA \cite{Sarton1952}

http://www.archive.org/stream/guidetohistoryof00sart/guidetohistoryof00sart\_djvu.txt

Q124 S24, 509 S251G
\end{itemize}

\vskip\baselineskip
{\bf 8/22/2010} Comments about why one publishes books

``The field is mature enough that it can benefit from highly
technical textbooks, meant to be archival in nature, that present
penetrating analyses.  But since computational science (and scientific
programming) is interdisciplinary, practiced by scientists from
various fields, it is also critical that thinking patterns and
problem-solving methodologies are clear to each member of a research
team. Therefore, an important role remains for tutorials that make
obvious the author's expertise and thinking patterns.''

quoted from D.\ E.\ Stevenson (Clemson University, Clemson, SC) in a
book review of {\it Scientific Computation}, Gaston H.\ Gonnet and
Ralk Scholl (Cambridge University Press, 2009, ISBN 978-0-521-84989-0)
in Physics Today, v63 number 8, August 2010, p50.

``I was expecting a clear exposition of the authors' viewpoints,
compared to others, on the points of contention. But I was left
disappointed. This poorly edited book largely focuses on the authors'
own papers; rare acknowledgments of others are just to point out that
they are wrong.''

quoted from Steve K.\ Lamoreaux (Yale University) in a book review of
{\it Advances in the Casimir Effect}, M.\ Bordag, G.\ L.\
Klimchitskaya, U.\ Mohideen, and V.\ M.\ Mostepanenko, (Oxford
University Press, 2009, ISBN 978-0-19-923874-3) in Physis Today, v63,
number 8, August 2010, p50.

\vskip\baselineskip
{\bf 2/6/2010}

{\it The Mount Wilson Observatory} Allan Sandage, Vol.\ 1, The
Centennial History of the Carnegie Institution of Washington,
\cite{Sandage2004}

\begin{itemize}

\item page 58, ``Young included a diagram showing these differences in
his {\it General Astronomy}, a text-book published in 1893. As updated
by Russell, Dugan, and Stewart in 1927, it became the canon that all
astronomers learned from in the 1920s into the 1050s.''

\end{itemize}

\vskip\baselineskip
{\bf 5/11/2020}

National Academy of Sciences Decadal Surveys in Astronomy

\url{http://sites.nationalacademies.org/cs/groups/ssbsite/documents/webpage/ssb_189906.pdf}



\section{Notes about Bibliographies}\
%%
%%
%% biblio_notes.tex
%%
%%
%% notes and references in the books about bibliography
%%  read for this project. This document is included in Notes.tex
%%
%%
%%   Copyright 2010 James R. Fowler
%%
%%   All rights reserved. No part of this publication may be
%%   reproduced, stored in a retrival system, or transmitted
%%   in any form or by any means, electronic, mechanical,
%%   photocopying, recording, or otherwise, without prior written
%%   permission of the author.
%% 
%% The last known changes were checked in by $Author$
%% as revision $LastChangedRevision$
%% on $Date$
%%
%%
{\bf 10/19/2010}
{\it Astronomischer Jahresbericht, volume 68 and 67} \cite{ajb:1900}

So the whole idea of trying to go through {\it Astronomischer
Jahresbericht} by hand was truly daunting and I avoided it for two
years.  However, it had to be done. I finally decided that by starting
at the last volume I would find more books in English as compared to
volume 1 (year 1899) and I could more easily figure out what the book
entries looked like.  Also if I worked my way backwards through the
book, starting at section 15, Star Systems, rather than in section 1,
History and Reports, or section 2 Border Fields, where the conference
proceeding are, the entries would be dominated by journal article
rather than books.  This turned out to be the case; indeed, sections 1
and 2 contributed at least 50 percent of the 247 books found for the year
1968.

Some general principles that seem to work, at least for the later
years of AJb. I'll add more as I work through the earlier volumes

\begin{itemize}

\item Look for a bold face number. This is usually a journal
volume number. Then look for the journal name preceding the
number. However, some journal, particularly Soviet ones, do not us a
journal number. Instead they use a Year and part/section number. For
example ``RJ UdSSR 1967 5.62.290'', see AJb 67.136.46.
\item Book entries are usually formatted as author, title, place, publisher,
year, page count, price, reviewed in.  Look for the work ``Pries''
(price) in the entry.  Page counts will be of the form ``nn+nnn S. +
nn Tafeln.''  where ``S.'' is the German abbreviation Seite, page, or
Seitenz\"{a}hlung, pagination, and ``Tafeln'' is the German word for
plates.
\item Try to avoid reading the actual entry.  They are interesting
but you are just wasting your time.
\end{itemize}

The main problem is determining what is a book. In general books have
publishers and page counts and prices.  Books do not have or are not
part of a regular sequence of publications, i.e.\ no volume numbers
tied to a particular year.  However, there are some entries that are
not journal publications but are not listed with a regular publisher,
The contributions from an observatory would be one example.  These may
be printed and bound in hardcover but may not have a price.  When in
doubt I include the entry.  It can always be removed later but it will
be difficult to include any potential books after the fact. I don't
want to have to go through the AJb again!

I do not include NASA Technical Notes (NASA-TN) or NASA Technical
Reports (NASA-TR) but I do include NASA Special Publication (NASA-SP).

Always proof read the list of book entries and compare to the entries
in AJb. I spent three days proof reading and parsing into Excel for
volume 68. This should be reduced for the other volumes as I was still
deciding on the format of my entries for this volume.

Note that the for conference proceedings the book is listed once in
the section on conference proceedings and all the articles are listed
by author under the various subject heading. So a 13 article volume of
proceedings would have 14 entries in AJB.

{\bf 10/23/2010}
I do not include SAO Special Reports as books. I do not include Part
4, section 34, Year books, Almanacs, and Ephemerides or Part 4 section
35 Eclipses and Chronology, although I do look through these section
just in case something looks interesting.

{\bf 11/6/2010}
AJB 66: is there a peak in book production during a fields life time?
I note that for this volume photography is a mature field in
astronomy.  Do I expect any new stuff to be published in books or will
there just be refinements?

Noted that AJB 66.11.23 and AJB 67.11.18 seem to be the same book, R.C
Jennison, Introduction to Radio Astronomy. The only difference is in
the reviews and the page count, 160 vs 168.  Will have to look for
duplicates, prior editions and translations when the data base is built.


{\bf 21/3/2011}
AJB65: There was no copy in the McDonald Library so I had to go through
this one by looking at the GIFs in ARIBIB.  While proof reading this list
I noted that many authors appear every year.  There must be some folks
who have found a niche in writing books.  We can do a count by year of
various authors and define some metric to determine prolific authors.

%% \vskip\baselineskip


%\section{History of Reading}\
%\input{hist_reading_notes}
%
%\section{Sociology of Science}\
%%% Begin copyright
%%
%%  /home/jrf/Documents/books/Books20/Docs/Notes/socsci_notes.tex
%%  
%%   Part of the Books20 Project
%%
%%   Copyright 2018 James R. Fowler
%%
%%   All rights reserved. No part of this publication may be
%%   reproduced, stored in a retrival system, or transmitted
%%   in any form or by any means, electronic, mechanical,
%%   photocopying, recording, or otherwise, without prior written
%%   permission of the author.
%%
%%
%% End copyright

%%
%% notes and references in the books read for this project. This 
%% document is included in Notes.tex
%%
{\bf 3/6/2014}

\bt{On the Frontier of Science: An American Rhetoric of Exploration
and Exploitation}, Leah Ceccarelli \cite{Ceccarelli2013}

At the Columbian Exposition of 1893, held in Chicago to celebrate the
400th anniversary of the ``discovery'' of America, Frederick Jackson
Turner discussed the differences between the American and the European
frontiers and argued that the American frontier had shaped the nature
and character of (white) Americans. Although Turner was trying to get
historians to be aware of the significance of the frontier, the idea of
the frontier and a pioneer spirit resonated with Americans and became part
of the popular culture of the \Cen{20}.

Turner expanded these notions in a speech given in 1910 where he
stated that with the closing of the geographical frontier Americans
must open up a new frontier in science and technology which have
unlimited resources of knowledge for conquest and discovery.

Ms.\ Ceccarelli then discusses how the rhetoric of the frontier of
science forms a ``terministic screen'' through which scientists look
at the outside world and themselves.  The choice of this particular
rhetorical path constrains how scientists can talk about themselves
and other; like blinders on a horse who can only see the road in front
of him, this device controls what scientists can see and say.

Within the general Rhetoric of Science, how scientists see and talk
about themselves, Ceccarelli is interested in the use of the frontier
of science metaphor as used within public speeches about science. She
discusses three people in particular, E.\ O.\ Wilson's speech about
biodiversity, Francis Collins' talk at the announcement of the completion
of the Human Genome, and George W. Bush's speech that closed the door
on stem cell research.

The book successfully trys to answer three questions:
\begin{enumerate}
  \item What is selected and what is deflected by this metaphor?
  \item What effects might these selections and deflections have on the
scientific research projects of those who use the metaphor?
  \item What moves do rhetors make when they try to escape the fly
paper trap of the metaphor?
\end{enumerate}

The question for me is to ask, what is the rhetoric of the astronomers
that I read.  How do they present themselves when talking to other
astronomers, other scientists, to non-scientists (public speeches),
and to both through auto-biography and popular books?


{\bf 4/22/2014}
\bt{The Social History of Knowledge: From Gutenburg to Diderot},
Peter Burke, \cite{Burke2000}

Chap 1. Discussion of the early history of the sociology of knowledge
and the purpose of the book.

Chap 2. Who were the intellectuals (clerisy) in early modern Europe?

Chap 3. What were the institutions involved in knowledge?

Chap 4. Where were the clerisy and institutions located? Also a discussion
of the centres and periphery in the context of these locations.

Chap 5. How was knowledge classified and how did this change with time?

{\bf 1/4/2022}
\bt{Revealing the Universe: Prediction and Proof in Astronomy},
James Cornell and Alan Lightman, eds., \cite{Cornell1982}

This book explores the relationship between theory and observations
using several topics of current (1982) interest in astronomy and
astrophysics. In the preface the editors note that theory without
observations can only suggest possibilities while observations without
theory is just a jumble of facts.  The book covers 6 topics, Einsteins
theory of relativity, the evolution of the solar system, the heating of
the sun's corona, quasars and black holes as energy sources, x-ray bursts,
and the age of the universe. The is also an introduction by Owen Gingerich
and a final chapter on three unanswered questions by George Field.

The editors note (page ix) that typically theory and observations are
not separate activities but that in practice observers reference and
develop theory while theoreticians utilize and even make observations
to verify theories.

In the opening chapter by Gingerich (page 1) he states, ``{\it
Science} (sic), \dots is not a noun at all; it is a verb.''  Science,
he says, ``is a way of discovering what the natural world is about by
conjecturing theories and by subjecting these theories to the test of
experimentation and observation.''  But he (as do the editors) note
that there are human factors at play so that theory and observation
can be confused by error and bias.  Gingerich also notes (page 12) the
that tension between theory and observation would entirely disappear
if the observations of perfect and unambiguous.

Rob Rosner (page 101), in his chapter on the heating of the solar
corona states that the connection between theory and observation is
indirect for two reasons. First, a rigorous theory, at least for the
complex solar corona, can not be constructed that accounts for all the
data. Second, in this case, laboratory data for the high temperature,
low density atoms can not yet be obtained.  In such situations Rosner
links theory and observations through a ``scenario'', which he calls a
plausible story linking the theory and the limited data.  He notes
that this is a metatheory that provides a framework and tools for
comparing predictions and data.  A good illustration is on page 102.


%%\backmatter
\vfil\eject

\bibliographystyle{plain}
\bibliography{MasterBib}

\end{document}
