%%
%%
%%  /home/jrf/Documents/books/Books20/Docs/DatabaseDesign/DatabaseDesign.tex
%%  
%%   Part of the Books20 Project
%%
%%   Copyright 208q8 James R. Fowler
%%
%%   All rights reserved. No part of this publication may be
%%   reproduced, stored in a retrival system, or transmitted
%%   in any form or by any means, electronic, mechanical,
%%   photocopying, recording, or otherwise, without prior written
%%   permission of the author.
%%

%%
%%  The design and planning for the Books20 database
%%
\documentclass{article}

\usepackage{books20}
\addglobalbib{./../../Docs/MasterBib.bib}

  
\begin{document}

\title{Books20: Project Plan}
\author{James R. Fowler}
\maketitle
\vfil\eject

\vspace*{5 in}
\ProjectCopyright{2009}

\vfil\eject

%
% Goals and Purpose of the Project
%
\section{Goals of the Project}
The primary goal of the Books20 project is to explore the question,
What are the important and influential books in astronomy and
astrophysics from the 20th Century? I come at this problem as a
working astronomer and bibliophile so my original intent was to
explore what were the collectable books from the 20th
Century. Important 20th Century books are generally less expensive
than the important books from the 19th or 18th Century.  Some of the
original questions included how to find out what books were published
during the 20th Century in astronomy and astrophysics and what makes a
book important? This led to a series of auxiliary questions such as,
how did astronomers use books during the 20th Century as information
transmission in astronomy transitioned from books to journals to the
web; what were publishers doing as the industry changed; what were the
important topics in astronomy during the 20th Century; how were books
used in the 19th Century and how was that different from the 20th
Century.  These questions lead me to research in the History of
Astronomy, Reading, Publishing and the Sociology of Science.

Although I had been thinking about these questions for many years the
formal start of the project was Nov 2008. With a full time job and
other volunteer activities I manage to get a few hours a week for the
project. During Nov '08 I spent the time thinking about the questions
and how to approach them.  I began looking for sources and references
and I took out a Community Borrowers card from the Sul Ross State
University library in Alpine.  The month of December '08 was spent in
the Sul Ross, McDonald Observatory, and University of Texas at Austin
libraries looking at books on the history of astronomy. The main
(re)discovery was of Astronomy and Astrophysics Abstracts and
Astronomischer Jahresbericht which lists all publications in astronomy
from 1899 to the current time. A complete set of these books are in
the McDonald Observatory library. During this time I also began a
draft of the project report, written in LaTeX, originally under
Windows Vista and TeXnic Center (www.texniccenter.org); this is a very
nice editor.  The draft gave a brief summary of the questions to ask,
a history of astronomy in the 19th and 20th Centuries, and a first
attempt at formatting the book list.  I began to have some idea of the
scope of what I was taking on.


%
% Design of the database
%
\section{Database Design}

The Books20 database is a relational database containing tables for
the basic book, author, and publisher information as well as a linking
tables between the books and authors.  The database is designed around
three views of the data, first, from the Web were a user can search
for information about a particular book or perform a search for a
particular class of books.  The second view is a reporting view. The
primary view is will be a sequential presentation of the books in
copyright sequence order with the output in TeX for incorporation in
the manuscript. Other views may be incorporated but these views are
principally for the project personel. The finally view is for the
editors.  It allows them to update information about a book, add a new
book, and change the status of a book from prospective to included.

\subsection{Database Tables}
\subsubsection{The Book Table}

The Book table contains the basic publishing information about a book
as well as bibliographic information and the project information. Such
items as title and subtitle, copyright year, the printing and edition,
the publisher and the location of publication. Note that author
information is handled in the BookAuthor table. For books that have
one, the ISBN number is included.  ISBN numbers were introduced in
1970 as 10 digit number but became thirteen digits in 2007.  A
checksum calculation to verify the validity of the ISBN should be
included in the entry form. The bibliographic description should
include the size of the book in millimeters, page counts, whether the
book is illustrated, figure count, table count and index count.  This
information should also include information about the binding and
about the dust jacket if the book was issued with one. Project
information will include a flag stating that this item is included in
the book as well as flags indicating whether the book is a sub item of
another book, e.g.\ a second edition or part of a series.

\begin{itemize}
\item Book Table
\begin{itemize}
 \item Book ID
 \item Publishing information
 \begin{itemize}
  \item Title
  \item sub-Title
  \item Copyright Year
  \item City of Publication
  \item Publisher ID
  \item Edition
  \item Printing
  \item ISBN
 \end{itemize}
 \item Bibliographic information
 \begin{itemize}
  \item Height in mm
  \item Width in mm
  \item Page counts
  \item Figure, table, index, illustration count
  \item Binding
  \item Dust jacket if issued
 \end{itemize}
 \item Project information
 \begin{itemize}
  \item Keywords
  \item Flag: include in Books20
  \item Parent Book if any
  \item Links to other bibliography lists
  \item Link to online book (google, amazon, etc.)
  \item Comment about why this book is important
 \end{itemize}
\end{itemize}
\end{itemize}


\subsubsection{The Author Table}

The Author table contains as complete a set of information about the
authors of the books as I can obtain.  At a minimum it contains an
author's last name. But may include first name, any middle names,
birth and death dates, as well as any biographical information about
the author. The database may contain a separate section with comments
about an author that will be included in the Books20 publication. Note
that some books may not have a published author so we need to account
for a null author. We also include links or references to other
biographical lists such as the Biographical Encyclopedia of Astronomer
or the Dictionary of Scientific Biography.  Consider importing the
Author list from the collection database as a starting point.

\begin{itemize}
\item Author Table
  \begin{itemize}
  \item Author ID
  \item Last name
  \item First name
  \item Middle names
  \item Born
  \item Died
  \item Biographical comments
  \item Links to other author lists (BEA; DSB; etc.)
  \item Books20 comments
  \item Flag: include comment
  \end{itemize}
\end{itemize}

\subsubsection{The Publisher Table}

The Publisher table contains information about the publishers of the
books.  The important information is the publisher's name and address
but any information about prior names or dates that the publisher
existed would be important. Finally we include any URLs as well as a
comment section that might be included in the Books20 publication.\

\begin{itemize}
\item Publisher Table
  \begin{itemize}
  \item Publsher ID
  \item Publisher Name
  \item Address
    \begin{itemize}
    \item Street 1
    \item Street 2
    \item City
    \item State or Province
    \item Country
    \item Postal Code
    \end{itemize}
  \item URL
  \item Use to be 1 (Publisher ID)
  \item Use to be 2 (Publisher ID)
  \item Use to be 3 (Publisher ID)
  \item Comment
  \item Books20 comment
  \item Flag: include comment?
  \end{itemize}
\end{itemize}

\subsubsection{The BookAuthor Table}

The BookAuthor table contains the links between books and authors.
This is neccesary because Book to Author is a many-to-many
relationship.  That is, one book many have many authors and one author
may write many books.  We need to state what order this author is for
this particular book and how this author's name was written on the
title page of this book. This may be different for the same author in
different books. We also include a flag is this link describes an
editor rather than an author. Finally, if we wish to add a comment
about that particular author with regard to this particular book we
include a comment section to be included in the Books20 publication.

\begin{itemize}
\item BookAuthor Table
  \begin{itemize}
  \item Book ID
  \item Author ID
  \item Priority (first, second, third author ...)
  \item As Written
  \item Editor?
  \item Books20 comments
  \item Flag: include comment?
  \end{itemize}
\end{itemize}
%
% Software
%
\section{Supporting Software}
The database software is MySql v5.1 (www.mysql.com) with the interface written
in PhP v 5.2.10 (www.php.net).

MySql issues: security! internationalization, localization (see the MySql
developer web site)

Documentation sofware is \TeX/\LaTeX, issues define the format of the manuscript.
This can be done late in the process.

Version control software is Subversion v1.6.5.

Blogging software is Apache Roller under Apache Tomcat 6.

%
% The End
%
\end{document}

%%
%% end of DatabaseDesign.tex
%%
