%% Begin copyright
%%
%%  /home/jrf/Documents/books/Books20/Docs/Style/style_latex.tex
%%  
%%   Part of the Books20 Project
%%
%%   Copyright 2018 James R. Fowler
%%
%%   All rights reserved. No part of this publication may be
%%   reproduced, stored in a retrival system, or transmitted
%%   in any form or by any means, electronic, mechanical,
%%   photocopying, recording, or otherwise, without prior written
%%   permission of the author.
%%
%%
%% End copyright

%%
%%  The style that I will use for LaTeX 2e documents 
%%
\section{General Comments}

Documents for the \ProjectTitle\ project should be written in \LaTeXe\ and use
the books20.sty package. Most documents will be articles but the
manuscript should be in the books document class. The reasoning behind
using \LaTeXe\ is that I expect academic publishers to prefer or at
least have available a \LaTeXe\ class package.  It will be difficult if
I need to convert these documents to Microsoft Word\texttrademark or
some other formatting package.

A good, general purpose on-line reference for \LaTeXe\ can be found at
the Share \LaTeX\ web site,
\href{http://www.sharelatex.com/learn}{www.sharelatex.com/learn}.
More detailed information can be found at The Comprehensive
\TeX\ Archive Network, \href{ctan.org}{ctan.org}

The \texttt{books20} style package imports the following packages so they are
not required in your \LaTeXe\ documents

\begin{itemize}
\item inputenc \cite{Jeffrey2018} (with the utf8 option)
\item hyperref \cite{Rahtz2017}
\item longtable \cite{Carlisle2014}
\item import \cite{Arseneau2009}
\item biblatex \cite{Lehman2018} (with biber backend and
  \texttt{./Books20/Docs/MasterBib.bib} as the master bibliography file)
\end{itemize}


\subsection{Document Design}

\begin{itemize}
  
\item The \LaTeXe\ package \texttt{books20} shall be used in all
  \LaTeXe\ documents. To include it use \verb|\usepackage{books20}|, there
  are no options to this style file at this time. Note that books20.sty
  is designed for \LaTeXe\ and may not work with \LaTeX\ 2.09.
  

\item The package \texttt{import} allows you to place sections in a
  separate file and use the \verb|\import{}{}| command in the document
  to include the file within the document. The first argument is the
  directory path, the second argument is the file name. As an example,
  use \verb|\import{./}{style_tex}|.  By using the \texttt{import}
  package it is possible to use the same file in different documents
  without having to clean up the file. The \verb|\subimport{}{}|
  command has the same calling syntax but the file locations are
  relative to the first \verb|\import|. It is possible to nest
  \verb|\import| commands whereas it is not possible to nest
  \verb|\include| commands. This feature shall be used inLaTeX2e\
  documents to allow greater flexibility for reuse. The imported
  files may have \verb|\section| commands but should not have
  \verb|\chapter| commands.

\item {The input encoding package \texttt{inputenc} shall be used
  so that UTF-8 character encoding can be utilized. Note that this
  package is included automatically with in the \texttt{books20} package.

  Then a sentence might be written as
  
  \verb| Das astronomische Weltbild gemäß jüngster Forshung|
  to produce

  Das astronomische Weltbild gemäß jüngster Forshung

  But you can still use the standard \LaTeX\ encoding
  
  \verb| Das astronomische Weltbild gem\"{a}{\ss} j\"{u}ngster Forshung|
  to produce the same thing.

  Das astronomische Weltbild gem\"{a}{\ss} j\"{u}ngster Forshung

  This will be useful when I create \LaTeXe\ files from the database
  which is in UTF-8 encoding.
}

\item In Emacs we can produce UTF-8 characters with C-x 8 mark letter.
  For example, to produce ö use the key strokes C-x 8 \verb|"| o. Use C-x 8 C-h
  to get a list of character encoding.
  
\item {Use the \texttt{biblatex} package rather than the
    \texttt{bibtex} package. The \texttt{biblatex} package allows both
    global and local bibliography files and is provided with the
    \texttt{books20} package.  For example, we can cite this book by
    Basbanes \cite{Basbanes2003} from the global file and we can cite
    this web page with the documentation for the \texttt{biblatex}
    package \cite{Lehman2018} from the local file. The file
    \texttt{./Docs/MasterBib.bib} is the global bibliography file and
    must be included in the document using the
    \verb|\addglobalbib{../MasterBib.bib}| command. You may use
    additional local biblography files as well with this package. Use
    the command \verb|\addbibresource{loacalbib.bib}|. These commands
    should be in the preface before the \verb|\begin{document}|
      command. The choice of which references should be global and
      which should be local will probably change with time.

      Need to work out how to handle link resources and references in
      footnotes, particular URLs in footnotes. Should all URLs be at
      the back of the document in the bibliography?  This is possible
      if we use local .bib files }

\item May wish to consider the use of the \texttt{subfiles} or
  \texttt{standalone} packages if the number of files gets
  large. These packages allow you to precompile files and avoid long
  compiles, simlar to the \texttt{make} command in Unix.  However,
  these packages may conflict with the \texttt{import} package.


\end{itemize}

\section{Defined Commanda in books20.sty}

\begin{itemize}

\item \verb|\Cen{19}| is used to produce century names,
  e.g.\ \Cen{19}. Note that it works only for centuries that use the
  \textsuperscript{th} format.  That is, it won't work for the
  21\textsuperscript{st} Century, the 2\textsuperscript{nd} Century,
  or the 3\textsuperscript{rd} Century, etc. I should probably fix
  this issue.

\item \verb|\bt{Book Title}| produces book titles in italic face,
e.g.\ \bt{Book Title} is the title of a book.

\item \verb|\ProjectTitle\ | produces the project title in bold face,
  e.g.\ the \ProjectTitle\ project is the name of this project.

\item \verb|\BookTitle\ | produces the book title in italic face,
  e.g.\ \BookTitle\ or whatever I (or my editor) chooses for a title.

\end{itemize}

\section{Still to Do}

\begin{itemize}

\item How to handle internal and external references.  Written
  documents will be used both as paper documents and as on-line
  eDocuments and the full link should be available in either
  case. External references should be handled in the bibliography
  package. Internal references need to be researched.  As an example,
  there is a need for mutual references between this chapter and the
  chapter on English Grammer and Style.

\item Determine a good (useful) format style for citations and the
  bibliography.  Then figure out how to do it in \texttt{biblatex}.
  Consider \texttt{(ref page \#)} which could be implemented as
  \verb|\cite[page 34]{Carlisle2014}| which should produce something
  like this \cite[page 34]{Carlisle2014}. The exact appearance will
  depend on the bibstyle file used with \texttt{biblatex}

\item Work out other font formats as required. For example, an included
  file or package should be in teletype format using \verb|\texttt{}|.
  Although there will not be much call for such in the manuscript.

\item Convert series.tex, handbuch.tex, the files in \texttt{./Docs}, as
  well as the manuscript to use the \texttt{books20} package.

\end{itemize}

\section{Bib\TeX\ and Bibliography}

I use the Bib\LaTeX\ style and Jabref \cite{Jabref} as my bibliography
tool.  When adding a new entry to the Jabref also add a note about where
the document was referenced. As well remember to add keywords so that we
can sort the items.

\section{An Example Document}

Here is an example document so you can see what the \LaTeX\ commands are.

\begin{verbatim}

%%
%%
%% style_guide.tex
%%
%%   The style that I will used for 
%%   the book project "Some Important Books in Astronomy
%%    and Astrophysics in the 20th Century"
%%
%%   Copyright 2017 James R. Fowler
%%
%%   All rights reserved. No part of this publication may be
%%   reproduced, stored in a retrival system, or transmitted
%%   in any form or by any means, electronic, mechanical,
%%   photocopying, recording, or otherwise, without prior written
%%   permission of the author.
%%
%%
%%
\documentclass{book}

\usepackage{books20}
\addglobalbib{./../../Docs/MasterBib.bib}
\addbibresource{localbib.bib}

\begin{document}
\title{Style Guide for the \ProjectTitle\ Project}
\author{James R. Fowler}
\date{2017-12-15}

\maketitle
\tableofcontents
\listoftables 

\Chapter{\LaTeX2e\ Documents}
\import{./}{sytle_latex}

\chapter{Python Programming}
\import{./}{style_python}
\end{document}

\end{verbatim}

 The following would be imported from  the style\_latex.tex document.

\begin{verbatim}

\section{General Comments}

The first of many sections for the book \BookTitle.  You can cite 
Basbanes \cite{Basbanes2003} book on libraries like this.

\section{Using UTF-8}

You can use UTF-8 characters by using the \textttusepackage[utf-8]inputenc}
package.  Then a sentence might be written as

Das astronomische Weltbild gemäß jüngster Forshung

But you can still use the standard \LaTeXe\ encodings

Der neuentdeckte Himmel; Das astronomische Weltbild gem\"{a}{\ss} j\'{u}ngster Forshung

\subsection{Long Tables}

You can make a table in longtable format.

\begin{longtable}[p]{l l l}
  \caption{\bf Harvard Observatory Monographs} \\
  \label{HMA:1} \\

  Title & Author(s) & Date \\
  \hline\hline
  \endfirsthead

  \multicolumn{3}{c}{Continuation of Harvard Obs.\ Monographs} \\
  Title & Author(s) & Date \\
  \hline\hline
  \endhead

  \hline
  \endfoot
  
  \hline\hline
  \endlastfoot

  1 \bt{Stellar Atmospheres} & Cecilia Payne & 1925 \\

  2 \bt{Star Clusters} & Harlow Shapley & 1930 \\

  3 \bt{The Stars of High Luminosity} & Cecilia Payne & 1930 \\

  
\end{longtable}

\printbibliography

%% End style_latex.tex
\end{verbatim}

