%% Begin copyright
%%
%%  /home/jrf/Documents/books/Books20/Docs/Style/style_guide.tex
%%  
%%   Part of the Books20 Project
%%
%%   Copyright 2018 James R. Fowler
%%
%%   All rights reserved. No part of this publication may be
%%   reproduced, stored in a retrival system, or transmitted
%%   in any form or by any means, electronic, mechanical,
%%   photocopying, recording, or otherwise, without prior written
%%   permission of the author.
%%
%%
%% End copyright

%%
%%  The style that I will use for for various aspects of the project
%%
\documentclass[twoside]{book}

\usepackage{books20}
\addglobalbib{./../../Docs/MasterBib.bib}
\addbibresource{localbib.bib}

\begin{document}

\frontmatter

\title{Style Guide for the \ProjectTitle\ Project}
\author{James R. Fowler}
\date{2017-12-15\\ updated 2020-03-14\\ copy of \today}

\maketitle

\tableofcontents

\mainmatter

\chapter{General Comments}

\section{Purpose}
This style guide is about maintaining consistency within the documents
of my project \ProjectTitle. Consistency is for readablilty and
helps avoid conflicting statements between documents. The tools
that I am using are,

\begin{itemize}
\item \texttt{Ubuntu} Linux at the current LTS release,
\item \texttt{Emacs} as the text editor,
\item \LaTeXe\ as the document processor,
\item with \texttt{BibLatex} bibliography control,
\item \texttt{Python} as the software development language,
\item \texttt{Sphinx} as the Python documentation language,
\item \texttt{Git} as the version control system,
\item with \texttt{GitHub} as the off-site repository,
\item \texttt{sqlite3} is used for the database structure,
\item \texttt{XML} is used for the intermediate data lists until
  I can migrate the data to the database,
\item Gnu \texttt{autotools} as the build environment (autoconf, automake,
  etc.),
\item Gnu \texttt{Make} is used as the build tool,
\item possibly \texttt{R} as the statistical analysis language,
  if I ever have to do detailed statistics, otherwise I'll just
  use the statitics package in python,
\item with \texttt{Sweave} or \texttt{RMarkdown} as the R documentation language,
\item \texttt{ESS} is an Emacs package for running R,
\end{itemize}
    

\section{Copyright}

The files within this project are currently under full copyright.  This
may change at a later date but for now I want to protect whatever
intellectual property there is.  Therefore all files in the project
should contain the copyright notice below with the appropriate comment
character for whatever langangue parser is used for that file; the
example below uses the comment chacter for \LaTeXe.  The
\texttt{<filename>} should contain the path name relative to the top
project directory. The \texttt{<file description>} should contain a
quick description of the file contents. The copyright \texttt{<Year>}
should be the year in which the file was created.

\begin{verbatim}
%%
%%
%%  <filename>
%%
%%  
%%  [file description]
%%
%%  Part of Books20 Project
%%
%%  Copyright <Year> James R. Fowler
%%
%%  All rights reserved. No part of this publication may be
%%  reproduced, stored in a retrival system, or transmitted
%%  in any form or by any means, electronic, mechanical,
%%  photocopying, recording, or otherwise, without prior written
%%  permission of the author.
%%
%%
%%
\end{verbatim}


\chapter{English Grammer and Syntax}
\import{./}{style_english}

\chapter{\LaTeX2e\ Documents}
\import{./}{style_latex}
\import{./}{style_bibliography}

\chapter{Python Programming}
\import{./}{style_python}

\chapter{Sphinx Documentation}
%%\import{./}{style_sphinx}

\chapter{Building Tools}
\import{./}{style_build}

%%
%% \appendix
%%   would go before \backmatter
%%
\printbibliography


\backmatter

\end{document}


