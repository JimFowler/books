%% Begin copyright
%%
%%  /home/jrf/Documents/books/Books20/Docs/Bibliography/bibliography.tex
%%  
%%   Part of the Books20 Project
%%
%%   Copyright 2019 James R. Fowler
%%
%%   All rights reserved. No part of this publication may be
%%   reproduced, stored in a retrival system, or transmitted
%%   in any form or by any means, electronic, mechanical,
%%   photocopying, recording, or otherwise, without prior written
%%   permission of the author.
%%
%%
%% End copyright

%%
%%   Tests and notes about biblatex and bibliographys
%%   with the book project "Some Important Books in Astronomy
%%    and Astrophysics in the 20th Century"
%%

\section{Bibliography and citations}

Use the \texttt{biblatex} package \cite{Lehman2018} and associated
documentation \cite{Kime2019} rather than the \texttt{bibtex}
package. The \texttt{biblatex} package allows both global and local
bibliography files and is provided with the \texttt{books20} package.
For example, we can cite this book by Basbanes \cite{Basbanes2003}
from the global file and we can cite this web page with the
documentation for the \texttt{biblatex} package \cite{Lehman2018} from
the local file. The file \texttt{./Docs/MasterBib.bib} is the global
bibliography file and must be included in the document using the
\verb|\addglobalbib{../MasterBib.bib}| command. You may use additional
local biblography files as well with this package. Use the command
\verb|\addbibresource{localbib.bib}|. These commands should be in the
preface before the \verb|\begin{document}| command. The choice of
which references should be global and which should be local will
probably change with time.
  
Need to work out how to handle link resources and references in
footnotes, particular URLs in footnotes. Should all URLs be at the
back of the document in the bibliography?  This is possible if we use
local .bib files

\subsection{Citation Styles}

I like the citation style \texttt{alphabetic} which is used in the
\texttt{books20} package. The command \verb|\cite{Fowler1956}| will
produce a citation as \cite{Fowler1956}. Note that square brackets are
placed around the citation in this style so we do not need to use
parenthesis. It is also possible to write our own \textbf{Books20}
citation style (\cite{Kime2019}, section 4) but if this work is
commercially published the publisher may have their own style; that
is, if they use \LaTeXe.  I have noted that many history books use a
footnote style in the back of the volume to include both references
and notes. There are no default history bibliography styles in the
generic \texttt{biblatex} package. Citation and bibliography styles
are defined in the \texttt{biblatex} manual (\cite{Kime2019},
p. 72).  Entry types are described in the \texttt{biblatex} manual
(\cite{Kime2019}, p. 8).

\subsection{Multi-volume books}

For a multi-volume book, where multiple books are published, perhaps
in different years, but they make a single complete work, use the
\texttt{@MvBook} type for the common information with a
\texttt{related} field entry to the individual volumes, with the
individual volumes entered as \texttt{@Book} types with the
particular information and a \texttt{crossref} field to the
multi-volume entry.

\begin{verbatim}
@Book{Tebbel1972,
  title    = {The Creating of an Industry 1630-1865},
  date     = {1972},
  volume   = {1},
  crossref = {Tebbel},
}

@Book{Tebbel1975,
  title    = {The Expansion of an Industry 1865-1919},
  date     = {1975},
  volume   = {2},
  crossref = {Tebbel},
}

@Book{Tebbel1978,
  title    = {The Golden Age between Two Wars 1920-1940},
  date     = {1978},
  volume   = {3},
  crossref = {Tebbel},
}

@Book{Tebbel1981,
  title    = {The Great Change 1940-1980},
  date     = {1981},
  volume   = {4},
  crossref = {Tebbel},
}

@MvBook{Tebbel,
  author        = {Tebbel, John William},
  title         = {A History of Book Publishing in the United States},
  volumes       = {4},
  publisher     = {R. R. Bowker Co},
  address       = {New York},
  groups        = {Books Noted},
  lccn          = {Z473 T42},
  owner         = {jrf},
  related       = {Tebbel1972, Tebbel1975, Tebbel1978, Tebbel1981},
  relatedtype   = {multivolume},
  timestamp     = {2018-09-03},
}
\end{verbatim}

In the default \texttt{biblatex} citation style two different
references to individual volumes of a multi-volume work will also add
the multi-volume item to the bibliography. So if we reference only two
volumes, say Tebbel, volume 1, \textit{The Creation of an Industry
  1630-1865} (\cite{Tebbel1972}) as well as Tebbel, volume
4. \textit{The Great Change 1940-1980} (\cite{Tebbel1981}), then the
bibliography will also show the the main entry in Tebbel as well. The
number of volumes needed to show the main entry is set by the package
option \texttt{mincrossrefs}; the default value is two
(\cite{Kime2019}, page 26).  One can also reference multiple works
with one citation command such as \verb|\cite{Tebbel1972, Tebbel1981}|
to produce \cite{Tebbel1972, Tebbel1981}.

\subsection{Book Series}

For books in a series simply use the \texttt{series} field to give
the name the series and the \texttt{number} field if the series is
numbered. For example, we can reference \textit{Book 9}
(\cite{book09}) and \textit{Book 10} (\cite{book10}) from the series
\textit{An Amazing Series of Books}.

\begin{verbatim}
@Book{series09,
  title        = {Book 9},
  ...
  series       = {An Amazing Series of Books},
  number       = {9},
  ...
}

@Book{series10,
  title        = {Book 10},
  ...
  series       = {An Amazing Series of Books},
  number       = {10},
  ...
}
\end{verbatim}


\subsection{Books with chapters by different authors}

Many works are written as monographs but they consist of chapters
written different authors. Such works might be reference as the entire
work or as single chapters by the individual authors. Citation for the
overall works are referenced as \texttt{@Collection} or
\texttt{@MvCollection}, similar to the multi-volume books.  The
individual chapter are referenced as \texttt{@InCollection}. As an
example, here are the citations for the book {\bfseries Europa} and a
chapter from that volume.  Note that the main volume is also part of
the series {\bfseries Space Science series}. Note also that the
\texttt{@InCollection} item should appear in the bib file before
the \texttt{@Collection} entry. As is the case with multi-volume
books if you reference two or more chapters, the main work will be
included in the bibliograpy as weil. Citing both chapters
\cite{Alexander2009, Canup2009} will show the main work {\bfseries
  Europa}. This behavior is again controlled by the value of the
package option \texttt{crossref}.


\begin{verbatim}
@InCollection{Alexander2009,
  author    = {Alexander, Claudia and Carlson, Robert},
  title     = {The Exploration History of Europa},
  booktitle = {Europa},
  date      = {2009},
  pages     = {3-26},
  crossref  = {Pappalardo2009},
  ...
}

@InCollection{Canup2009,
  author    = {Canup, R. M. and Ward, R. M},
  title     = {Origin of Europa and the Galilean Satellites},
  booktitle = {Europa},
  date      = {2009},
  pages     = {59-84},
  crossref  = {Pappalardo2009},
  ...
}

@Collection{Pappalardo2009,
  editor    = {Pappalardo, Robert T. and McKinnon, William B.},
  title     = {Europa},
  date      = {2009},
  pagetotal = {727},
  series    = {Space Science Series},
  publisher = {University of Arizona Press},
  ...
}
\end{verbatim}

\subsection{Conference Proceedings}

Confernce proceedings are much like a multi-author books or
collections.  There is the main proceedings volume or multi-volumes
which are referenced as \texttt{@MvProceedings} or
\texttt{@Proceedings} and the individual articles which may be
reference by \texttt{@InProceedings}.

\subsection{Online resources}

Online resources are meant for any resource that are intrinsically
online only.  A required field in \texttt{@Online} types is one of
either the \texttt{URL}, \texttt{DOI}, or \texttt{eprint}
keyword. As an example, the manual for the \texttt{biblatex} system
is an online resource \cite{Kime2019}. Note the \texttt{url} field
as well as the access date in the \texttt{urldate}. This date
string is normally given is ISO-8601 format as specified in the
{\bfseries Books20 Style Guide}. Note that the description of ISO-8601
is itself an online resource \cite{isotime}.

\begin{verbatim}
@Online{Kime2019,
  author   = {Kime, Philip and Wemheuer, Moritz
              and Lehman, Philipp},
  title    = {The {\tt biblatex} Package},
  date     = {2019-08-17},
  url      = {http://mirrors.ibiblio.org/CTAN/macros/
              latex/contrib/biblatex/doc/biblatex.pdf},
  subtitle = {Programmable Bibliographies and Citations},
  version  = {3.13},
  urldate  = {2019-10-20},
}

@Online{isotime,
  title    = {ISO 8601 Time},
  url      = {https://en.wikipedia.org/wiki/ISO_8601},
  urldate  = {2017-12-15},
}
\end{verbatim}

The author/editor is normally a required field for online resources
but in many cases it may not be possible to identify an
author/editor. This doesn't matter if the bibliography style is
\texttt{numeric} but the reference will show up with a strange
format, if at all, in the style \texttt{alphabetic}.

\subsection{Journal articles}

Journal articles are referenced with the\texttt{@Article} entry type.
For example, Yan 2018 is \cite{Yan2018}, The only required fields are
the \texttt{author}, \texttt{title}, \texttt{date}, and
\texttt{journaltitle}.  However, the \texttt{volume},
\texttt{pages}, and perhaps \texttt{issue} should also be included or
we will not be able to find the article in the library.

\begin{verbatim}
@Article{Yan2018,
  author       = { Yan, Hong-Liang and Shi, Jian-Rong},
  title        = {The nature of the lithium enrichment in
                  the most Li-rich giant star}
  journaltitle = {Nature Astronomy},
  date         = {2018},
  doi          = {https://doi-org.ezproxy.lib.utexas.edu/
                  10.1038/s41550-018-0544-7}
  volume       = {2},
  issue        = {10},
  pages        = {790-795},
}

\end{verbatim}

\subsection{Everything else}
This is a catch-all category for anything not covered by the standard
entry types.  The fields are \texttt{author}, \texttt{title}, and
\texttt{date} are normally required but the \texttt{howpublished}
field should be include as well to indicate how we might find the
reference. For example, we reference the following as
\cite{Smith1982}.

\begin{verbatim}
@Misc{Smith1982,
  author  = {Smith, William J.},
  title   = {An Important Work},
  date    = {1982},
  howpublished = {privately published},
}
\end{verbatim}
