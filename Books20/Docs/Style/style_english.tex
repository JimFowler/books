%% Begin copyright
%%
%%  /home/jrf/Documents/books/Books20/Docs/Style/style_english.tex
%%  
%%   Part of the Books20 Project
%%
%%   Copyright 2018 James R. Fowler
%%
%%   All rights reserved. No part of this publication may be
%%   reproduced, stored in a retrival system, or transmitted
%%   in any form or by any means, electronic, mechanical,
%%   photocopying, recording, or otherwise, without prior written
%%   permission of the author.
%%
%%
%% End copyright

%%
%%   The style guide for English grammer and writing
%%
\begin{itemize}

\item Dashes and Hyphens --- The dash `-' is used for compound-words and for
  hyphenation at the end of a line.  The en-dash `--' is used for a range
  of numbers, 1 -- 10. The em-dash `---' is used for punctuation ---
  e.g.\ separating phrases.

\item Date format --- Use the ISO-8601 \cite{isotime} style
  YYYY-MM-DD for dates and hh:mm:ss for time. Use YYYY or YYYY-MM
  and hh:mm for abbreviated year-month and hours-minutes.

\item Tables should be in longtable style and should have a caption
  with two hlines after the heading and at the end. The
  \texttt{longtable} \cite{Carlisle2014} package is loaded in the \texttt{books20}
  package.

\item You may also use the functions in table.py with a python script to create
  a table in \LaTeX\ longtable format.  I'll write some documentation someday
  but for now you can look at ./Doc/Series/assl\_table.py.

\item Use the serial comma --- for example, we write this, that, and
  use a serial comma for the last item. This is also known as the Oxford
  comma.

\item Use footnotes for comments; bibliography for references to
  external literature; internal references to refer to other parts of
  the document.

\item Books title shall be in \bt{italics}. Use the command
  \verb|\bt{My Book Title}| to produce \bt{My Book Title} in a
  sentence. Note that \verb|\bt{}| can be changed in \texttt{books20}
  packages to produce another style if necessary.

\end{itemize}

\section{Spelling Conventions}

