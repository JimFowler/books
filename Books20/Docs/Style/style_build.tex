%% Begin copyright
%%
%%  /home/jrf/Documents/books/Books20/Docs/Style/style_build.tex
%%  
%%   Part of the Books20 Project
%%
%%   Copyright 2018 James R. Fowler
%%
%%   All rights reserved. No part of this publication may be
%%   reproduced, stored in a retrival system, or transmitted
%%   in any form or by any means, electronic, mechanical,
%%   photocopying, recording, or otherwise, without prior written
%%   permission of the author.
%%
%%
%% End copyright

%%  
%%   My style guide for building products and using autotools.
%%
\section{Autotools}
Use the \texttt{./bootstrap} program to setup \texttt{autotools}.
This will run \texttt{aclocal}, \texttt{autoconf}, and
\texttt{automake}.  If you are unfamilar with the use of the GNU
\texttt{autotools} see the introduction at

\url{https://opensource.com/article/19/7/introduction-gnu-autotools}.

Run the command \texttt{./configure --prefix=~} to configure the
environment and create the Makefiles.

Then run \texttt{make}, \texttt{make install\_documents} and
\texttt{make install\_software} to populate the environment

\section{Makefile.am}

The default \texttt{Makefile.am} can be found in
\texttt{./Tools/build}.  It includes the standard targets.  Most of the
targets do nothing and they may need to be filled out with the
specific files.
