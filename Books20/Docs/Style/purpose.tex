%% Begin copyright
%%
%%  /home/jrf/Documents/books/Books20/Docs/Style/purpose.tex
%%
%%   Part of the Books20 Project
%%
%%   Copyright 2021 James R. Fowler
%%
%%   All rights reserved. No part of this publication may be
%%   reproduced, stored in a retrival system, or transmitted
%%   in any form or by any means, electronic, mechanical,
%%   photocopying, recording, or otherwise, without prior written
%%   permission of the author.
%%
%%
%% End copyright
This style guide is about maintaining consistency within the documents
of my project \ProjectTitle. Consistency is for readablilty and
helps avoid conflicting statements between documents. The tools
that I am using are,

\begin{itemize}
\item \texttt{Ubuntu} Linux at the current LTS release,
\item \texttt{Emacs} as the text editor,
\item \LaTeXe\ as the document processor,
\item with \texttt{BibLatex} bibliography control,
\item \texttt{Python} as the software development language,
\item \texttt{Sphinx} as the Python documentation language,
\item \texttt{Git} as the version control system,
\item with \texttt{GitHub} as the off-site repository,
\item \texttt{sqlite3} is used for the database structure,
\item \texttt{XML} is used for the intermediate data lists until
  I can migrate the data to the database,
\item Gnu \texttt{autotools} as the build environment (autoconf, automake,
  etc.),
\item Gnu \texttt{Make} is used as the build tool,
\item possibly \texttt{R} as the statistical analysis language,
  if I ever have to do detailed statistics, otherwise I'll just
  use the statitics package in python,
\item with \texttt{Sweave} or \texttt{RMarkdown} as the R documentation language,
\item \texttt{ESS} is an Emacs package for running R,
\end{itemize}
    

