%% Begin copyright
%%
%%  /home/jrf/Documents/books/Books20/Docs/Hjs/library.tex
%%
%%   Part of the Books20 Project
%%
%%   Copyright 2022 James R. Fowler
%%
%%   All rights reserved. No part of this publication may be
%%   reproduced, stored in a retrival system, or transmitted
%%   in any form or by any means, electronic, mechanical,
%%   photocopying, recording, or otherwise, without prior written
%%   permission of the author.
%%
%%
%% End copyright


The Harlan J.~Smith Book Collection at the McDonald Observatory
consists of 511 catalogued entries. However, a number of the entries
refer to multi-volume works or contain multiple reprints and articles
so there are quite a few more than that number of items in the
physical collection.

These works represents only a small part of Harlan's reading
interests.  The collection consists of his science library;
the working books he might have used on a daily basis or for
reference.  Harlan's interests, as indicated by these books, are
primarily in planetary studies and space exploration/development. But
there are also works on math, physics, the history of astronomy,
philosophy of science, life on earth, alien intelligence, and the
colonization of space. There is also a strong interest in climate
change dating back to the 1970's.  The collection includes reports
from his work with the various boards and committees on which he
served, while others are gifts from their authors or interesting
works he picked up in his travels.

\begin{wrapfigure}{r}{0.45\textwidth}
  \centering
  \includegraphics[height=0.34\textheight]{hjs_bookplate_small.png}
  {\small Bookplate from the HJS library. Designed by Joan Smith.}
  \label{fig:bookplate}
\end{wrapfigure}

Many of the items contain the bookplate {\bfseries\textsc{`From the
    library of} \textit{Harlan J.~Smith'}}, usually on the inside
front cover. This bookplate was designed by Joan Smith and added to
the books when they were first sent to the McDonald Library. The plate
shows Mt.~Locke as viewed from the north-east with the 107-inch Harlan
J.~Smith telescope in the foreground, the 82-inch Otto Struve
telescope behind it, and the 36-inch telescope in the small dome to
the left.The Latin inscription \textsc{``Sic itur ad astra''}
translates as \textsc{``So we go to the stars''}\footnote{At least
according to \href{https://translate.google.com}{Google Translate}}.

Many of the books also contain the library stamp of Harlan J.~Smith,
usually in the lower-right corner of the title page. These
personalized stamps were quite popular in the 1980s and this stamp was
added by Harlan.

In addition, there is usually a signature or initials from Harlan in
the upper-right corner of the front free end paper, though
occasionally they may appear elsewhere.  These ownership marks are
indicated in the catalogue.  Signatures and initials are given as
written in the item.

A number of books appear to have been purchased from used book
dealers.  There are also a number of duplicate books to be found among
all the entries.

The first books were sent to the McDonald library in 1993--94. At
first, the books were housed in the Superintendent's office, at least
according to Phil Kelton, the Superintendent at the time.  Jane Wiant,
who started as the librarian in ???? recalls that the collection was
housed in the HJS telescope building when she arrived.  At some point
the collection was moved to the Otto Struve library.  When the books
arrived in west Texas an inventory of the collection was created
listing 315 items. I have a paper copy of the Excel spreadsheet of
this inventory.  Comparison with the 307 items identified in the
Struve library for this catalogue located 303 item on the 1994
inventory. There are also 15 items in the current inventory which are
not in the 1994 list. Some of these additional items contain the HJS
bookplate so they belong with the collection; other works are relevant
to the observatory so they certainly might belong in this
collection. The counts don't match because some items in the current
list are grouped as one item while they are listed as separate items
in the 1994 list.  The inventory work for this catalogue on this
group of books was begun on 1 Dec 2021 and finally finished on 23 Oct
2022, after a long delay.

A second set of books were sent by Nat Smith on 2 November 2021.
Cataloguing occurred 10--11 April 2022. There were 4 boxes and 57
items in this second shipment. A third set of books was received from
Jeffrey Mallon on 29 March 2022. These were catalogued on 9--10 April
2022. There were three boxes and 94 items in the third
shipment. Finally, a fourth shipment was received on 13 April 2022 and
catalogued on 16--17 April 2022. There were two boxes (1 cu ft) and
one small box (1/2 cu ft) with a total of 54 items in the fourth
shipment.  Work on the software to produce the catalog was started on
18 April 2022 and mostly finished at the end of June 2022. Further
work on the design and layout was incorporated during the proofreading of
the catalogue in the fall of 2022.
\newpage
\noindent
The catalogue entry format is,
\newline

\vbox{%
  \vspace*{0.1 cm}
  \noindent
  {\footnotesize{index}} \textit{Author/Editors(s)} \textsc{\bfseries Title}

year, place, publisher,\hspace{1em}pagination

edition/printing if known

series name if a part of a published series

publishing comments

comments about the condition of the item

ownership marks of HJS, if any

HJS catalogue number
}
\hbox{}
\vspace{\parskip} The catalogue is sorted by year and then
author/editor. If there is no note about the binding, then you may
assume a hard-bound book. If there is an intact dust jacket, it is
noted. A paperback book is notated as having paper covers. (Entry
\myhref{98} is interesting; it has paper covers but also a paper dust
jacket as well as a one inch paper ribbon.) Any interesting features
of the item are noted as are any articles by HJS. The HJS catalogue
number is ``HJS Shipment\#.Box\#.count'' where shipment 1 is the
existing collection at McDonald while shipments 2, 3, and 4 were the
later boxes shipped in 2021/22. The HJS catalogue number is my
ordering of the books as I catalogued them and form a basic shelf list
that allows me to locate the books again.  This will become
meaningless once the books are actually put on the shelves in the
library. Finally, an index of author/editor(s) and the associated
entry number(s) is provided.

