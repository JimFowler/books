%% Begin copyright
%%
%%  /home/jrf/Documents/books/Books20/Docs/Hjs/bibliography.tex
%%
%%   Part of the Books20 Project
%%
%%   Copyright 2022 James R. Fowler
%%
%%   All rights reserved. No part of this publication may be
%%   reproduced, stored in a retrival system, or transmitted
%%   in any form or by any means, electronic, mechanical,
%%   photocopying, recording, or otherwise, without prior written
%%   permission of the author.
%%
%%
%% End copyright

Harlan James Smith (1924--1991) was born in Wheeling, West Virginia on
25 August 1924, graduating high school there in 1942\footnote{Much of
this information has been taken from the obituary written by James
N.~Douglas in the Bulletin of the AAS (\cite{Douglas1992Harlan}) as
well as from the pamphlet ``The Harlan J.~Smith Centennial
Professorship in Astronomy'' (see entry 7 in the catalogue.) and the
pamphlet ``The Edward Randall Jr., M.D.~Centennial Professorship in
Astronomy'' (see entry 9 in the catalogue).}. Joining the Army Air
Corp in 1943, he finished his service career as a meteorologist on
Guadalcanal in 1946.  Harlan obtained his undergraduate degree in 1949
and his PhD in 1955, both from Harvard. His thesis was on short period
variable stars where he recognized that they were more like Cepheids
than RR Lyrae stars.  These stars he called dwarf Cepheids though
they are now called delta Scuti stars (\cite{HJS1955}, \cite{HJSPhD}),

Harlan joined the faculty at Yale in 1953, obtaining the rank of
Associate Professor. There he began working in radio astronomy where
he and his students studied the decameter radio pulses from
Jupiter. He and Dorrit Hoffleit also studied the optical variablity of
the newly discovered quasars and found unequivocal evidence for
variations in 3C273 implying a small physical size and
(\cite{HJS1963}).

In 1950, while in graduate school at Harvard, Harlan met Joan Greene
who was attending Ratcliffe College. There were married in December
1950 and had four children, Nat, Julie, Tad, and Hannah.

In 1963 Harlan was recruited to be the first ``Texas'' director of the
McDonald Observatory as well as Chairman of the Astronomy Department
at The University of Texas at Austin. During his directorship the
observatory and the department grew in size in both the staff and the
faculty with a strong emphasis on undergraduate teaching in the
department. He resigned as Deparment Chairman in 1978 citing the
heavy work load from the enlarged staff but remained as Director of
McDonald Observatory. He was appointed to the Edward Randell Jr., MD
Centennial Professorship in Astronomy in 1984 (see catalogue entry 9).
Harlan was also committed to public education and outreach.  He
encouraged the McDonald Observatory News (later Star Date magazine)
and the Moody Visitor's Center.

A large observatory needed to have the best telescopes and
instrumentation so the 82-inch telescope was refurbished in the
mid-1960's and talks for a larger telescope were started.  The
107-inch telescope was dedicated in 1969 and was later renamed the
Harlan J.~Smith Telescope in 1995. Lunar laser ranging signals from
the moon were first detected at McDonald in Aug 1969 and the laser
ranging continued at the 107-inch until 1985. This project was
continued on a series of smaller telescopes until 2020(?);
one of the longest experiments at the McDonald Observatory. The
construction of the 107-inch telescope required support for the
visiting astronomers as well as the larger staff so the Transient
Quarters (later the Astronomer's Lodge) was built as well as 15
additional staff houses. A program in radio astronomy was developed
with the construction of the University of Texas Radio Observatory
(UTRAO) in Marfa, Texas. The Millimeter Wave Observatory was later
moved from campus to the observatory. In 1978 he started advocating
for an even larger telescope at McDonald, the 300-inch Eye of Texas
(see entry 396 of the catalogue). This was not to be due to the
collapse of the Texas economy in 1985. He did, however, start
discussions with Penn State University about the 433-inch Hobby-Eberly
Telescope (the former Spectroscopic Survey Telescope, see entry 8 of
the catalogue) which was dedicated in 1997,

Harlan was strongly committed to public service for astronomy and was
a member of the Royal Astronomical Society, the International
Astronomical Union, the American Association for the Advancement of
Science, and the American Astronomical Society (AAS). He was Chairman of the
Planetary Division of the AAS from 1974--1976, AAS Council Member from
1975--1978, and Vice-President of the AAS from 1977-1979. He served on
the national stage as a member of the Committee on the Large Space
Telescope (later the Hubble Space Telescope) from 1966--1970 and a
member of the Space Science Board (SSB) of the National Academy of
Sciences where he was also Chairman of the Committee on Space
Astronomy and Astrophysics of the SSB. Harlan was also a member of the
Board of Directors of the Associated Universities for Research in
Astronomy (AURA), 1972--1983, and was Chairman of the Board from
1980--1983. He was also Chairman of the Managment/Operations Working
Group for Planetary Astronomy (NASA) from 1988--1991.  He received
the NASA Distinguished Public Service Medal in 1991 (see entry 497 of
the catalogue).

Harlan retired as Director of the McDonald Observatory in 1989 but
continued to teach at the university and engage in public service work
until his death on 17 October 1991.


