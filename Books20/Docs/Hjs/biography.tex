%% Begin copyright
%%
%%  /home/jrf/Documents/books/Books20/Docs/Hjs/bibliography.tex
%%
%%   Part of the Books20 Project
%%
%%   Copyright 2022 James R. Fowler
%%
%%   All rights reserved. No part of this publication may be
%%   reproduced, stored in a retrival system, or transmitted
%%   in any form or by any means, electronic, mechanical,
%%   photocopying, recording, or otherwise, without prior written
%%   permission of the author.
%%
%%
%% End copyright


Harlan James Smith (1924--1991) was born in Wheeling, West Virginia on
25 August 1924, graduating high school there in 1942.
\footnote{Much of this information has been
taken from the obituary written by James N.~Douglas in the Bulletin of
the AAS\cite{Douglas1992Harlan} as well as from the pamphlet ``The
Harlan J.~Smith Centennial Professorship in Astronomy'' (see entry
7 in the catalogue.) ``The Edward Randall Jr.,M.D. Centennial Professorship
in Astronomy'' (see entry 9 in the catalogue)} 

Born date location:  25 August 1924 Wheeling West Virginia

early schooling through high school in Wheeling

service career
Joined U.S. Army Air Corps in Feb 1943, assigned as Meteorologist,
send to eDennison University in Ohio for one year of concentracted
math and physics followed by three month course at Harvard in electronics.


academic career, important papers

married Joan Greene 1950, four children Nat, Julie, Tad, Hannah
visited McDonald in 1952 with Joan on belated honeymoon  (Randall)

undergraduate/graduate degree from Harvard 1946 BA 1949 graduate shool
1949, study short period variables, found they were more like Cepheids
than RR Lyrae start, now call Delta Scuti
stars\cite{1955PhDT.........2S} 1953 started teaching at Yale.
Worked on radio emission from Jupiter with Douglas
Lots of planetary work to characterize atmospheres
Discovered optical variability of 3C273\cite{1963Natur.198..650S}
suggesting a small size for the object

McDonald career,

1963--1989 Director of McDonald
rejuvenated 82 inch
1965 105 inch project started
appointed Edward Randell Jr. M.D. Centennial Professor in Astronomy, 1983
important project both him and others
Asked for five things when becoming McDonald directory
(reference Randall)
\begin{itemize}
  \item The faculty should be allowed to grow substantially in both
  depth and breadth, in order to be competative and to take advantage of
  the facilities with McDonald offerred.
    Used NSF and NASA monies to fund 
  \item The physical facilities of the UT Astronomy Department and
  McDonald Observatory both need substantial maintenance and upgrading.
    Started in Painter Hall, moved to PMA in 1973, upgraded the 82 inch
  \item McDonald Observatory needs a substantially larger and more
  modern telescope.
     NASA supplited about 2/3 of the funding
  \item McDonald Observatory and teh UT Astronomy Department would benefit
  from an active radio astronomy group, and from the cross-fertilization
  of optical and radio work.
    Hired James Douglas, who as a student at Yale proposed a radio
    astronomy program to Smith
  \item There should be an eventual support, when it becomes appropriate,
  for expanding into space astronomy.
\end{itemize}

Formed the Advisory Council in 1069
Started work on Eye of Texas in 1978, but the economic collapse
of the Texas oil economy in 1985 halted progress.  Started work
with Penn State on the SST

died data location 17 Oct 1991 cancer

interesting anecdotes
Runnerup in the first Westinghouse National Science Talent Search, met
Harlow Shapley, one of the judges
