%% Begin copyright
%%
%%  /home/jrf/Documents/books/Books20/Docs/Bibliography/bibliography.tex
%%  
%%   Part of the Books20 Project
%%
%%   Copyright 2019 James R. Fowler
%%
%%   All rights reserved. No part of this publication may be
%%   reproduced, stored in a retrival system, or transmitted
%%   in any form or by any means, electronic, mechanical,
%%   photocopying, recording, or otherwise, without prior written
%%   permission of the author.
%%
%%
%% End copyright

%%
%%   Tests and notes about biblatex and bibliographys
%%   with the book project "Some Important Books in Astronomy
%%    and Astrophysics in the 20th Century"

%%

\documentclass{article}

\usepackage[
  backend=biber,
  style=alphabetic-verb,
]{biblatex}
\addbibresource{testbib.bib}

\begin{document}

%%\frontmatter
\title{Test the {\tt biblatex} package}
\author{James R. Fowler}
\date{copy of \today}
\maketitle


\section{Test my Knowledge}

For my {\bf Books20} project I will be using the {\tt biblatex}\cite{Kime2019}
package for the bibliography and citations.
How does this {\tt biblatex} thing work anyway? This
file contains tests and other material to explore that question.


\section{Things to Do}

\begin{enumerate}
  \item Test books published in 1805 with a translation in 1910
    but referenced by original author and date,
    c.f. Laplace 1805. \cite{Laplace1805}

  \item Add information to the style guide because I am sure to forget
    how to work with {\tt jabref} and {\tt biblatex}.

  \item When the database gets built, add a hook to convert database
    entries to {\tt biblatex} format.
\end{enumerate}

\section{How To}

\subsection{Citation Styles}

I like the citation style {\tt alphabetic-verb} which will
list citation as {\tt Fow56} when citing a work like Fowler 1956.

\subsection{Multi-volume books}

For a multi-volume book, where multiple books are published,
perhaps in different years, but they make a single complete work,
use the {\tt @MvBook} type for the common information with a {\tt related}
field entry to the individual volumes, with the individual volumes entered
as {\tt @Book} types with the particular
information and a {\tt crossref} field to the multi-volume entry.

\begin{verbatim}
@Book{Tebbel1972,
  title    = {The Creating of an Industry 1630-1865},
  date     = {1972},
  volume   = {1},
  crossref = {Tebbel},
}

@Book{Tebbel1975,
  title    = {The Expansion of an Industry 1865-1919},
  date     = {1975},
  volume   = {2},
  crossref = {Tebbel},
}

@Book{Tebbel1978,
  title    = {The Golden Age between Two Wars 1920-1940},
  date     = {1978},
  volume   = {3},
  crossref = {Tebbel},
}

@Book{Tebbel1981,
  title    = {The Great Change 1940-1980},
  date     = {1981},
  volume   = {4},
  crossref = {Tebbel},
}

@MvBook{Tebbel,
  author        = {Tebbel, John William},
  title         = {A History of Book Publishing in the United States},
  volumes       = {4},
  publisher     = {R. R. Bowker Co},
  __markedentry = {[jrf:]},
  address       = {New York},
  groups        = {Books Noted},
  lccn          = {Z473 T42},
  owner         = {jrf},
  related       = {Tebbel1972, Tebbel1975, Tebbel1978, Tebbel1981},
  relatedtype   = {multivolume},
  timestamp     = {2018-09-03},
}
\end{verbatim}

In the default {\tt biblatex} citation style two different references
to individual volumes of a multi-volume work will also add the
multi-volume item to the bibliography. So if we reference only two
volumes, say Tebbel, volume 1, {\it The Creation of an Industry
  1630-1865}\cite{Tebbel1972} as well as Tebbel, volume 4. {\it The
  Great Change 1940-1980}\cite{Tebbel1981}, then the bibliography will
also show the the main entry in Tebbel as well. The number of volumes
needed to show the main entry is set by the parameter.

\subsection{Book Series}

For books in a series simply use the {\tt series} field to give the
name the series and the {\tt number} field if the series is
numbered. For example, we can reference {\it Book 9} \cite{book09} and
{\it Book 10} \cite{book10} from the series {\it An Amazing Series of
  Books}.

\begin{verbatim}
@Book{series09,
  title        = {Book 9},
  ...
  series       = {An Amazing Series of Books},
  number       = {9},
  ...
}

@Book{series10,
  title        = {Book 10},
  ...
  series       = {An Amazing Series of Books},
  number       = {10},
  ...
}
\end{verbatim}

\vfil\eject

%%\backmatter
\printbibliography

\end{document}
