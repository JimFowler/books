%% Begin copyright
%%
%%  /home/jrf/Documents/books/Books20/Docs/Bibliography/bibliography.tex
%%  
%%   Part of the Books20 Project
%%
%%   Copyright 2019 James R. Fowler
%%
%%   All rights reserved. No part of this publication may be
%%   reproduced, stored in a retrival system, or transmitted
%%   in any form or by any means, electronic, mechanical,
%%   photocopying, recording, or otherwise, without prior written
%%   permission of the author.
%%
%%
%% End copyright

%%
%%   Tests and notes about biblatex and bibliographys
%%   with the book project "Some Important Books in Astronomy
%%    and Astrophysics in the 20th Century"

%%

\documentclass{article}

\usepackage[
  backend=biber,
  style=alphabetic,
]{biblatex}
\addbibresource{testbib.bib}

\begin{document}

%%\frontmatter
\title{Test the {\ttfamily biblatex} package}
\author{James R. Fowler}
\date{copy of \today}
\maketitle


\section{Test my Knowledge}

For my {\bfseries Books20} project I will be using the {\ttfamily
  biblatex} \cite{Kime2019} package for the bibliography and citations.
How does this {\ttfamily biblatex} thing work anyway? This file
contains tests and other material to explore that question.  The
descriptions of how to use the various feature should be incorporated
into the {\bfseries Books20 Style Guide} when they are complete.


\section{Things to Do}

\begin{enumerate}
  \item Test books published in 1805 with a translation in 1910 but
    referenced by original author and date, c.f. Laplace 1805
    \cite{Laplace1805} if this is possible. I don't think it is!

  \item Add information to the style guide because I am sure to forget
    how to work with {\ttfamily jabref} and {\ttfamily biblatex}.

  \item What should be my required fields for each type of reference?

  \item Add page references to the {\ttfamily biblatex} guide
    \cite{Kime2019} where applicable.
    
  \item When the database gets built, add a hook to convert database
    entries to {\ttfamily biblatex} format.

\end{enumerate}

\section{How To}

\subsection{Citation Styles}

I like the citation style {\ttfamily alphabetic} which is used in this
document. The command \verb|\cite{Fowler1956}| will produce a
citation as \cite{Fowler1956}. Note that square brackets are placed
around the citation in this style so we do not need to use
parenthesis. It is also possible to write a {\bfseries Books20}
citation style (\cite{Kime2019}, section 4) but if this is
commercially published the publisher may have their own style; that
is, if they use \LaTeXe.  I have noted that many history books use a
footnote style in the back of the volume to include references and
notes. There are no history bibliography styles in the generic
{\ttfamily biblatex} package.

\subsection{Multi-volume books}

For a multi-volume book, where multiple books are published, perhaps
in different years, but they make a single complete work, use the
{\ttfamily @MvBook} type for the common information with a {\ttfamily
  related} field entry to the individual volumes, with the individual
volumes entered as {\ttfamily @Book} types with the particular
information and a {\ttfamily crossref} field to the multi-volume
entry.

\begin{verbatim}
@Book{Tebbel1972,
  title    = {The Creating of an Industry 1630-1865},
  date     = {1972},
  volume   = {1},
  crossref = {Tebbel},
}

@Book{Tebbel1975,
  title    = {The Expansion of an Industry 1865-1919},
  date     = {1975},
  volume   = {2},
  crossref = {Tebbel},
}

@Book{Tebbel1978,
  title    = {The Golden Age between Two Wars 1920-1940},
  date     = {1978},
  volume   = {3},
  crossref = {Tebbel},
}

@Book{Tebbel1981,
  title    = {The Great Change 1940-1980},
  date     = {1981},
  volume   = {4},
  crossref = {Tebbel},
}

@MvBook{Tebbel,
  author        = {Tebbel, John William},
  title         = {A History of Book Publishing in the United States},
  volumes       = {4},
  publisher     = {R. R. Bowker Co},
  address       = {New York},
  groups        = {Books Noted},
  lccn          = {Z473 T42},
  owner         = {jrf},
  related       = {Tebbel1972, Tebbel1975, Tebbel1978, Tebbel1981},
  relatedtype   = {multivolume},
  timestamp     = {2018-09-03},
}
\end{verbatim}

In the default {\ttfamily biblatex} citation style two different
references to individual volumes of a multi-volume work will also add
the multi-volume item to the bibliography. So if we reference only two
volumes, say Tebbel, volume 1, {\itshape The Creation of an Industry
  1630-1865} (\cite{Tebbel1972}) as well as Tebbel, volume
4. {\itshape The Great Change 1940-1980} (\cite{Tebbel1981}), then the
bibliography will also show the the main entry in Tebbel as well. The
number of volumes needed to show the main entry is set by the package
option {\ttfamily mincrossrefs}; the default value is two
(\cite{Kime2019}, page 26).  One can also reference multiple works
with one citation command such as \verb|\cite{Tebbel1972, Tebbel1981}|
to produce \cite{Tebbel1972, Tebbel1981}.

\subsection{Book Series}

For books in a series simply use the {\ttfamily series} field to give
the name the series and the {\ttfamily number} field if the series is
numbered. For example, we can reference {\itshape Book 9}
(\cite{book09}) and {\itshape Book 10} (\cite{book10}) from the series
{\itshape An Amazing Series of Books}.

\begin{verbatim}
@Book{series09,
  title        = {Book 9},
  ...
  series       = {An Amazing Series of Books},
  number       = {9},
  ...
}

@Book{series10,
  title        = {Book 10},
  ...
  series       = {An Amazing Series of Books},
  number       = {10},
  ...
}
\end{verbatim}


\subsection{Books with chapter by different authors}
\begin{verbatim}
 @mvcollection
 @collection
 @incollection
 @suppcollection
\end{verbatim}

\subsection{Conference Proceedings}
\begin{verbatim}
 @mvproceedings
 @proceedings
 @inproceedings
\end{verbatim}

\subsection{Online resources}
\begin{verbatim}
 @online
\end{verbatim}

\subsection{Journal articles}
\begin{verbatim}
 @article
\end{verbatim}

\subsection{Software}
\begin{verbatim}
 @software
\end{verbatim}

\subsection{Anything else}
\begin{verbatim}
 @misc
\end{verbatim}

%%\backmatter
\printbibliography

\vfil\eject
\end{document}
