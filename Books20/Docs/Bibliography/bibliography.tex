%% Begin copyright
%%
%%  /home/jrf/Documents/books/Books20/Docs/Bibliography/bibliography.tex
%%  
%%   Part of the Books20 Project
%%
%%   Copyright 2019 James R. Fowler
%%
%%   All rights reserved. No part of this publication may be
%%   reproduced, stored in a retrival system, or transmitted
%%   in any form or by any means, electronic, mechanical,
%%   photocopying, recording, or otherwise, without prior written
%%   permission of the author.
%%
%%
%% End copyright

%%
%%   Tests and notes about biblatex and bibliographys
%%   with the book project "Some Important Books in Astronomy
%%    and Astrophysics in the 20th Century"

%%

\documentclass{article}

\usepackage[
  backend=biber,
  style=alphabetic,
]{biblatex}
\addbibresource{testbib.bib}

\begin{document}

%%\frontmatter
\title{Test the {\ttfamily biblatex} package}
\author{James R. Fowler}
\date{copy of \today}
\maketitle


\section{Test my Knowledge}

For my {\bfseries Books20} project I will be using the {\ttfamily
  biblatex} \cite{Kime2019} package for the bibliography and citations.
How does this {\ttfamily biblatex} thing work anyway? This file
contains tests and other material to explore that question.  The
descriptions of how to use the various feature should be incorporated
into the {\bfseries Books20 Style Guide} when they are complete.


\section{Things to Do}

\begin{enumerate}
  \item Test books published in 1805 with a translation in 1910 but
    referenced by original author and date, c.f. Laplace 1805
    \cite{Laplace1805} if this is possible. I don't think it is!

  \item Add information to the style guide because I am sure to forget
    how to work with {\ttfamily jabref} and {\ttfamily biblatex}.

  \item What should be my recommended fields for each type of reference?
    The required fields are defined by the {\ttfamily biblatex} package.

  \item Add page references to the {\ttfamily biblatex} guide
    \cite{Kime2019} where applicable.
    
  \item When the database gets built, add a hook to convert database
    entries to {\ttfamily biblatex} format.

\end{enumerate}

\section{How To}

\subsection{Citation Styles}

I like the citation style {\ttfamily alphabetic} which is used in this
document. The command \verb|\cite{Fowler1956}| will produce a citation
as \cite{Fowler1956}. Note that square brackets are placed around the
citation in this style so we do not need to use parenthesis. It is
also possible to write our own {\bfseries Books20} citation style
(\cite{Kime2019}, section 4) but if this work is commercially
published the publisher may have their own style; that is, if they use
\LaTeXe.  I have noted that many history books use a footnote style in
the back of the volume to include both references and notes. There are
no default history bibliography styles in the generic {\ttfamily
  biblatex} package. Citation and bibliography styles are defined in
the {\ttfamily biblatex} manual (\cite{Kime2019}, p. 72).  Entry types
are described in the {\ttfamily biblatex} manual (\cite{Kime2019},
p. 8).

\subsection{Multi-volume books}

For a multi-volume book, where multiple books are published, perhaps
in different years, but they make a single complete work, use the
{\ttfamily @MvBook} type for the common information with a {\ttfamily
  related} field entry to the individual volumes, with the individual
volumes entered as {\ttfamily @Book} types with the particular
information and a {\ttfamily crossref} field to the multi-volume
entry.

\begin{verbatim}
@Book{Tebbel1972,
  title    = {The Creating of an Industry 1630-1865},
  date     = {1972},
  volume   = {1},
  crossref = {Tebbel},
}

@Book{Tebbel1975,
  title    = {The Expansion of an Industry 1865-1919},
  date     = {1975},
  volume   = {2},
  crossref = {Tebbel},
}

@Book{Tebbel1978,
  title    = {The Golden Age between Two Wars 1920-1940},
  date     = {1978},
  volume   = {3},
  crossref = {Tebbel},
}

@Book{Tebbel1981,
  title    = {The Great Change 1940-1980},
  date     = {1981},
  volume   = {4},
  crossref = {Tebbel},
}

@MvBook{Tebbel,
  author        = {Tebbel, John William},
  title         = {A History of Book Publishing in the United States},
  volumes       = {4},
  publisher     = {R. R. Bowker Co},
  address       = {New York},
  groups        = {Books Noted},
  lccn          = {Z473 T42},
  owner         = {jrf},
  related       = {Tebbel1972, Tebbel1975, Tebbel1978, Tebbel1981},
  relatedtype   = {multivolume},
  timestamp     = {2018-09-03},
}
\end{verbatim}

In the default {\ttfamily biblatex} citation style two different
references to individual volumes of a multi-volume work will also add
the multi-volume item to the bibliography. So if we reference only two
volumes, say Tebbel, volume 1, {\itshape The Creation of an Industry
  1630-1865} (\cite{Tebbel1972}) as well as Tebbel, volume
4. {\itshape The Great Change 1940-1980} (\cite{Tebbel1981}), then the
bibliography will also show the the main entry in Tebbel as well. The
number of volumes needed to show the main entry is set by the package
option {\ttfamily mincrossrefs}; the default value is two
(\cite{Kime2019}, page 26).  One can also reference multiple works
with one citation command such as \verb|\cite{Tebbel1972, Tebbel1981}|
to produce \cite{Tebbel1972, Tebbel1981}.

\subsection{Book Series}

For books in a series simply use the {\ttfamily series} field to give
the name the series and the {\ttfamily number} field if the series is
numbered. For example, we can reference {\itshape Book 9}
(\cite{book09}) and {\itshape Book 10} (\cite{book10}) from the series
{\itshape An Amazing Series of Books}.

\begin{verbatim}
@Book{series09,
  title        = {Book 9},
  ...
  series       = {An Amazing Series of Books},
  number       = {9},
  ...
}

@Book{series10,
  title        = {Book 10},
  ...
  series       = {An Amazing Series of Books},
  number       = {10},
  ...
}
\end{verbatim}


\subsection{Books with chapters by different authors}

Many works are written as monographs but they consist of chapters
written different authors. Such works might be reference as the entire
work or as single chapters by the individual authors. Citation for the
overall works are referenced as {\ttfamily @Collection} or {\ttfamily
  @MvCollection}, similar to the multi-volume books.  The individual
chapter are referenced as {\ttfamily @InCollection}. As an example,
here are the citations for the book {\bfseries Europa} and a chapter
from that volume.  Note that the main volume is also part of the
series {\bfseries Space Science series}. Note also that the {\ttfamily
  @InCollection} item should appear in the bib file before the
{\ttfamily @Collection} entry. As is the case with multi-volume books
if you reference two or more chapters, the main work will be included
in the bibliograpy as weil. Citing both chapters \cite{Alexander2009,
  Canup2009} will show the main work {\bfseries Europa}. This behavior
is again controlled by the value of the package option {\ttfamily
  crossref}.


\begin{verbatim}
@InCollection{Alexander2009,
  author    = {Alexander, Claudia and Carlson, Robert},
  title     = {The Exploration History of Europa},
  booktitle = {Europa},
  date      = {2009},
  pages     = {3-26},
  crossref  = {Pappalardo2009},
  ...
}

@InCollection{Canup2009,
  author    = {Canup, R. M. and Ward, R. M},
  title     = {Origin of Europa and the Galilean Satellites},
  booktitle = {Europa},
  date      = {2009},
  pages     = {59-84},
  crossref  = {Pappalardo2009},
  ...
}

@Collection{Pappalardo2009,
  editor    = {Pappalardo, Robert T. and McKinnon, William B.},
  title     = {Europa},
  date      = {2009},
  pagetotal = {727},
  series    = {Space Science Series},
  publisher = {University of Arizona Press},
  ...
}
\end{verbatim}

\subsection{Conference Proceedings}

Confernce proceedings are much like a multi-author books or
collections.  There is the main proceedings volume or multi-volumes
which are referenced as {\ttfamily @MvProceedings} or {\ttfamily
  @Proceedings} and the individual articles which may be reference by
{\ttfamily @InProceedings}.

\subsection{Online resources}

Online resources are meant for any resource that are intrinsically
online only.  A required field in {\ttfamily @Online} types is one of
either the {\ttfamily URL}, {\ttfamily DOI}, or {\ttfamily eprint}
keyword. As an example, the manual for the {\ttfamily biblatex} system
is an online resource \cite{Kime2019}. Note the {\ttfamily url} field
as well as the access date in the {\ttfamily urldate}. This date
string is normally given is ISO-8601 format as specified in the
{\bfseries Books20 Style Guide}. Note that the description of ISO-8601
is itself an online resource \cite{isotime}.

\begin{verbatim}
@Online{Kime2019,
  author   = {Kime, Philip and Wemheuer, Moritz
              and Lehman, Philipp},
  title    = {The {\tt biblatex} Package},
  date     = {2019-08-17},
  url      = {http://mirrors.ibiblio.org/CTAN/macros/
              latex/contrib/biblatex/doc/biblatex.pdf},
  subtitle = {Programmable Bibliographies and Citations},
  version  = {3.13},
  urldate  = {2019-10-20},
}

@Online{isotime,
  title    = {ISO 8601 Time},
  url      = {https://en.wikipedia.org/wiki/ISO_8601},
  urldate  = {2017-12-15},
}
\end{verbatim}

The author/editor is normally a required field for online resources
but in many cases it may not be possible to identify an
author/editor. This doesn't matter if the bibliography style is
{\ttfamily numeric} but the reference will show up with a strange
format, if at all, in the style {\ttfamily alphabetic}.

\subsection{Journal articles}

Journal articles are referenced with the{\ttfamily @Article} entry
type.  For example, Yan 2018 is \cite{Yan2018}, The only required
fields are the {\ttfamily author}, {\ttfamily title}, {\ttfamily
  date}, and {\ttfamily journaltitle}.  However, the {\ttfamily
  volume}, {\ttfamily pages}, and perhaps {\ttfamily issue} should
also be included or we will not be able to find the article in the
library.

\begin{verbatim}
@Article{Yan2018,
  author       = { Yan, Hong-Liang and Shi, Jian-Rong},
  title        = {The nature of the lithium enrichment in
                  the most Li-rich giant star}
  journaltitle = {Nature Astronomy},
  date         = {2018},
  doi          = {https://doi-org.ezproxy.lib.utexas.edu/
                  10.1038/s41550-018-0544-7}
  volume       = {2},
  issue        = {10},
  pages        = {790-795},
}

\end{verbatim}

\subsection{Everything else}
This is a catch-all category for anything not covered by the standard
entry types.  The fields are {\ttfamily author}, {\ttfamily title},
and {\ttfamily date} are normally required but the {\ttfamily
  howpublished} field should be include as well to indicate how
we might find the reference. For example, we reference the following
as \cite{Smith1982}.

\begin{verbatim}
@Misc{Smith1982,
  author  = {Smith, William J.},
  title   = {An Important Work},
  date    = {1982},
  howpublished = {privately published},
}
\end{verbatim}

%%\backmatter
\printbibliography

\vfil\eject
\end{document}
