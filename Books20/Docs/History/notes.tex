\bt{Astrophysics and Cosmology}

Malcom S.\ Longair

In Chapter 23 of \bt{Twentieth Century Physics} (\cite{Physics1995},
page 1691), Malcom Longair (\cite{Longair1995}) briefly discusses the
history of astrophysics and cosmology in \Cen{20}.  This 132 page
chapter was later expanded into his 2006 book \bt{The Cosmic Century};
a much longer work on \Cen{20} astrophysics.

Longair divides this work and the century into four parts

\begin{enumerate}
\item Stars and Stellar Evolution up to World War II
  \begin{enumerate}
  \item The legacy of the nineteenth century
  \item The origin of the Hertsprung-Russell diagram
  \item Stellar structure and evolution
  \end{enumerate}

\item The Large-Scale Structure of the Universe 1900-1939
  \begin{enumerate}
  \item The structure of our galaxy
  \item The great debate
  \item The development of relativistic cosmology
  \end{enumerate}

\item The Opening Up of the Electromagnetic Spectrum
  \begin{enumerate}
  \item The changing astronomical perspective
  \end{enumerate}

\item Astrophysics and Cosmology since 1945
  \begin{enumerate}
  \item Stars and stellar evolution since 1945
  \item The physics of the interstellar medium
  \item The physics of galaxies and cluster of galaxies
  \item High-energy astrophysics
  \item Astrophysics Cosmology
  \item The classical cosmological problem
  \item Galaxy formation
  \item The very early universe
  \end{enumerate}
\end{enumerate}

As other authors have also noted, he make a clear distinction between
pre- and post-World War II science.


\section{Stars and Stellar Evolution up to World War II}

\subsection{The legacy of the nineteenth century}

Through the end of the nineteenth century\marginpar{p. 1691} astronomy
meant positional astronomy, the observation and recording of the
positions of the stars in the sky.  Longair identifies three major
development which created astrophysics; first, the measurement of
parallax; second, the development of spectroscopy; and thirdly, the
development of photography.

The first parallax measurement was made in 1838 by Fredrich
Bessel (\cite{Bessel1839}) of the star 61 Cygni. However by 1900 less
that 100 parallax measurement had been made with any accuracy.

Wollaston (\cite{Wollaston1802}) recorded the first solar spectrum who
observed absorption line but did not note the significance.
Fraunhofer (\cite{Fraunhofer1817}), in 1814, recorded ten strong lines
and 574 fainter ones. He also observed spectra of the planets and
stars (\cite{Fraunhofer1823}). But is was not until 1863 that Bunsen
and Kirchoff (\cite{Kirchoff1861} identified these lines with chemical
elements.

Photography was developed by Daguerre in France and Fox Talbot in
England in 1839. But is was not until the creation of the dry
collodion plate in 1870 and the subsequent invention of gelatin
emulsion that photography because useful for long exposures of stellar
spectra.

By the 1880s\marginpar{p. 1692} the tools were in place to being a
detailed study of the spectra of stars and thus astrophysics was born.


\subsection{The origin of the Hertsprung-Russell diagram}


\subsection{Stellar structure and evolution}

