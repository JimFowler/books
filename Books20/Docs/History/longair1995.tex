\bt{Astrophysics and Cosmology}

Malcom S.\ Longair, 1995

In Chapter 23 of \bt{Twentieth Century Physics} (\cite{Physics1995},
page 1691), Malcom Longair \cite{Longair1995} briefly discusses the
history of astrophysics and cosmology in the \Cen{20}.  This 132 page
chapter was later expanded into his 2006 book \bt{The Cosmic Century};
a much longer work on \Cen{20} astrophysics.

Longair divides this work and the century into four parts

\begin{enumerate}
\item Stars and Stellar Evolution up to World War II
  \begin{enumerate}
  \item The legacy of the nineteenth century
  \item The origin of the Hertzsprung-Russell diagram
  \item Stellar structure and evolution
  \end{enumerate}

\item The Large-Scale Structure of the Universe 1900-1939
  \begin{enumerate}
  \item The structure of our galaxy
  \item The great debate
  \item The development of relativistic cosmology
  \end{enumerate}

\item The Opening Up of the Electromagnetic Spectrum
  \begin{enumerate}
  \item The changing astronomical perspective
  \end{enumerate}

\item Astrophysics and Cosmology since 1945
  \begin{enumerate}
  \item Stars and stellar evolution since 1945
  \item The physics of the interstellar medium
  \item The physics of galaxies and cluster of galaxies
  \item High-energy astrophysics
  \item Astrophysics Cosmology
  \item The classical cosmological problem
  \item Galaxy formation
  \item The very early universe
  \end{enumerate}
\end{enumerate}

As other authors have also noted, he makes a clear distinction between
pre- and post-World War II science.


\section{Stars and Stellar Evolution\\ up to World War II}

\subsection{The legacy of the nineteenth century}

Through the end of the nineteenth century\marginpar{p. 1691} astronomy
meant positional astronomy, the observation and recording of the
positions of the stars in the sky.  Longair identifies three major
development which created astrophysics; first, the measurement of
parallax; second, the development of spectroscopy; and thirdly, the
development of photography.

The first parallax measurement\marginpar{\textbf{Parallax}} was made
in 1838 by Fredrich Bessel \cite{Bessel1839} of the star 61
Cygni. However by 1900 less that 100 parallax measurement had been
made with any accuracy.

Wollaston \cite{Wollaston1802}
recorded \marginpar{\textbf{Spectroscopy}} the first solar spectrum
who observed absorption line but did not note their significance.
Fraunhofer \cite{Fraunhofer1817}, in 1814, recorded ten strong lines
and 574 fainter ones. He also observed spectra of the planets and
stars \cite{Fraunhofer1823}. But is was not until 1863 that Bunsen
and Kirchoff \cite{Kirchoff1861} identified these lines with
chemical elements.

Photography\marginpar{\textbf{Photography}} was developed by Daguerre
in France and Fox Talbot in England, in 1839. But is was not until the
creation of the dry collodion plate in 1870 and the subsequent
invention of gelatin emulsion that photography because useful for long
exposures of stellar spectra.

By the 1880s\marginpar{p. 1692} the tools were in place to being a
detailed study of the spectra of stars and thus astrophysics was born.


\subsection{The origin of the Hertzsprung-Russell diagram}

\subsubsection{The classification of stellar spectra}

Once stellar spectra were obtainable \marginpar{p. 1692} they were
noted to have a diversity of feature and the second half of the
nineteenth century saw much effort to classify and understand these
features.  The Jesuit priest Father Angelo Secchi was one of the
first, classifying over 4000 stars in the 1860s.  The final version of
his classification scheme \marginpar{p. 1693} divided stars into four
classes, white or blue stars, class I, yellow or solar type stars,
class II, red stars with wide absorption bands, class III
\cite{Secchi1863}, and red stars with 'luminous bands separated by
dark intervals', class IV, now known as carbon stars
\cite{Secchi1868}. Photography enhanced the details and
subcategories were added to Secchi's scheme.  Stellar classification
became a major industry and by 1900 there were 23 different systems in
use (c.f.\ DeVorkin, PhD.~Thesis [get reference]).

It was the Harvard Observatory, under the directorship of Edward C.\
Pickering which brought the modern nomenclature to stellar spectra.
The first publication \cite{Pickering1890} was based on Fleming's
work classifying 10,351 stars on 633 objective prism plates of the
southern(?) sky. The scheme was based on Secchi's but subdivided his
class further. Class I had four sub-classes, A, B, C, and D, Class II
was divided into seven sub-classes E, F, G, H, I, K, and L, while
Classs III and IV were renamed M and N.  The Wolf-Rayet stars were
given the designation O.

Cannon \marginpar{p. 1683} revised this scheme in 1901
\cite{Cannon1901} [Verify this statement!] so that the types were
based on the presence or absence and strength of different spectral
lines and she removed many of the sub-classes.  The final sequence is
the one we know today as O, B, A, F, G, K, M. This was known the be a
temperature sequence with the hottest stars first and cooler ones
later in the sequence based on the investigations of Norman Lockyer
but it was not until the development of quantum mechanics and the work
of Saha, Fowler, and Milne in the early 1920s that the relationship
between temperature and spectral lines was made clear. The
publication \marginpar{p. 1694} of the final Henry Draper catalogue
occurred in a series of publications between 1918 and 1924
\cite{Cannon1918a, Cannon1918b, Cannon1919a, Cannon1919b, Cannon1920,
Cannon1921, Cannon1922, Cannon1923, Cannon1924} with 225,300 stars
classified by A.\ J.\ Cannon.

\subsubsection{Early theories of stellar structure and evolution}

With a better understanding\marginpar{p. 1694} of thermodynamics in the
1850s Lord Kelvin and Helmholtz (1854) calculated the life time of the
sun at \Ord{10}{7} years, which was much shorter than the best
estimate of the age of the earth calculated from stratigraphic
analysis. J.~Homer Lane was the first to investigate the internal
structure of the sun as a gaseous body \cite{Lane1870}. Ritter, in
1883, independently did calculations as well, \cite{Ritter1883,
Ritter1883a, Ritter1898}. These\marginpar{p. 1695} early physical
models were summarized by Robert Emden in his book \bt{Gaskugeln}
published in 1907 \cite{Emden1907}.

\subsubsection{The Hertzsprung-Russell Diagram}

In 1914 \marginpar{p. 1695} Henry Norris Russell\cite{Russell1914a,
Russell1914b} published what later became known at the
Hertzsprung-Russell diagram by plotting absolute luminosity against
spectral type (from the Draper Catalogue). The absolute luminosities
were acquire through parallax studies. E.~Hertzsprung
\marginpar{p. 1696} had published similar diagrams for the Pleiades
and Hyades cluster and named the sequence of dwarf stars
the \textit{main sequence} \cite{Hertzsprung1911}.  He was able to do
this because the distinction between dwarfs and giants was known
observationally though parallax and spectral studies though the
theoretical reasons were still unknown. Adams and \marginpar{p. 1698}
Kohlschütter \cite{Adams1914} independently determined spectral
luminosity indicators while Shapley determined stellar masses of giant
stars from the analysis of eclipsing binary orbits. He found that the
mass was weakly correlated with luminosity.  This was consistent with
the prevailing view that stars started as contracting red-giants that
moved down the main sequence as they contracted.


\subsection{Stellar structure and evolution}

\subsubsection{The impact of the new physics}

This \marginpar{p. 1699} view would change within ten
years. Hertzsprung \cite{Hertzsprung1919} had derived an empirical
mass-luminosity relationship for main-sequence stars. Although energy
transport with in stars by radiation rather than convective transport
had been discussed as early as 1894 it wasn't until 1916 that
Eddington \cite{Eddington1916} revived the notion in his article on
radiative equilibrium. But it was the notion of quantum theory that
really jump started the field. Bohr's theory of atomic structure had
an immediate impact \cite{Bohr1913}. That stellar lines corresponded
to energy state transitions meant that the temperatures of stellar
atmospheres could be determined. Saha, in 1919 \cite{Saha1919},
determined how the state of ionization depended on density and
temperature. He employed the Harvard criteria of first appearance and
disappearance of different spectral lines to conclude that the main
sequence is actually a temperature sequence. This idea was later
refined by Cecilia Payne \marginpar{p. 1701} in her 1925 Harvard
thesis and book, \bt{Stellar Atmospheres} \cite{Payne1925}. She noted
that stars have roughly the same chemical composition and that the
main difference is temperature.

\subsubsection{Eddington and the theory of stellar structure}

Eddington was the central figure in the development of the theory of
stellar evolution and the internal structure of stars. His
book, \bt{The Internal Constitution of the Stars}, published in
1926, was the summation of his work. According to Chandrasekhar
he was responsible for our understanding that,
\begin{itemize}
\item radiation pressure must play an important role,
\item in radiative equilibrium, the temperatures gradient is
      determined jointly by the opacity of the matter and the
      distribution of the energy sources,
\item the principal process contributing to the opacity is
      determined by the photo-electric absorption of the inner-most K-
      and L-shells for the highly ionized atoms,
\item with electron scattering as the ultimate source of opacity
      there is an upper limit to the luminosity that can support a
      given mass.
\item to a first approximation, in main sequence stars the mass, luminosity,
      and effective temperature relations are not very sensitive to
      the distribution of energy sources,
\item the burning of hydrogen into helium is the most likely source of
      energy.
\end{itemize}
      
\subsubsection{The impact of quantum mechanics\\ and the discovery of new particles}

The energy problem for stars was solved during the period
1920--1939. By 1931 \marginpar{p. 1707} hydrogen was recognized as the
most abundant element and the by 1938 Bethe and
Critchfield \cite{Bethe1938} had calculated the energy production of
the P-P chain and in 1939 Bethe had calculated the energy production
of the CNO cycle.

\subsubsection{The red-giant problem}

The explanation \marginpar{p. 1708} to the enormous size of the
red-giants was solved by Opik in 1938 \cite{Opik1938} when he realized
that nuclear burning takes place only in the central 10\% by mass of
the star. His major discovery was that as the core collapses the outer
envelope expands to great size.  This was later discussed by
Salpeter \cite{Salpeter1952}.

\subsubsection{White dwarfs}

Found be Russell \marginpar{p. 1710} and included, but unremarked, in
his first Russell diagram \cite{Russell1914a, Russell1914b}.  Remarked
on by Adams \cite{Adams1914, Adams1915} who found that the companion
to Sirius was also a white dwarf.  Both were member of a binary system
so their masses could be determined as $10^8 kg m^{-3}$.
Eddington \cite{Eddington1924} argued that such masses were not
inconceivable. He calculated the gravitational redshift expected which
Adams \cite{Adams1925} was able to verify.


\subsubsection{Supernovae and neutron stars}

