%%
%%
%% book_notes.tex
%%
%%
%% notes and references in the books read for this project. This 
%% document is included in Notes.tex
%%
%%
%%   Copyright 2009 James R. Fowler
%%
%%   All rights reserved. No part of this publication may be
%%   reproduced, stored in a retrival system, or transmitted
%%   in any form or by any means, electronic, mechanical,
%%   photocopying, recording, or otherwise, without prior written
%%   permission of the author.
%% 
%% The last known changes were checked in by $Author$
%% as revision $LastChangedRevision$
%% on $Date$
%%
%%
{\it A Splendor of Letters: The Permanence of Books in an Impermanent
  World}, Nicolas A.\ Basbanes, chapter 7. \cite{basbanes:2003}

Books are not only important for their contents they are also
important for the contextual information they provide about the
period, about the publishing, and about the author.  The materials
that make up the physical artifact, the cover, bookjacket,
etc.\ provide some commentary about the work and its place in the world.

Many books in the \Cen{20} were printed on acidic paper making them 
unlikely to survive for another century.


\vskip\baselineskip
{\it Under Newton's Shadow: Astronomical Practices in the Seventeenth Century},
 Lesley Murdin, p 39, \cite{murdin:1985}

 The first 10 sections of Newton's Principia, 1686, were given as
 lectures at Cambridge in 1684.


\vskip\baselineskip
{\it The Scientific Literature}, Joseph E.\ Harmon and Alan G.\ Gross,
\cite{harmon:2007}

A good short review of how science journal articles developed and changed
over time from the mid-\Cen{17} to the begining of the \Cen{21}.
 
\begin{itemize}
\item p XVII, Jan 1665, Journaldes S\c{c}avans (Journal of the Learned)
first published in Paris by Denys de Sallo.
\item Mar 1665 Philosophical Transactions first published in London for
the Royal Society by Henry Olenburg.
\item p XVIII, prior to the journals publication was by letters (limited
circulation) or by books (long gestation period)
\item p XIX, goal to show variety of written and visual expression over time.
\begin{enumerate}
  \item focus on equations, tables, and visuals as well as words  
  \item include commentary about selections
  \item include articles from all over the world
  \item tell story about the origin of the scientific journal
\end{enumerate}
\item p XX limitations, only brief article selection, limited to western science,
too few women.
\item p XXII, writing style of article depends on the goal of the author,
theoretical, experimental, observational, or review.
\item p 2, books were important for communicating the scientific revolution
1600-1700, new journals were seen as ancillary, more formal letters.
\item p 4, Philosophical Transactions avoids ``fine speaking'' in preference
for the facts.
\item p 5 no formal style but modern parts, introduction, methods and materials,
results, and discussion are there.
\item p 7, in the history of science new theories grab the headlines, but progress
is made mostly by creation and coninual refinement of new research tools, c.f.\ Kuhn, Scientific Revolution.
\item p 13 multiple voices in early journals, the editor commenting on a recieved
communication, the author relating observations, only latter do we find
the impersonal voice of nature speeking through the scientist.
\item p 37 1752 Philosophical Transactions institutes an editorial committee
to review articles, 1831 it institutes outside peer review.
\item p 76, increased specialization in journals, in 1863 there were 1500 journals
while in 1763 there were only 100.
\item p 301, 1858 Darwin publishes a brief article in the Journal of the
Proceedings ore the Linnaen Society in order to establish his independent 
claim with Wallace. The article has no impact but his book in 1859 is
instantly controversial.
\item p 188, modern organization of articles reached in 1830-1840 in
German chemical literature.
\item p 190, review articles describe and evaluate the recent literature
in the field.  They also serve another purpose, that of culling unfit
knowledge from the succesful ones. Review article generally have high
citation rates.
\item p 192, abstracts first appears in the 1920s but did not become
routine practice until the 1950s.
\item p 193, the introduction defines the historical context of the probem,
states the specific problem the paper looks at, and finally summarizes to
solution found.
\end{itemize}


\vskip\baselineskip
{\it Scientific Books, Libraries and Collectors} John L.\ Thornton
and R.\ I.\ J.\ Tully \cite{thornton:1971}

\begin{itemize}
\item page 200 ``Classics of science may not be recognized until a
considerable period after publication, and best-sellers of today may
be entirely neglected a few years after their appearance. It is
therefore dificult to assess the value of comparatively modern
literature, and in this chapter we mention merely the high lights in
each major branch of science.

In some subjects in is impossible to indicate certain books as
outstanding contributions to scientific thought, either because the
more eminent exponents did not write books, or their discoveries were
announced in the periodical press. Astronomy, for example, made
extraordinary progress, largely as the result of great improvements in
the size of the telescopes employed.  Yet few eminent books devoted to
the subject were published in the nineteenth century. This despite the fact
that several prominent astronomers, both amateur and professional, were
recording important discoveries in the heavens.''

\item p 294 ``Every investigation must begin with a bibliography,
and end with a better bibliography.'' George Sarton, {\it A Guide to
the History of Science}, Chronica Botanica, 1951, Waltham MA

http://www.archive.org/stream/guidetohistoryof00sart/guidetohistoryof00sart\_djvu.txt

Q124 S24, 509 S251G
\end{itemize}
