%%
%%
%% BooksRead.bib
%%
%%   Quotes noted during readings for
%%   the book project "Some Important Books in Astronomy
%%   and Astrophysics in the 20th Century"
%%
%%   Copyright 2009 James R. Fowler
%%
%%   All rights reserved. No part of the publication may be
%%   reproduced, stored in a retrival system, or transmitted
%%   in any form or by any means, electronic, mechanical,
%%   photocopying, recording, or otherwise, without prior written
%%   permission of the author.
%% 
%% The last known changes were checked in by $Author$
%% as revision $LastChangedRevision$
%% on $Date$
%%
%%
\documentstyle{book}
\usepackage{books20}

\begin{document}

\mainmatter

\begin{quote}

\ldots [this story] is especially pertinent for a writer who is
obsessed with books in every imaginable sense and nuance of the
word. I am fascinated by their history and composition, by the many
shapes and forms they have assumed over time. I want to know
everything I can about the people who write them, make them, preserve
them, sell them, cover them, collect them, fear them, ban them,
destroy them, and, most of all, about those who are moved,
entertained, instructed, awed, and inspired by them\ldots

         Nicholas A. Basbanes, p xvii, {\it A Splendor of Letters} (2003)\cite{basbanes:2003}
\end{quote}


\begin{quote}

"An image is worth a thousand words, but a spectrum is worth a
million," said lead author Ben R. Oppenheimer, associate curator and
chair of the Astrophysics Department at the American Museum of
Natural History.

 \url{http://www.facebook.com/photo.php?fbid=490576217674222&set=a.334832996581879.82450.334816523250193&type=1}
\end{quote}

\backmatter

\bibliographystyle{plain}
\bibliography{BooksRead}

\end{document}
