%% Begin copyright
%%
%%  /home/jrf/Documents/books/Books20/Docs/Collecting/Collecting.tex
%%  
%%   Part of the Books20 Project
%%
%%   Copyright 2017 James R. Fowler
%%
%%   All rights reserved. No part of this publication may be
%%   reproduced, stored in a retrival system, or transmitted
%%   in any form or by any means, electronic, mechanical,
%%   photocopying, recording, or otherwise, without prior written
%%   permission of the author.
%%
%%
%% End copyright

%%
%%   Notes on the books and articles that have been part of a series
%%   from a publisher.  This is information used
%%   with the book project "Some Important Books in Astronomy
%%    and Astrophysics in the 20th Century"
%%
\documentclass[letterpaper]{article}

\usepackage{books20}

\addglobalbib{./../../Docs/MasterBib.bib}

% need to set text width to soak up space
% otherwise is is soaked up in the right hand margin
% \paperwidth is 614.295pt (8.5in)

\begin{document}

%% Front matter
\title{Books Series in Astronomy and Astrophysics}
\author{James R. Fowler}
\date{created 1 Dec 2019\\ last updated 1 Dec 2019\\ copy of \today}

\maketitle

Why are books important? Historical importance for idea. Association
copies with individuals, institutions, or a place/time location.

Why are book important in \Cen{20} astronomy and astropysics? How
has that changed, if at all, over the century.

\section{Why do I collect books}

\begin{itemize}
\item I want to read them (but I haven't yet). I buy them faster than
  I can read them.
\item I want to have them as references
\item I am an introvert
\item Averse to throwing books away
\item Fantasy that if I have the book I have the knowledge. Only true
  if I read the book, remember the information, and integrate it with
  other information.
\end{itemize}

Gerald M. Cataldo in \bt{The Passion for Books} says
\begin{quotation}
  It is like no other collecting specialty because the books we collect
  surpass mere objects. The are alive with wonder, interest, education,
  pleasure, and contentment. We collect them not to possess objects, but
  to be caretakers of the human inquisitiveness, excellence, imagination,
  and achievement that created these magnificent works. These books are
  placed in our hands to be conserved and passed along, our contribution
  to the culture. \cite{Cataldo2010} page xii.
\end{quotation}

\section{Why is this book important to me?}

What is the focus of my collection?

\begin{itemize}
\item UT/UoC connections
\item Written by or association copies of astronomers I know
\item My subject/interest area
\end{itemize}

\section{Resources}

\section{Why Others Collect}

\section{How Others Collect}

From {\itshape Fine Books and Collection}, winter 2020 p59, collecter
Glynn Crain offers this advice to collectors, ``Get out there and meet
people, meet the collectors, meet the dealers and participate in
auctions and go to conventions. That's how you put a collection
together. It's really the only way to do it''.

\end{document}
