%% Begin copyright
%%
%%  /home/jrf/Documents/books/Books20/Docs/Collecting/Collecting.tex
%%  
%%   Part of the Books20 Project
%%
%%   Copyright 2019 James R. Fowler
%%
%%   All rights reserved. No part of this publication may be
%%   reproduced, stored in a retrival system, or transmitted
%%   in any form or by any means, electronic, mechanical,
%%   photocopying, recording, or otherwise, without prior written
%%   permission of the author.
%%
%%
%% End copyright

%%
%%   Notes on the process of collecting books. The whys and
%%   hows of the process.  This is information used
%%   with the book project "Some Important Books in Astronomy
%%    and Astrophysics in the 20th Century"
%%
\documentclass[letterpaper]{article}

\usepackage{books20}

\addglobalbib{./../../Docs/MasterBib.bib}

% need to set text width to soak up space
% otherwise is is soaked up in the right hand margin
% \paperwidth is 614.295pt (8.5in)

\begin{document}

%% Front matter
\title{Collecting Books; the whys and wherefores}
\author{James R. Fowler}
\date{created 1 Dec 2019\\ last updated 29 Dec 2019\\ copy of \today}

\maketitle

\begin{quotation}
  The sayings of the wise are like goads, and like nails firmly fixed
  are the collected sayings which are given by one Shepherd.  My son,
  beware of anything beyond these. Of making of many books there is no
  end, and much study is a weariness of the flesh

  Ecclesiates 12:11-12
\end{quotation}

Why are books important? Historical importance for idea. Association
copies with individuals, institutions, or a place/time location.

Why are book important in \Cen{20} astronomy and astrophysics? How
has that changed, if at all, over the century.

Books create a conversation over the years. They illustrate the thinking
at a particular time and place.

\section{Why do I collect books}

\begin{itemize}
\item I want to read them (but I haven't yet). I buy them faster than
  I can read them.
\item I want to have them as references
\item I am an introvert
\item Averse to throwing books away
\item Fantasy that if I have the book I have the knowledge. Only true
  if I read the book, remember the information, and integrate it with
  other information.
\item I find it fascinating to hold a book with some association
  to a past astronomer.
\end{itemize}


\section{Why is this book important to me?}

What is the focus of my collection?

\begin{itemize}
\item UT/UoC connections
\item Written by or association copies of astronomers I known
\item My subject/interest area
\end{itemize}

\section{Resources}

\begin{itemize}
\item Fine Books and Collections, collector basics and collector interviews.

\item John Carter, \bt{Taste and Technique in Book Collecting} (\cite{Carter1970}.

\item Nicholas A. Basbanes, \bt{A Gentle Madness} and \bt{Among the
  Gentle Mad}.
\end{itemize}

\section{Other Collections and Collectors}

\subsection{Dibner Collection, Smithsonian Institution}

Bern Dibner, engineer, businessman, and collector in the history of
science was interested in learning about the history of science. He
studied books and also took time off to research the history as
well. He donated 10,000 books to the Smithsonian Institution and is
the curator of \bt{Heralds of Science}
(\cite{Dibner1955}, \cite{Dibner1980}). He not only acquired rare early
books in science as well as incunabula, but also association copies.
He was inspired by Leonardo da Vinci's work as an artist and engineer.

He was inspired to create \bt{Heralds of Science} from the Herbert
McLean Evans exhibit in 1934 at the \Ord{94}{th} meeting of the American
Association for the Advancement of Science (AAAS) held at the Berkeley
(\cite{Dibner1980}, p5.)  There is a pamphlet that went with this
exhibit listing the books entitled \bt{Exhibition of First Editions of
  Epochal Achievements in the History of Science}. The selection
criteria for Dibner's books 'was made on the basis of the impact of
their message upon thought, the interpretation of the laws of nature,
and the introduction of industrial change and improvement'
(\cite{Dibner1980}, p6.)

\subsection{Hinkes Collection, Johns Hopkins Univ.}

The Hinkes Collection (\cite{Havens2011}) is a small collection, less
than 300 works, housed in the Sheridan Libraries at the Johns Hopkins
University. It is described as consisting of the ``Eureka'' moments in
the history of science, primarily physics and astronomy. However,
the \Cen{20} books do not seem to fit that category. There are a few
books on astronomy from \Cen{20} but books were never the ``Eureka''
moments in this time period.

The essays described the collection as not containing works by
compilers or synthesizers and this appears to be true.  My interest
however is in just such people as their works easily show the history
of the fields.  The interest is in ``normal science'' as opposed to
``revolutionary science'' as defined by Kuhn (Kuhn 19??). The essays
do make a note that these works are a conversation and tell a story
both about astronomy, Hinkes himself, and ourselves.  \Cen{2} books
don't really generate a conversation but the choosing of them can
tell a story.

There are no writing from Dr.~Hinkes himself and so we are unable to
identity his purposes in collection these works.  There is one quote
attributed to him that seems to show he collected in this interest
area but was not clear why he collected.  He did spend time taking
classes to better learn the various fields and to inform his
collecting.


\subsection{Biblotheca Osleriana}

One of the many libraries of Sir William Osler (1849--1919) resides at
McGill University.  Osler was a physician and teacher at a number of
medical schools and collected books throughout his life.  Usually
these collections and books would be donated to some institution in
order to free up shelf space in his house for the next round of
collecting.  The books he purchased would also be used in his classes
and he always worked some part of the history of the topic into his
lectures.  Towards the latter part of his life he conceived of
collecting a library which would have (a) a definite educational
value, (b) a literary, and (c) an historical interest. As always, to
divide such a collection into arbitrary divisions is a personal choice
and interpretation of the topics.  Such a division then becomes part
of the story that Osler wants to tell.  With boldness then he followed
his own plan and for the library and used the following divisions.

\begin{itemize}
\item[I.] Prima, which gives in chronological order a
  bio-bibliographical account of the evolution of science, including
  medicine.
\item[II.] Secunda, the works of men who have made notable
  contributions, or whose works have some special interest, but
  scarcely up to the mark of those in Prima.
\item[III.] Litteraria, the literary works written by medical men, and
  books dealing in a general way with doctors and the profession.
\item[IV.] Historica, with the stories of institutions, etc.
\item[V.] Biographica.
\item[VI.] Bibliographica.
\item[VII.] Incunabula, and
\item[VIII.] Manuscripts.
\end{itemize}

There are 7,787 numbered works in the index, \bt{Bibliotheca
  Osleriana} (\cite{Francis2000}), that was published by his friends
in 1929 following his death in 1919 and where the above information
was gathered.  While Osler followed the guidelines above, he also
included a number of personal association copies that had belonged to
favorite instructors or personal friends. He clearly had a love for
books as material and association objects.

As the index was originally published in 1929 and the focus is on
medicine there is little in the way of \Cen{20} works in astronomy or
astrophysics.  However, his Introduction, printed in the
\bt{Bibliotheca Osleriana}, is of value is deciding why and how he
collected. Most notably, he clearly had great fun finding and
collecting books.


\subsection{Miscellaneous Collectors}

From {\itshape Fine Books and Collection}, winter 2020 p59, collector
Glynn Crain offers this advice to collectors, ``Get out there and meet
people, meet the collectors, meet the dealers and participate in
auctions and go to conventions. That's how you put a collection
together. It's really the only way to do it''.

Gerald M. Cataldo in \bt{The Passion for Books} says about the
collecting of books,
\begin{quotation}
  It is like no other collecting specialty because the books we collect
  surpass mere objects. The are alive with wonder, interest, education,
  pleasure, and contentment. We collect them not to possess objects, but
  to be caretakers of the human inquisitiveness, excellence, imagination,
  and achievement that created these magnificent works. These books are
  placed in our hands to be conserved and passed along, our contribution
  to the culture (\cite{Cataldo2010} page xii.)
\end{quotation}

Zinner Collection, San Diego State University (\cite{Kenny1988}). Zinner
was a German astronomer and historian.  There is no documentation
on why or how he collected in this bibliography.

Lowes Collection, Renaissance Books of Science (\cite{Godine1970})
was collected for the importance of the work as well as for the
illustrations.

\subsection{Summary Notes}

A lot of the collectors cite money and/or availability as the reason
for the lack of a book in their collection.  Lowes in particular also
stated that while he collected high points, somehow the other works
muscled in and he did not object.

\begin{quotation}
The classics are too well known; they have been studied and restudied
until all the sap has been squeezed out of them. It is the secondary
and tertiary books that give color to collecting and make it fun.  Of
course, I tell myself, I don't {\itshape collect} these books; yet
here they are. Sometimes I think that {\itshape they} must collect
{\itshape me}\thinspace! (\cite{Godine1970}, p12.) (Emphasis in the
original.)
\end{quotation}

\printbibliography

\end{document}
