Davis S. Evans, in \emph{Lacaille: Astronomy, Traveler}, \cite{evans:1992}
gives a brief introduction and overview of \Cen{18} astronomy.

Lacaille 1713---1762

Spain was losing its empire, Britain and France were gaining ground.
Many wards between the commercial empires for control of trade.  To
maintain their control of these commercial empires nations need good
protection and good navigation.  The British Board for the Discovery
of Longtitude was created in 1714 but Royal Societies had be sent up
in France and England in the \Cen{17}.  Physics was becoming
mathematical with the publication of Newton's Principia follewed by
additional work in Classical Mechanics. Precession, nutation, and the
abberation of starlight were known effects but there quatitative
contributions were not yet know with enough accuracy.  The lunar
method and the occultations of the satellites of Jupiter were known
to be able to measure longitude but the accuracy of star catalogs and
orbits was not know well enough to provide the accuarcy required.

Evans lists four problems that dominated the \Cen{18}.

\begin{itemize}

\item Law of atmospheric refraction --- necessary for the accurate
  determination of latitude.

\item Figure of the earth --- length of a degree of latitude was necessary
  for accurate maps of the earth.

\item Distance to the planets --- essentially the length of one
  astronomical unit.  Best method would be transit of Venus.

\item Refinement of earth's orbital parameters --- for calculating
ephemerides for naviagation and for accurate star catalogs.

\end{itemize}

