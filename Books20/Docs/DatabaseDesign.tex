%% Based on a TeXnicCenter-Template by Gyorgy SZEIDL.
%%%%%%%%%%%%%%%%%%%%%%%%%%%%%%%%%%%%%%%%%%%%%%%%%%%%%%%%%%%%%
%%
%%  Some Important Books in Astronomy and Astrophysics 
%%   from the 20th Century
%%
%%   Copyright 2008 James R. Fowler
%%
%%   All rights reserved. No part of this publication may be
%%   reproduced, stored in a retrival system, or transmitted
%%   in any form or by any means, electronic, mechanical,
%%   photocopying, recording, or otherwise, without prior written
%%   permission of the author.
%% 
%%   The last known changes were checked in by $Author$
%%   as revision $LastChangedRevision$
%%   on $Date$
%%
%----------------------------------------------------------
%
\documentclass{article}%
%
%----------------------------------------------------------
%
\begin{document}

\title{Books20: Project Plan}
\author{James R. Fowler}
\maketitle
\vfil\eject

\vspace*{5 in}
Copyright \copyright 2009 by James R. Fowler

All rights reserved. No part of this publication may be
reproduced, stored in a retrival system, or transmitted
in any form or by any means, electronic, mechanical,
photocopying, recording, or otherwise, without prior written
permission of the author.

\vfil\eject

%
% Goals and Purpose of the Project
%
\section{Goals of the Project}
The primary goal of the Books20 project is to explore the question,
What are the important and influential books in astronomy and
astrophysics from the 20th Century? I come at this problem as a
working astronomer and bibliophile so my original intent was to
explore what were the collectable books from the 20th
Century. Important 20th Century books are generally less expensive
than the important books from the 19th or 18th Century.  Some of the
original questions included how to find out what books were published
during the 20th Century in astronomy and astrophysics and what makes a
book important? This led to a series of auxiliary questions such as,
how did astronomers use books during the 20th Century as information
transmission in astronomy transitioned from books to journals to the
web; what were publishers doing as the industry changed; what were the
important topics in astronomy during the 20th Century; how were books
used in the 19th Century and how was that different from the 20th
Century.  These questions lead me to research in the History of
Astronomy, Reading, Publishing and the Sociology of Science.

Although I had been thinking about these questions for many years the
formal start of the project was Nov 2008. With a full time job and
other volunteer activities I manage to get a few hours a week for the
project. During Nov '08 I spent the time thinking about the questions
and how to approach them.  I began looking for sources and references
and I took out a Community Borrowers card from the Sul Ross State
University library in Alpine.  The month of December '08 was spent in
the Sul Ross, McDonald Observatory, and University of Texas at Austin
libraries looking at books on the history of astronomy. The main
(re)discovery was of Astronomy and Astrophysics Abstracts and
Astronomischer Jahresbericht which lists all publications in astronomy
from 1899 to the current time. A complete set of these books are in
the McDonald Observatory library. During this time I also began a
draft of the project report, written in LaTeX, originally under
Windows Vista and TeXnic Center (www.texniccenter.org); this is a very
nice editor.  The draft gave a brief summary of the questions to ask,
a history of astronomy in the 19th and 20th Centuries, and a first
attempt at formatting the book list.  I began to have some idea of the
scope of what I was taking on.


%
% Design of the database
%
\section{Database Design}

The Books20 database is a relational database containing tables for
the basic book, author, and publisher information as well as a linking
tables between the books and authors.  The database is designed around
three views of the data, first, from the Web were a user can search
for information about a particular book or perform a search for a
particular class of books.  The second view is a reporting view. The
primary view is will be a sequential presentation of the books in
copyright sequence order with the output in TeX for incorporation in
the manuscript. Other views may be incorporated but these views are
principally for the project personel. The finally view is for the
editors.  It allows them to update information about a book, add a new
book, and change the status of a book from prospective to included.

\subsection{Database Tables}
\subsubsection{The Book Table}
  The Book table contains the basic publishing information about a
book.  Such items as title and subtitle, copyright year, the printing
edition, the publisher and the location of publication.  For books
that have one, the ISBN number is included.  ISBN number were
introduced in ???? and became manditory in ????.  The bibliographic
description should include the size of the book in millimeters, page
counts, whether the book is illustrated, figure count, table count and
index count.  This information should also include information about
the binding and about the dust jacket if the book was issued with one.

\subsubsection{The Author Table}

\subsubsection{The Publisher Table}

\subsubsection{The BookAuthor Table}


%
% Software
%
\section{Supporting Software}
The database software is MySql (www.mysql.com) with the interface written
in PhP (www.php.net).

MySql issues: security! internationalization, localization (see the MySql
developer web site)

%
% The End
%
\end{document}

%%
%%   $Log$
%%
%%
%%   $History$
%%
%%
%%
