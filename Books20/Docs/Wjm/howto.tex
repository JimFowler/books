%% Begin copyright
%%
%%  /home/jrf/Documents/books/Books20/Docs/Wjm/howto.tex
%%
%%   Part of the Books20 Project
%%
%%   Copyright 2023 James R. Fowler
%%
%%   All rights reserved. No part of this publication may be
%%   reproduced, stored in a retrival system, or transmitted
%%   in any form or by any means, electronic, mechanical,
%%   photocopying, recording, or otherwise, without prior written
%%   permission of the author.
%%
%%
%% End copyright

\documentclass{article}

\usepackage{books20}

\begin{document}

\title{How to Create wjm\_catalogue.pdf}
\author{James R. Fowler}
\date{3 March 2023}

\maketitle

\section{Dependencies}

The wjm\_catalogue.pdf file is dependent on
\begin{itemize}
  \item wjm\_catalogue.tex, biography.tex, library.tex,
  \item books.tex, author.ind.
\end{itemize}

\noindent
The production via the \LaTeX\ system is dependent on the packages
\begin{itemize}
  \item fontenc, graphicx, wrapfig,
  \item imakeidx, colophon,
  \item books20.
\end{itemize}


% create initial books.tex and index with catalogue_xml.py
\section{Creating the initial files}

\subsection{wjm\_books.xml}
The comma-separated variable file WJ\_McDonald\_library\_sorted.csv
contains the original 2017 inventory of the WJM collection updated
with the catalogue numbers from the 1971 inventory.  This file was
created from the original Excel\textsuperscript{\texttrademark}\ file,
WJ\_McDonald\_library.xlsx.
To convert this file to an XML file use \texttt{convert\_cvs.py}.
\begin{verbatim}
convert_cvs.py -o wjm_books.xml WJ_McDonald_library_sorted.csv
\end{verbatim}

\subsection{the\_books.tex}
To create books.tex and author.idx from scratch first run
\texttt{catalogue\_xml.py} to create the author2.idx and
the\_books.tex. Note that enty WJM 01.01(0).67 is not included
in the catalogue or index

\begin{verbatim}
python3 catalogue_xml.py --output the_books.tex --sort Year wjm_full.xml
\end{verbatim}
or
\begin{verbatim}
make the_books.tex
\end{verbatim}

\section{Editing the files}
% edit books.tex by hand
\subsection{Editing books.tex}
Copy the\_books.tex to books.tex and hand edit the file for final
production. This is necessary because creating software to deal with
all possible situation is difficult.  It is easier to edit the file by
hand.

\begin{itemize}
\item For long titles split as
\begin{verbatim}
      \textbf{The Jupiter Effect; The Planets as Triggers}
 
      \hspace{1em}\textbf{of Devastating Earthquakes}
\end{verbatim}

\item For long comments split as
\begin{verbatim}
    laid in with a letter from William Brunk, Chief Plantary Astronomy

     \hspace{1em}forwarding this complementary copy to WJM
\end{verbatim}

 \item replace three initials from A.B. C. with A.B.C.
   in author.index and books.tex.  Use the command
   
   \texttt{isearch-forward-regex "[A-Z]\.[A-Z]\. [A-Z]\."}

 \item replace upper right, lower right with upper-right, lower-right
   replace upper left, lower left with upper-left, lower-left.

 \item Look for any " quote characters at the begining and end of strings
       that have not been converted to LaTeX quotes

     \item Separate inscriptions on individual lines and indent by 2em
\begin{verbatim}
inscription on inside front cover

\hspace{2em}``In remb: of our trip together in the `City of Paris'

\hspace{2em}To Samuel Goodman Esqr.

\hspace{2em}with the best wishes of

\hspace{2em}T. Reginald Jones
\end{verbatim}

  \item Swap any entries where it make sense for the ordering. Usually
    putting a volume 1 before a volume 2.  Update the author.index
    file if any references will change.

\end{itemize}

\section{Create the Final Index}
Steps to creating the author.ind file
\begin{itemize}
  \item cp author2.idx to author.idx
  \item run 'makeindex author.idx'
  \item cp author.ind to author.index
  \item Commit author.index
\end{itemize}
\noindent
Edit the author.index file as follows,
\begin{itemize}
  \item replace three initials from A.B. C. with A.B.C.
    in author.index and books.tex.  Use the following command in Emacs
    to locate these occurrences
   
   \verb|isearch-forward-regex "[A-Z]\.[A-Z]\. [A-Z]d\."|

 \item replace \verb|\myhref{74--75}| style entries
   with \verb|\myhref{74}, \myhref{75}|
\end{itemize}
% create inventory94.tex

% make wjm_catalogue.pdf

\end{document}
